\documentclass[13pt,a4paper,oneside]{scrbook} %#try: report, article, book, amsart

%%
%%------------------- Eigne Kommandos
%%
%Kommando zum erzeugen eines deutlichen REMOVE Blocks
\newcommand{\tr}[1]{TOREMOVE-->\linebreak{#1} \linebreak <--TOREMOVE}

%Kommando zum erzeugen eines Index Wortes
\newcommand{\mi}[1]{\index{#1}#1}

%Kommando zum erzeugen einer freizeile
\renewcommand{\\}{\bigskip}

%Kommando zum einfügen und skalieren eines Bildes
\newcommand{\ig}[3]{
\begin{figure}[htbp]
	\centering
	\includegraphics[width=\linewidth,height=\textheight,keepaspectratio]{#1}%
	\caption[#2]{#3}%
\end{figure}
}

%%
%%------------------- Imports
%%
\RequirePackage{ifpdf}
\usepackage{fancyunits}
\usepackage[entwurf]{bhtThesis}
\usepackage{makeidx}
\usepackage{graphicx}
\usepackage{float}
\makeindex 

%%
%% Pfad zu den Bildern
%%
\graphicspath{
  {pictures/}
}

%\usepackage{makeidx}
%\makeindex


%%
%% Titel, Autor und Betreuer
%%
\version{0.1$\alpha$}
\datum{\today}
\fachbereich{VI -- Informatik und Medien --} 
\studiengang{Medieninformatik}
\autor{Adrian Randhahn}
\edvnr{744818}
\titel{Evaluierung von Techniken zur parallel-synchronen Bedienung einer Web-Applikation auf verschiedenen mobilen Endgeräten} 
\abschluss{Bachelor of Science (B.Sc.)}

\betreuerFeld{
  \begin{tabular}{lr}
    \multicolumn{2}{l}{\textbf{Gutachter}}\\
    Prof.~Knabe & Beuth Hochschule für Technik\\
    Prof.~Dr. Wambach & Beuth Hochschule für Technik
  \end{tabular}
}

\begin{document}
\pagestyle{fancy}

%%
%% ############# Deckblatt
%%
\maketitle

\pagenumbering{roman}

%%
%% ############# Erklärung
%%
\chapter*{Erklärung}
Ich  versichere, dass  ich diese  Abschlussarbeit ohne  fremde  Hilfe selbstständig
verfasst und  nur die  angegebenen Quellen und  Hilfsmittel benutzt  habe. Wörtlich
oder dem  Sinn nach  aus anderen  Werken entnommene Stellen  sind unter  Angabe der
Quellen kenntlich gemacht.
Ich erkläre weiterhin, dass die vorliegende Arbeit noch nicht im Rahmen eines anderen Prüfungsverfahrens eingereicht wurde.
\vspace{10ex}\\
\hrule
{\small{Datum}}\hfill{\small{Unterschrift}}

%%
%% ############# Datenschutz
%%
\chapter*{Sperrvermerk}
Die vorliegende Arbeit beinhaltet interne und vertrauliche Informationen der Firma New Image Systems GmbH. Die Weitergabe des Inhalts der Arbeit im Gesamten oder in Teilen sowie das Anfertigen von Kopien oder Abschriften - auch in digitaler Form - sind grundsätzlich untersagt. Ausnahmen bedürfen der schriftlichen Genehmigung der Firma New Image Systems GmbH.

%%
%% ############# Inhaltsverzeichnis
%%
\tableofcontents

%%
%% ############# Abbildungsverzeichnis
%%
\listoffigures

%%
%% ############# Tabellenverzeichnis
%%
\listoftables

%%
%% ############# Maincontent
%%
\pagenumbering{arabic}

%%
%% ############# Einleitung
%%
\chapter{Einleitung}
%\tr{Eine Einleitung bietet die Möglichkeiten den Sinn und Zweck der Diplomarbeit für einen Durchschnittsinformatiker (ohne die Spezialkenntnisse, die Sie jetzt haben) verständlich zu beschreiben. Hier können Sie Hintergründe darstellen, wie die Arbeit und das Thema entstanden und selbstverständlich für Ihre Arbeit werben. Interessierte Leser entscheiden hier, ob diese Arbeit für sie fachlich interessant ist.}

%\tr{	Eine kurze Beschreibung des allgemeinen Forschungsgebietes in ein bis zwei Absätzen. Die Einleitung sollte am Ende in ein bis zwei Sätzen die eigentlich untersuchte Fragestellung benennen.}
%Thematische Hinführung (2 Absätze); diese besteht im Idealfall aus dem Einstieg (etwas, was an die allgemeine Erfahrung anknüpft und unmittelbar ersichtlich ist) und einem weiteren Absatz, in dem – ausgehend vom 		Einstieg – auf das eigentliche Thema fokussiert wird. In einer Arbeit über Online-Marketing mit Facebook beispielsweise würde es im 1. Absatz um Online-Marketing allgemein gehen und im 2. Absatz auf die besonderen 		Anforderungen im Zusammenhang mit Facebook verwiesen. (Kontrollfrage: „Worum geht es hier?“)


%\tr{Hier sollte der Hintergrund und die Motivation der Arbeit kurz angerissen werden. Nach Möglichkeit sollte man alle Aspekte, die im zweiten Kapitel ("Grundlagen") besprochen werden, hier schon einmal ansprechen, damit 		diese nicht später aus heiterem Himmel fallen. Insbesondere sollte der Hintergrund quasi "zwingend" den nächsten Teil der Einleitung motivieren:

%\section{Hintergrund}	
In der modernen Webentwicklung durchläuft eine Anwendung verschiedene Etappen eines Entwicklungszykluses. Er beginnt bei einem Auftrag oder einer Idee, darauf folgt dann die Spezifikation einzelner \mi{Usecases}\footnote{Szenario oder auch Anwendungsfall}. Im Anschluss folgt in der Regel die Entwicklung und Implementation\footnote{Einbindung} der einzelnen Komponenten. Am Ende der jeweiligen Implementationsphase durchläuft das Produkt\footnote{hier: einzelne Softwarekomponente} die Qualitätskontrolle. Sollten in diesem Abschnitt Fehler auftreten wird das Produkt dem Entwickler zur erneuten Bearbeitung vorgelegt. 
\\
Dieser Vorgang kann sich beliebig oft wiederholen. Bei großen und komplexen Softwaresystemen ist es trotz zeitgemäßer Implementierung nicht immer Ausgeschlossen, dass \mi{Kaskadierungsfehler}\footnote{Fehler die nicht im eigentlichen Segment auftreten, sondern eine oder mehr Ebenen weiter unten in der Systemhirarchie} entstehen. Aus Sicht der Qualitätssicherung ist dies ein lästiges Problem, da diese nach jedem erneuten Modifikationsvorganges eines Softwaresegments einen größeren Segmentblock, wenn nicht sogar das gesamte Softwareystem erneut testen muss.
\\
Bei der Entwicklung auf und für mobile Endgeräte\footnote{Smartphones, Tabletts  oder Ähnliche} kommt noch ein erschwerender Faktor hinzu, nämlich die diversen, verschiedenen Bildschirmauflösungen. Diese können nicht nur die Darstellung des Inhaltes beeinflussen, sondern auch daraus folgend die Interaktionskonformität beeinflussen.

\ig{../pictures/Entwicklungsprozess}{Entwicklungsprozess}{Vereinfachte Darstellung eines Softwareentwicklungsprozesses}

Im Optimalfall wird die Software erst nach vollständiger Homogenität auf allen unterstützen Geräten freigegeben.

%\section{Problemstellung}
\\
Dieser zyklisch wiederkehrende Prozessablauf ist sehr Zeitintensiv und nimmt linear mit der Anzahl der zu testenden Geräte zu.

%\section{Operationalisierung der Fragestellung}
\\
Das Ergebnis dieser Forschungsarbeit soll zeigen, wie verschiedene \mi{Softwareframeworks} die Zeit, die in die Qualitätssicherung investiert wird, beeinflussen können, indem sie die Steuerung diverser Geräte parallel-synchron steuern. Die Evaluierung soll zeigen wo die Vorteile und Nachteile der einzelnen Werkzeuge liegen. Weiterhin soll gezeigt werden ob aktuelle \mi{Frameworks} erweiterbar sind um Beispielsweise automatisierte \mi{Testunits} zu implementieren.
%\section{Untersuchungsverlauf}

\subsubsection{Anmerkung}
Aus Gründen der besseren Lesbarkeit wird für alle Personen und Funktionsbezeichnungen durchgängig das generische Maskulinum angewendet und bezieht in gleicher Weise Frauen und Männer ein.

%%
%% ############# Aufgabenstellung
%%
\chapter{Aufgabenstellung}
%\tr{Durch eine klare Beschreibung der Aufgabenstellung wird die zu lösende Aufgabe deutlich. Vorhandene Teillösungen oder -systeme können hier ebenfalls dargestellt werden. In vielen Fällen ist es auch hilfreich Sachverhalte oder Problemstellungen zu beschreiben, die nicht zur Aufgabenstellung gehören (Abgrenzung).}
Die Aufgaben dieser Thesis ist die Evaluierung von Techniken zur parallel-synchronen Steuerung von Webapplikationen auf mobilen Endgeräten, um damit die Produktivität der Qualitätssicherung zu optimieren.
	%%
	%% ############# Annahmen und Einschränkungen
	%%
	\section{Problemstellung}
	%\tr{Aus dem Hintergrund sollte die Wissenslücke klar werden, die durch die Abschlussarbeit geschlossen werden soll. Kurz sollte beschrieben werden, mit welchen Methoden die Arbeit versuchen will, diese zu schließen 			(empirische 	Untersuchung, Loganalyse, neuartige Programmkomponenten, etc.). Schließlich sollte man herausstellen, warum es wichtig ist, diese Wissenslücke zu schließen (wie profitiert die Welt davon).}
	Ein Problem in der aktuellen Softwareentwicklung ist die immer mehr wachsende Anzahl an Endgeräten, welche mit verschiedenen Bildschirmauflösungen und eigenen Betriebsystemen in unterschiedlichen Versionen auftreten. Ein Qualitätsprüfer der einen hohen Qualitätsstandard hat investiert daher linear zu der Anzahl der zu testenden Geräte ansteigend Zeit, lediglich um vereinzelte Testszenarien durchzuarbeiten. Solch ein Testszenraio kann Navigationsabläufe\footnote{ein Nutzerspezifischer Gang durch die Webseite}, das ausfüllen und validieren eines Formular oder auch das überprüfen funktionaler\footnote{aktive Links und deren Aufruf} Links sein. Bereits an dieser Stelle ist die zu investierende Zeit, und dies wiederholt, enorm.
\\
Wenn der Qualitätsprüfer innerhalb eines Testszenarios einen schwerwiegenden Fehler bei einem der Geräte entdeckt, muss dieser den Vorgang beenden. Abgebrochen werden muss deshalb, da bei korrigierter Implementierung der Qualitätsprüfer nicht davon ausgehen darf, das bereits kontrollierte Abschnitte immernoch voll funktionsfähig sind, da eventuell neue Fehler in bereits Kontrollierten Segmenten auftreten können.
\\
Sollte ein Szenario aufgrund eines Fehler abgebrochen worden sein, wird dem Entwickler das Problem möglichst konkret geschildert. Dessen Aufgabe ist es nun das Problem zu beheben. Ist dies geschehen startet der Prüfer einen erneuten Durchgang des Szenarios. Ein generelles Problem was hier noch zusätzlich entstehen kann, ist der Umstand, dass sich grade bei nur kleineren fixes\footnote{Problemlösungen, Codeanpassungen} und immer wieder auftretenden Testszenarioschleifen eine gewisse Routine einschleichen kann, worunter die Qualität des Produkts leidet.

\ig{../pictures/Testszenario}{Qualitätssicherung Testszenario}{Darstellung eines Qualitätssicherungsablaufes in der mobilen Anwendungsentwicklung}}
\pagebreak

	%%
	%% ############# Annahmen und Einschränkungen
	%%
	\section{Annahmen und Einschränkungen}
	%\tr{Wenn die Arbeit wichtige Annahmen trifft, unter denen die Untersuchungsergebnisse gültig sind, oder die Allgemeinheit der getroffenen Aussagen wichtigen Einschränkungen unterliegt, sollten diese ebenfalls in der 		Einleitung 	beschrieben werden.}
	
	%%
	%% ############# Zielsetzung
	%%
	\section{Zielsetzung}
	Das Ziel dieser Arbeit ist es, bestehende \mi{Frameworks} auf ihre Tauglichkeit in Bezug auf die parallel-synchrone Steuerung von mobilen Endgeräten zur Durchführung von Testszenarien zu evaulieren.
	%%
	%% ############# Abgrenzungskriterien
	%%
	\section{Abgrenzungskriterien}
	%\tr{Hier werden die Grundlagen für das zu entwickelnde Softwaresystem definiert. Zwar noch aus fachtechnischer Sicht werden hier die Anforderungen an das geplante Softwaresystem in möglichst formaler Form spezifiziert. 	Es 	sollen hier keine Lösungen präsentiert werden, sondern möglichst präzise die Anforderungen (Sollkonzept) an das geplante Softwaresystem mit seinen Schnittstellen, Informationsflüssen und Systemfunktionen 			dokumentiert 	werden. Verwendete Methoden können z.B. SA, SADT, Petri-Netze oder andere sein. Das Ergebnis ist ein für die Systementwicklung verwendbares Pflichtenheft. Über Art und Umfang des Pflichtenhefts sollten 	Sie mit Ihrem Betreuer sprechen.}

%%
%% ############# Grundlagen
%%
\chapter{Grundlagen}
	In diesem Abschnitt behandle ich spezifische Definitionen wie zum Beispiel verwendetes Fachvokabular, allgemeine technische Abläufe die Notwendig sind um diese Arbeit und die darin verwendetet Techniken zu verstehen, sowie verwendete Hardwarekomponenten.
%\tr{Dieser Teil beschreibt das fachliche Umfeld der Aufgabenstellung. Hier werden die wesentlichen fachlichen Begrifflichkeiten, die für die Aufgabe relevanten Problemstellungen und Lösungsansätze des Fachgebietes vorgestellt. Der Sprachgebrauch sollte einen direkten Bezug zum Fachgebiet haben. Die notwendigen Darstellungsmethoden, die Art und der Umfang der Beschreibung hängen wesentlich von der jeweiligen Fachdisziplin ab und sollten im Dialog mit dem Betreuer entschieden werden. Beispielsweise wird sich die Beschreibung eines Hotelreservierungssystems sehr von einer Beschreibung mathematischen Transformationen auf Grafikobjekte unterscheiden.Dies ist oft vor der Einleitung das erste Kapitel, das man schreibt, und sollte einen Überblick über die Literatur und existierende Arbeiten im Bereich der Arbeit liefern (Welche Grundlagen gibt es in diesem Bereich? Haben andere Autoren schon etwas zu verwandten Themen veröffentlicht?). Die hier vorgestellten Konzepte sollten in der Einleitung zumindest schon einmal angesprochen worden sein. Bei der Vorstellung verwandter Arbeiten sollten neuere Erkenntnisse bevorzugt werden. Bei jeder in das Grundlagenkapitel aufgenommenen Arbeit gilt es herauszustellen, was die Ergebnisse der Arbeit waren und warum diese Ergebnisse (oder Einschränkungen der vorgestellten Arbeit) eben noch keine Schließung der Wissenslücke oder keine Lösung der Aufgabenstellung darstellen? Am Ende erfolgt eine kurze Zusammenfassung der Grundlagen, die begründete Schlussfolgerung, dass das zu untersuchende Problem noch ungelöst ist, und ggf. wieder eine Vorschau auf das folgende Kapitel.}
	%%
	%% ############# Begriffsklärung
	%%
	\section{Begriffsklärung}	
		\subsection{parallel-synchron}
		\subsection{Web-Applikation}
		\subsection{Endgeräte}
	%%
	%% ############# technischer Aufbau
	%%
	\section{technischer Aufbau}
	%%
	%% ############# Komponenten
	%%
	\section{Komponenten}
		\subsection{\mi{Raspberry Pi}}
		\subsection{Hardware}

%%
%% ############# Technologien
%%
\chapter{Technologien}
	\section{\mi{Ghostlab}}
	\section{\mi{NodeJS}}
	\section{\mi{Zombie.js}}
	\section{\mi{W3C Touch Events Extensions}}
	\section{\mi{Phantom Limb}}
	\section{\mi{jQuery} \mi{UI Touch Punch}}
	\section{\mi{jQuery} \mi{Touchit}}
	\section{\mi{NPM} \mi{touchit}}

%%
%% ############# Lösungsansätze
%%
%\tr{Nach der Beschreibung des fachlichen Umfeldes können ausgewählte (vielleicht alle?) für die Aufgabenstellung relevanten Probleme vorgestellt, Vor- und Nachteile bestehender Lösungen argumentiert und die voraussichtlich angestrebte (weil vorteilhafte) Lösung herausgestellt werden.}
\chapter{Lösungsansätze}
	\section{\mi{Adobe Edge Inspect}}
	\section{\mi{Ghostlab}}
	\section{\mi{Remote Preview}}
	\section{\mi{Browser-Sync}}
	\section{Eigenes \mi{Framework}}
		
%%
%% ############# Evaluation
%%
\chapter{Evaluation der Techniken}
	\section{\mi{Adobe Edge Inspect}}
	\section{\mi{Ghostlab}}
	\section{\mi{Remote Preview}}
	\section{\mi{Browser-Sync}}
	\section{Eigenes \mi{Framework}}
		\subsection{\mi{Systementwurf}}
		\subsubsection{\mi{Ablaufdiagram}}
		\subsubsection{\mi{Klassendiagramm}}

\chapter{Ausblick}

%%
%% ############# Helpers
%%
\chapter{Helpers}
	\section{quote} 
		\begin{quote}
			Dies ist ein Zitat.
		\end{quote}
	\section{longquote} 	
		\begin{quotation}
			Dies ist ein längeres Zitat.
		\end{quotation} 
	\section{fussnote}
		Dies ist der Text\footnote{Und dies ist die Fußnote dazu.}


\printindex

\end{document}



