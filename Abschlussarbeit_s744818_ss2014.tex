\documentclass[13pt,a4paper,oneside]{scrbook} %#try: report, article, book, amsart

%%
%%------------------- Eigne Kommandos
%%
%Kommando zum erzeugen eines deutlichen REMOVE Blocks
\newcommand{\tr}[1]{TOREMOVE-->\linebreak{#1} \linebreak <--TOREMOVE}

%Kommando zum erzeugen eines Index Wortes
\newcommand{\mi}[1]{\index{#1}#1}

%Kommando zum erzeugen einer freizeile
\renewcommand{\\}{\bigskip}

%Kommando zum einfügen und skalieren eines Bildes
\newcommand{\ig}[3]{
\begin{figure}[htbp]
	\centering
	\includegraphics[width=400,keepaspectratio]{#1}%
	\caption[#2]{#3}%
\end{figure}
}

%Kommando zur erstellung eines Zitats
\newcommand{\z}[2]{
	\begin{quote}
			" #1 " \cite{#2}
	\end{quote}
	
} 

\newcommand{\iga}[3]{
\begin{figure}[H]
	\centering
	\includegraphics[keepaspectratio]{#1}%
	\caption[#2]{#3}%
\end{figure}
}

\newcommand{\igp}[5]{
\begin{figure}[H]
	\centering
	\includegraphics[width=#4, height=#5,,keepaspectratio]{#1}%
	\caption[#2]{#3}%
\end{figure}
}

%Kommando zur Erzeugung der evaluationstablle
\usepackage{fp} 
\usepackage[table]{xcolor}

\newcommand{\met}[9]{

	\begin{table}[H]
	 \vspace{-20pt}
 		\centering
		\rowcolors{1}{white}{lightgray}
			\begin{tabular}{| p{8cm} | p{2cm} | p{2cm} |}
			\hline
				Komponente		 	&	Punkte	&	Wertigkeit\\
			\hline
			\hline
				Installation			&	#2	&	10 \%\\
				Konfiguration			&	#3	&	10 \%\\
				Funktion: Desktop		&	#4	&	25 \%\\
				Funktion: Mobil			&	#5	&	25 \%\\
				Erweiterbarkeit			&	#6	&	10 \%\\
				unterst{\"u}tze Browser				&	#7	&	10 \%\\
				Aktivit{\"a}t			&	#8	&	10 \%\\
				\hline
				\hline
				Gesamt				&	\FPeval\resultfp{clip((#2*1)+(#3*1)+(#4*2.5)+(#5*2.5)+(#6*1)+(#7*1)+(#8*1))} 												\resultfp 	&	100 \%\\
				\hline
				\end{tabular}
			\caption{#1}
	\end{table}

}

\newcommand{\mmet}[6]{
	\subsubsection{#1}
	 \vspace{-20pt}
	\begin{table}[H]
 		\centering
		\rowcolors{1}{white}{lightgray}
			\begin{tabular}{| p{8cm} | p{5cm} |}
			\hline
				Komponente		 &	\\
			\hline
			\hline
				Betriebssystem				&	#2	\\
				Versionsnummer			&	#3	\\
				Bildschirmdiagonale			&	#4	\\
				Aufl{\"o}sung				&	#5	\\
				priem{\"a}re Ausrichtung		&	#6	\\
				\end{tabular}
			\caption{{\"U}bersicht #1}
	\end{table}

}

%%
%%------------------- Imports
%%
\RequirePackage{ifpdf}
\usepackage{fancyunits}
\usepackage[entwurf]{bhtThesis}
\usepackage{makeidx}
\usepackage{graphicx}
\usepackage{float}
\usepackage[toc]{glossaries}

\makeindex 


 
\makeglossaries
 
 
%%
%% ======================= GLOSSAR ANFANG
%% 
 
\newglossaryentry{Ajax}
{
    name=Ajax,
    description={\z{Ajax (Apronym von engl. Asynchronous JavaScript and XML) bezeichnet ein Konzept der asynchronen 				Datenübertragung zwischen einem Browser und dem Server. Dieses ermöglicht es, HTTP-Anfragen durchzuführen, während 		eine HTML-Seite angezeigt wird, und die Seite zu verändern, ohne sie komplett neu zu laden}{2}
	}
}

 \newglossaryentry{moEn}
{
    name=mobiles Endgerät,
    description={Komponenten mit primärer mobiler Nutzung werden umfassend als mobile Endgeräte gruppiert. Hierzu zählen 			Smarthphones, Tablets, sowie das Microsoft Surface}
}

 \newglossaryentry{Framework}
{
    name=Framework,
    description={\z{Ein Framework ist eine semi-vollständige Applikation. Es stellt für Applikationen eine wiederverwendbare, 			gemeinsame Struktur zur Verfügung. Die Entwickler bauen das Framework in ihre eigene Applikation ein, und erweitern es 		derart, dass es ihren spezifischen Anforderungen entspricht. Frameworks unterscheiden sich von Toolkits dahingehend, dass sie 	eine kohärente Struktur zur Verfügung stellen, anstatt einer einfachen Menge von Hilfsklassen.}{1}
    Der Einfachheit halber wurden Sammlungen die nach dieser Definition eventuell unter den Begriff eines Toolkits fallen, ebenfalls als Framework betitelt
	}
}
 
\newglossaryentry{Webbrowser}
{
    name=Webbrowser,
    description={Computerprogramm zur Darstellung von Inhalten des World Wide Web}
}

\newglossaryentry{HTML}
{
    name=HTML,
    description={Die Hypertext Markup Language ist eine Auszeichnungsprache zur Beschreibung von Inhalten. Sie dient der Strukturierung von Texten, Links\footnote{Verweise zu anderen Inhalten}, Listen und Bildern eines Dokumentes. Eine HTML Seite wird von einem Webbrowser interpretiert und anschließend dargestellt. Die Entwicklung von HTML geschieht durch das World Wide Web Consortium(W3C) und den Web Hypertext Application Technology Working Group (WHATWG) }
}

\newglossaryentry{Javascript}
{
    name=Javascript,
    description={\z{JavaScript (kurz JS) ist eine Skriptsprache, die ursprünglich für dynamisches HTML in Webbrowsern entwickelt wurde, um Benutzerinteraktionen auszuwerten, Inhalte zu verändern, nachzuladen oder zu generieren und so die Möglichkeiten von HTML und CSS zu erweitern. Heute findet JavaScript auch außerhalb von Browsern Anwendung, so etwa auf Servern und in Microcontrollern}{4}}
}

\newglossaryentry{NodeJS}
{
    name=NodeJS,
    description={Plattform um serverseitige eventgesteuerte Javascriptanwendungen zu entwickeln}
}

\newglossaryentry{PHP}
{
    name=PHP,
    description={Eine an C und Perl angelehnte Scriptsprache für dynamische Webseiten}
}

\newglossaryentry{NPM}
{
    name=NPM,
    description={Node Packaged Modules, eine Software zur Installation von NodeJS Bibliotheken}
}

\newglossaryentry{qs}
{
    name=Qualitätssicherung,
    description={Station, welche ein Produkt (hier die Anwendung) durchlaufen und bestehen muss um eine gewisse Qualität zu gewährleisten}
}

\newglossaryentry{VirtualBox}
{
    name=VirtualBox,
    description={Virtuelle Desktopumgebung von Oracle. Wird genutzt um zum Beispiel ein anderes Betriebssystem als das eigentlich genutzte zu Emulieren}
}

\newglossaryentry{Smartphone}
{
    name=Smartphone,
    description={Mobiltelefon mit Computerähnlicher Struktur, meist mit Touchdisplay ausgestattet}
}

\newglossaryentry{Tablet}
{
    name=Tablet,
    description={tragbarer flacher Computer mit einem Touchscreen}
}

\newglossaryentry{PoW}
{
    name=Portrait View,
    description={Vertikalausrichtung des Bildschirms eines mobilen Endgerätes}
}

\newglossaryentry{PaW}
{
    name=Panorama View,
    description={Horizontalausrichtung des Bildschirms eines mobilen Endgerätes}
}

\newglossaryentry{Pixel}
{
    name=Pixel,
    description={Auch bekannt als Bildpunkt. Farbwert einer digitalen Rastergrafik}
}

\newglossaryentry{BA}
{
    name=Bildauflösung,
    description={\z{Die Bildauflösung ist ein umgangssprachliches Maß für die Bildgröße einer Rastergrafik. Sie wird durch die Gesamtzahl der Bildpunkte oder durch die Anzahl der Spalten (Breite) und Zeilen (Höhe) einer Rastergrafik angegeben.}{5}}
}



%%
%% ======================= GLOSSAR ENDE
%%


\usepackage{tikz}
\usetikzlibrary{calc}

%%
%% Pfad zu den Bildern
%%
\graphicspath{
  {pictures/}
}

%%
%% Titel, Autor und Betreuer
%%
\version{1.4}
\datum{\today}
\fachbereich{VI -- Informatik und Medien --} 
\studiengang{Medieninformatik}
\autor{Adrian Randhahn}
\edvnr{744818}
\titel{Evaluierung von Techniken zur parallel-synchronen Bedienung einer Web-Applikation auf verschiedenen mobilen Endgeräten} 
\abschluss{Bachelor of Science (B.Sc.)}

\betreuerFeld{
  \begin{tabular}{lr}
    \multicolumn{2}{l}{\textbf{Gutachter}}\\
    Prof. Dipl.-Inform.~Knabe & Beuth Hochschule für Technik\\
    Prof.~Dr.-Ing. Wambach & Beuth Hochschule für Technik
  \end{tabular}
}

\begin{document}

\pagestyle{fancy}

%%
%% ############# Deckblatt
%%
\maketitle
\chapter* { }
\pagenumbering{roman}

%%
%% ############# Erklärung
%%
\chapter*{Erklärung}
Ich  versichere, dass  ich diese  Abschlussarbeit ohne  fremde  Hilfe selbstständig
verfasst und  nur die  angegebenen Quellen und  Hilfsmittel benutzt  habe. Wörtlich
oder dem  Sinn nach  aus anderen  Werken entnommene Stellen  sind unter  Angabe der
Quellen kenntlich gemacht.
Ich erkläre weiterhin, dass die vorliegende Arbeit noch nicht im Rahmen eines anderen Prüfungsverfahrens eingereicht wurde.
\vspace{10ex} \\
\hrule
\\
{\small{Datum}}\hfill{\small{Unterschrift}}

%%
%% ############# Datenschutz
%%
\section*{Sperrvermerk}
Die vorliegende Arbeit beinhaltet interne und vertrauliche Informationen der Firma New Image Systems GmbH. Die Weitergabe des Inhalts der Arbeit im Gesamten oder in Teilen sowie das Anfertigen von Kopien oder Abschriften - auch in digitaler Form - sind grundsätzlich untersagt. Ausnahmen bedürfen der schriftlichen Genehmigung der Firma New Image Systems GmbH.
\null
\vfill
\section*{Rechtliches}
Alle in dieser Arbeit genannten Unternehmens- und Produktbezeichnungen sind in der Regel geschützte Marken- oder Warenzeichen. Auch ohne besondere Kennzeichnung sind diese nicht frei von Rechten Dritter zu betrachten. Alle erwähnten Marken- oder Warenzeichen unterliegen uneingeschränkt der länderspezifischen Schutzbestimmungen und den Besitzrechten der jeweiligen eingetragenen Eigentümern.

%%
%% ############# Abstract
%%
\section*{Kurzfassung}
Im Zuge dieser Arbeit werden verschiedene Frameworks auf ihre Fähigkeit hin untersucht, eine Webapplikation \gls{parallel-synchron}, vorzugsweise auf mobilen Endgeräten, zu steuern. Das Ziel der Arbeit ist es, einen Überblick über die sich am Markt befindlichen Werkzeuge zu erbringen, die geeignet scheinen, die Arbeit des Qualitätsmanagements für Webapplikationen hinsichtlich Zeit und Qualität zu optimieren.

\\Die \Gls{Framework}s werden im Allgemeinen daraufhin untersucht, ob sie kompatibel mit Desktop-, sowie Mobilbrowsern sind. Ein Teil der Frameworks wirbt mit einer Vielzahl an Funktionen, wohingegen andere nur Teilfunktionalitäten abdecken. Letztere werden als Einzelframeworks bezeichnet. Des Weiteren werden diese Einzelframeworks dahingehend analysiert, ob sie in kombinierter Form in der Lage sind, ein eigenständiges \Gls{Framework} zu bilden.

\\ Zu Zwecken der Evaluation wurden verschiedene Kriterien wie die Installation und Konfiguration, die Erweiterbarkeit oder der Browsersupport festgelegt und vom Autor gewichtet. Die einzelnen Technologien werden kurz vorgestellt, deren Installationsprozess beschrieben und einer Reihe von Tests unterzogen. Nach erfolgreicher Evaluation werden die Ergebnisse noch einmal zusammengefasst und tabellarisch gegenübergestellt.


%
%\\ \\ Während der Entstehung einer Webapplikation durchläuft diese wiederholt die Qualitätskontrolle. Innerhalb dieser Arbeit werden bestehende Technologien auf deren Nutzen hin untersucht die Qualitätssicherung qualitativ zu verbessern. Die \Gls{Framework}s werden auf ihre Kompatibilität für Desktop-, sowie Mobilbrowser untersucht. Des Weiteren werden alleinstehende Frameworks, dahin gehend untersucht, ob sie in kombinierter Form in der Lage sind ein eigenständiges \Gls{Framework} zu erschaffen.
%

%\section*{Abstract}
%During the origin of a web application this runs through repeats the high-class control. Within this work existing technologies are examined for their use to improve the quality assurance qualitatively. Besides become single Frameworks, passing examined whether they in combined form in the situation are to be fulfilled the demanded criteria.

%%
%% ############# Inhaltsverzeichnis
%%
\tableofcontents

%%
%% ############# Abbildungsverzeichnis
%%
\listoffigures

%%
%% ############# Tabellenverzeichnis
%%
\listoftables

%%
%% ############# Maincontent
%%
\pagenumbering{arabic}

%%
%% ############# Einleitung
%%
\chapter{Einleitung}
%\tr{Eine Einleitung bietet die Möglichkeiten den Sinn und Zweck der Diplomarbeit für einen Durchschnittsinformatiker (ohne die Spezialkenntnisse, die Sie jetzt haben) verständlich zu beschreiben. Hier können Sie Hintergründe darstellen, wie die Arbeit und das Thema entstanden und selbstverständlich für Ihre Arbeit werben. Interessierte Leser entscheiden hier, ob diese Arbeit für sie fachlich interessant ist.}

%\tr{	Eine kurze Beschreibung des allgemeinen Forschungsgebietes in ein bis zwei Absätzen. Die Einleitung sollte am Ende in ein bis zwei Sätzen die eigentlich untersuchte Fragestellung benennen.}
%Thematische Hinführung (2 Absätze); diese besteht im Idealfall aus dem Einstieg (etwas, was an die allgemeine Erfahrung anknüpft und unmittelbar ersichtlich ist) und einem weiteren Absatz, in dem – ausgehend vom 		Einstieg – auf das eigentliche Thema fokussiert wird. In einer Arbeit über Online-Marketing mit Facebook beispielsweise würde es im 1. Absatz um Online-Marketing allgemein gehen und im 2. Absatz auf die besonderen 		Anforderungen im Zusammenhang mit Facebook verwiesen. (Kontrollfrage: „Worum geht es hier?“)


%\tr{Hier sollte der Hintergrund und die Motivation der Arbeit kurz angerissen werden. Nach Möglichkeit sollte man alle Aspekte, die im zweiten Kapitel ("Grundlagen") besprochen werden, hier schon einmal ansprechen, damit 		diese nicht später aus heiterem Himmel fallen. Insbesondere sollte der Hintergrund quasi "zwingend" den nächsten Teil der Einleitung motivieren:

%\section{Hintergrund}	
In der modernen Webentwicklung durchläuft eine Anwendung verschiedene Etappen eines Entwicklungszykluses. Er beginnt bei einem Auftrag oder einer Idee, darauf folgt dann die Spezifikation einzelner \mi{Usecases}\footnote{Szenario oder auch Anwendungsfall}. Im Anschluss folgt in der Regel die Entwicklung und Implementation\footnote{Einbindung} der einzelnen Komponenten. Am Ende der jeweiligen Implementationsphase durchläuft das Produkt\footnote{hier: einzelne Softwarekomponente} die Qualitätskontrolle. Sollten in diesem Abschnitt Fehler auftreten wird das Produkt dem Entwickler zur erneuten Bearbeitung vorgelegt. 
\\
Dieser Vorgang kann sich beliebig oft wiederholen. Bei großen und komplexen Softwaresystemen ist es trotz zeitgemäßer Implementierung nicht immer Ausgeschlossen, dass \mi{Kaskadierungsfehler}\footnote{Fehler die nicht im eigentlichen Segment auftreten, sondern eine oder mehr Ebenen weiter unten in der Systemhirarchie} entstehen. Aus Sicht der Qualitätssicherung ist dies ein lästiges Problem, da diese nach jedem erneuten Modifikationsvorganges eines Softwaresegments einen größeren Segmentblock, wenn nicht sogar das gesamte Softwareystem erneut testen muss.
\\
Bei der Entwicklung auf und für mobile Endgeräte\footnote{Smartphones, Tabletts  oder Ähnliche} kommt noch ein erschwerender Faktor hinzu, nämlich die diversen, verschiedenen Bildschirmauflösungen. Diese können nicht nur die Darstellung des Inhaltes beeinflussen, sondern auch daraus folgend die Interaktionskonformität beeinflussen.

\ig{../pictures/Entwicklungsprozess}{Entwicklungsprozess}{Vereinfachte Darstellung eines Softwareentwicklungsprozesses}

Im Optimalfall wird die Software erst nach vollständiger Homogenität auf allen unterstützen Geräten freigegeben.

%\section{Problemstellung}
\\
Dieser zyklisch wiederkehrende Prozessablauf ist sehr Zeitintensiv und nimmt linear mit der Anzahl der zu testenden Geräte zu.

%\section{Operationalisierung der Fragestellung}
\\
Das Ergebnis dieser Forschungsarbeit soll zeigen, wie verschiedene \mi{Softwareframeworks} die Zeit, die in die Qualitätssicherung investiert wird, beeinflussen können, indem sie die Steuerung diverser Geräte parallel-synchron steuern. Die Evaluierung soll zeigen wo die Vorteile und Nachteile der einzelnen Werkzeuge liegen. Weiterhin soll gezeigt werden ob aktuelle \mi{Frameworks} erweiterbar sind um Beispielsweise automatisierte \mi{Testunits} zu implementieren.
%\section{Untersuchungsverlauf}

\subsubsection{Anmerkung}
Aus Gründen der besseren Lesbarkeit wird für alle Personen und Funktionsbezeichnungen durchgängig das generische Maskulinum angewendet und bezieht in gleicher Weise Frauen und Männer ein.

%%
%% ############# Aufgabenstellung
%%
\chapter{Aufgabenstellung}
%\tr{Durch eine klare Beschreibung der Aufgabenstellung wird die zu lösende Aufgabe deutlich. Vorhandene Teillösungen oder -systeme können hier ebenfalls dargestellt werden. In vielen Fällen ist es auch hilfreich Sachverhalte oder Problemstellungen zu beschreiben, die nicht zur Aufgabenstellung gehören (Abgrenzung).}
Die Aufgaben dieser Thesis ist die Evaluierung von Techniken zur parallel-synchronen Steuerung von Webapplikationen auf mobilen Endgeräten, um damit die Produktivität der Qualitätssicherung zu optimieren.
	%%
	%% ############# Annahmen und Einschränkungen
	%%
	\section{Problemstellung}
	%\tr{Aus dem Hintergrund sollte die Wissenslücke klar werden, die durch die Abschlussarbeit geschlossen werden soll. Kurz 		sollte beschrieben werden, mit welchen Methoden die Arbeit versuchen will, diese zu schließen 			(empirische 		Untersuchung, Loganalyse, neuartige Programmkomponenten, etc.). Schließlich sollte man herausstellen, warum es wichtig ist, 	diese Wissenslücke zu schließen (wie profitiert die Welt davon).}
	Ein Problem in der aktuellen Softwareentwicklung ist die immer mehr wachsende Anzahl an Endgeräten, welche mit 			verschiedenen Bildschirmauflösungen und eigenen Betriebsystemen in unterschiedlichen Versionen auftreten. Ein 				Qualitätsprüfer der einen hohen Qualitätsstandard hat investiert daher linear zu der Anzahl der zu testenden Geräte ansteigend 	Zeit, lediglich um vereinzelte Testszenarien durchzuarbeiten. Solch ein Testszenraio kann Navigationsabläufe\footnote{ein 		Nutzerspezifischer Gang durch die Webseite}, das ausfüllen und validieren eines Formular oder auch das überprüfen 			funktionaler\footnote{aktive Links und deren Aufruf} Links sein. Bereits an dieser Stelle ist die zu investierende Zeit, und dies 		wiederholt, enorm.
	\\
	Wenn der Qualitätsprüfer innerhalb eines Testszenarios einen schwerwiegenden Fehler bei einem der Geräte entdeckt, muss 		dieser den Vorgang beenden. Abgebrochen werden muss deshalb, da bei korrigierter Implementierung der Qualitätsprüfer nicht 	davon ausgehen darf, das bereits kontrollierte Abschnitte immer noch voll funktionsfähig sind, da eventuell neue Fehler in bereits 	Kontrollierten Segmenten auftreten können.
	\\
	Sollte ein Szenario aufgrund eines Fehler abgebrochen worden sein, wird dem Entwickler das Problem möglichst konkret 		geschildert. Dessen Aufgabe ist es nun das Problem zu beheben. Ist dies geschehen startet der Prüfer einen erneuten 			Durchgang des Szenarios. Ein generelles Problem was hier noch zusätzlich entstehen kann, ist der Umstand, dass sich gerade 		bei nur kleineren fixes\footnote{Problemlösungen, Codeanpassungen} und immer wieder auftretenden Testszenarioschleifen 		eine gewisse Routine einschleichen kann, worunter die Qualität des Produkts leidet.

	\ig{../pictures/Testszenario}{Qualitätssicherung Testszenario}{Darstellung eines Qualitätssicherungsablaufes in der mobilen 		Anwendungsentwicklung}}
	\pagebreak
	
	%%
	%% ############# Annahmen und Einschränkungen
	%%
	
	%\section{Annahmen und Einschränkungen}
	%\tr{Wenn die Arbeit wichtige Annahmen trifft, unter denen die Untersuchungsergebnisse gültig sind, oder die Allgemeinheit der 	getroffenen Aussagen wichtigen Einschränkungen unterliegt, sollten diese ebenfalls in der 		Einleitung 	beschrieben 		werden.}
	
	%%
	%% ############# Zielsetzung
	%%
	\section{Zielsetzung}
	 Das Ziel dieser Arbeit ist es, bestehende \mi{Frameworks} auf ihre Tauglichkeit in Bezug auf die parallel-synchrone Steuerung 	von mobilen Endgeräten zur Durchführung von Testszenarien zu evaluieren.
	Hierzu werden auf mobilen Endgeräten die internen Browser getestet. Hinzu kommen auf Desktopgeräten die aktuellen 			Versionen von Firefox, Chrome, Safari(nur für Mac-Desktopgeräte) und der Internet Explorer(nur für Windows-Desktops). Um 		eine Allgemeine Testbarkeit zu gewährleisten werden die Frameworks auch auf Genauigkeit in virtuellen Umgebungen analysiert. 	Dabei können Abweichungen, seien sie noch so klein, entstehen. Bereits 1 Pixel Abweichung kann bereits ausschlaggebend 		sein einen Umbruch zu erzeugen und damit das Layout negativ zu verändern.
	%%
	%% ############# Abgrenzungskriterien
	%%
	\section{Abgrenzungskriterien}
	%\tr{Hier werden die Grundlagen für das zu entwickelnde Softwaresystem definiert. Zwar noch aus fachtechnischer Sicht werden 	hier die Anforderungen an das geplante Softwaresystem in möglichst formaler Form spezifiziert. 	Es 	sollen hier keine 		Lösungen präsentiert werden, sondern möglichst präzise die Anforderungen (Sollkonzept) an das geplante Softwaresystem mit 	seinen Schnittstellen, Informationsflüssen und Systemfunktionen 			dokumentiert 	werden. Verwendete Methoden 	können z.B. SA, SADT, Petri-Netze oder andere sein. Das Ergebnis ist ein für die Systementwicklung verwendbares 			Pflichtenheft. Über Art und Umfang des Pflichtenhefts sollten 	Sie mit Ihrem Betreuer sprechen.}
	\subsubsection{Zeit}
	Als eins der wichtigsten Abgrenzungskriterien gilt es die Einarbeitungszeit zu bewerten. Hier gilt je kürzer desto besser, immer gesehen in Relation 	zu dem Umfang des Frameworks. So ist ein Framework qualitativer zu bewerten, wenn es exponentielle Lernkurve in Relation zur Zeit vorweist.

\begin{figure}[H]
	\centering
	\begin{tikzpicture}
		
	\draw[thick,->] (0,0) -- (10,0) node[right]{$Zeit$};
	\draw[thick,->] (0,0) -- (0,10) node[above]{$Lernfortschritt$};

	\draw[red,id=test1,samples=100,domain=0.0:9.0] plot(\x,{1.3*ln(\x+1)}); 
	\draw[blue,domain=0:3.3] plot (\x,{-1+exp(ln(2)*\x)});
	\draw[red,thick] (9,5) -- +(0.3,0) node[anchor=mid west,black] {unproduktive Lernkurve};
	\draw[blue,thick] (9,5.5) -- +(0.3,0) node[anchor=mid west,black] {effiiziente Lernkurve};

	\end{tikzpicture}
	\vspace{-25pt}
	\caption[Darstellung der Lernkurve für Frameworks]{Lernkurve für Frameworks}
\end{figure}
\vspace{-40pt}


	\pagebreak
	 \subsubsection{Erweiterbarkeit}
	 In Hinsicht auf mehrere Endgeräte gibt es mehrere Aspekte zum analysieren. Zum einen ob das Framework auf verschiedene 	Arten an System gebunden sind, wie etwa Android oder iOS. Zum anderen ob es eine Limitierung der Anzahl der 				anzuschliessenden Geräte gibt. 

	\\Unter den Aspekt der Erweiterbarkeit fällt auch die Möglichkeit das Framework um eigene Funktionalitäten zu erweitern. So 		sollte im Idealfall ein Framework ein Grundgerüst liefern, auf das der Entwickler mit eigenen Erweiterungen aufbauen kann um 	das gewünscht Ziel zu erreichen.
	
	\subsubsection{Steuerbefehle}
	Ein wichtiger Aspekt dieser Arbeit ist die Steuerung beziehungsweise die Generierung von Steuerbefehlen und deren Verteilung auf alle verbundenen Klienten. Insbesondere wurde in dieser Arbeit auf das Scrollen innerhalb einer Web-Applikation, das ausfüllen von Formularen mit den dazugehörigen Eingabefeldern, sowie das ausführen Eventgesteuerter Aktionen.

	\subsubsection{Unterstütze Browser}
	Ein wichtiger Aspekt der durchzuführenden Tests wird die Unterstützung möglichst verschiedener Browser beinhalten. Dies hat 	den Grund, dass lediglich ein Framework was eine große Spanne an Endgeräten abdeckt in der Lage ist effektiv genutzt werden 	zu können.

	\subsubsection{Virtuelle Umgebung}
	Ein positiv in die Validierung einfließender Aspekt ist die Einbindung oder Verwendung des Frameworks innerhalb einer 			virtuellen Umgebung. Das wird durch den Fakt begründet das der Tester nicht immer im Besitz aller Testgeräte oder 				Umgebungen ist. So ist es ohne eine virtuelle Maschine zum Beispiel nicht möglich eine Seite im Internet Explorer innerhalb 		einer MacOS-Umgebung zu testen, da diese ihn nicht unterstützt.



%%
%% ############# Grundlagen
%%
\chapter{Grundlagen}
In diesem Abschnitt werden allgemeine technische Abläufe, die Notwendig sind um diese Arbeit und die darin verwendetet Techniken zu verstehen, behandelt. Des Weiteren werden die verwendeten Hardwarekomponenten, sowie genutzte Browser und Software aufgeführt.

	\section{Grundsätzlicher Aufbau}
	Alle in dieser Arbeit behandelten Softwareprodukte und Frameworks benötigen entweder einen vorhandenen Webserver, wie z. B. Apache, um ihre Daten zu übermitteln, oder sie bringen einen softwareinternen Server mit sich. Als Vorraussetzung für eine funktionierende Kommunikation müssen die Testgeräte sich alle innerhalb eines gemeinsamen Netzwerkes befinden. Zum Arbeiten empfiehlt sich ausserdem eine durchgängige Stromzufuhr der verbundenen Geräte.
	
	
	\igp{../pictures/netzwerkaufbau}{Startbildschirm Ghostlab}{Startbildschirm von Ghostlab nach der Installation}{400}{300}
	%%
	%% ############# Begriffsklärung
	%%
	
	\section{Begriffsklärung - entfernt und ausgelagert in glossar}	
	\tr{
		\Gls{parallel-synchron}
		\Gls{Web-Applikation}

		\Gls{Computer}

		\Gls{Ajax}
		\Gls{moEn}
		\Gls{Framework}
		\Gls{Webbrowser}
		\Gls{HTML}
		\Gls{Javascript}
		
		\Gls{NodeJS}
		\Gls{PHP}
		\Gls{NPM}
		\Gls{qs}
		\Gls{VirtualBox}
		
		\Gls{Smartphone}
		\Gls{Tablet}
		\Gls{PoW}
		\Gls{PaW}
		\Gls{Pixel}
		\Gls{BA}
		
		\Gls{Viewport}
		\Gls{Event}
		\Gls{DOM}
		\Gls{Apache}
		
		\Gls{Checkbox}
		\Gls{Radiobox}
		\Gls{Input}
		\Gls{Anker}
		
		\Gls{Zertifizierung}
		
		\Gls{Workspace}
		\Gls{Commit}
		\Gls{Deamon}
		
		\Gls{App}
		
		\Gls{htaccess}
		
		\Gls{GPL}
		\Gls{MIT}
		\Gls{AL2}
		
		
		\Gls{Cloud}
		\Gls{iFrame}
		\Gls{Grunt}
		\Gls{Test}
		
		}
	
	%%
	%% ############# verwendete Hardware
	%%
	\pagebreak
	\section{Verwendete Hardware}
	Alle in dieser Thesis aufgeführten Tests wurden mit den nachfolgenden Geräten und Umgebungen ausgeführt und validiert.
	
	\subsubsection{Apple iMac 27\texttt\dq}
	Zur Durchführung dieser Arbeit und der darin enthaltenen Evaluationsverfahren wurde ein Apple iMac mit folgenden 				Spezifikationen genutzt.
	
	\begin{table}[H]
	 \vspace{-20pt}
 		\centering
		\rowcolors{1}{white}{lightgray}
			\begin{tabular}{| p{4cm} | p{8cm}  |}
			\hline
				Prozessor			&	3,4GHz Intel Core i7 \\
				Speicher			&	8GB 1600Mhz DDR3\\
				Grafikkarte		&	NVIDIA GeForce GTX 675MX 1024 MB\\
				Betriebssystem		&	OS X 10.8.5 (12F45)\\

				\hline
				\end{tabular}
			\caption{verwendete Hardware}
	\end{table}

	\subsubsection{Mobile Endgeräte}
	Die in dieser Arbeit durchgeführten Tests nutzen folgende Endgeräte. 
	
	\mmet{Nokia Lumia 920}{Windows Phone}{8.0}{11,4 cm (4,5 Zoll)}{768x1280}{Portrait}
	\mmet{LG Nexus 4}{Android}{4.4.2 (KitKat)}{11,9 cm (4,7 Zoll)}{768x1280}{Portrait}
	\mmet{Apple iPhone4 32 GB}{iOS}{6.1.3 (10B329)}{8,9 cm (3,5 Zoll)}{640 x 960}{Portrait}
	\mmet{Apple iPhone5s 16 GB}{iOS}{7.0.6 (11B651)}{10,2 cm (4,0 Zoll) }{640 x 1136}{Portrait}
	\mmet{Apple iPad mini Wi-Fi 32GB}{iOS}{7.0.4 (11B554a)}{20,1 cm (7,9 Zoll)}{1024 x 768}{Landschaft}
	\mmet{Microsoft Surfcae}{Windows}{8.1 Pro}{26,9 cm (10,6 Zoll)}{1920 x 1080}{Landschaft}
	
	
	%%
	%% ############# Komponenten
	%%

	\section{Verwendete Software}
	
	\subsubsection{Virtuelle Maschine}
	\begin{table}[H]
	 \vspace{-20pt}
 		\centering
		\rowcolors{1}{white}{lightgray}
			\begin{tabular}{| p{4cm} | p{8cm}  |}
			\hline
				Hersteller			&	Oracle VM\\
				Product			&	VirtualBox\\
				Version			&	4.2.16 r86992\\
				Image			&	Windows Vista\\
				Virtueller Speicher	&	1024 MB\\
				\hline
				\end{tabular}
			\caption{verwendete virtuelle Maschine}
	\end{table}
	
	\subsubsection{Verwendete Browser}	
	\begin{table}[H]
	 \vspace{-20pt}
 		\centering
		\rowcolors{1}{white}{lightgray}
			\begin{tabular}{| p{8cm} | p{4cm}  |}
			\hline
				Browser		 	&	Version	\\
			\hline

			\hline
				Google Chrome			&	34.0.1847.116\\
				Google Chrome (virtuell)		&	33.0.1750.149 m\\
				Mozilla Firefox				&	28.0\\
				Mozilla Firefox (virtuell)		&	25.0.1\\
				Opera					&	20.0\\
				Safari					&	6.1.3 (8537.75.14)\\
				Internet Explorer (virtuell)		&	9.0.8112.16421\\
				\hline
				\end{tabular}
			\caption{verwendete Browser}
	\end{table}


	
	
	


%%
%% ############# Technologien
%%
\chapter{Auswahl der \Gls{Framework}s}

\section{\Gls{Framework}s als Komplettlösung}
Unter diesem Punkt werden alle \Gls{Framework}s gelistet, welche damit werben das \Gls{parallel-synchron}e Testen von mobilen Seiten zu ermöglichen und somit den Entwicklungsprozess zu optimieren. Im Gegensatz zu den Einzelkomponenten, sollten diese im Idealfall keine weiteren Technologien benötigen um genutzt zu werden.
	
	\subsection{\mi{Ghostlab}}
	Ghostlab ist ein \Gls{Framework} des schweizer Unternehmens Vanamco. Es verspricht das synchrone Testen von Webseiten in Echtzeit. Weiterhin wirbt das Unternehmen mit einem umfangreichen Repertoire an nützlichen Fähigkeiten. Der Funktionsumfang umfasst das Scrollen innerhalb einer Seite, das Ausfüllen von  Formularen, das Wahrnehmen und Reproduzieren von Click-Events sowie dem Neuladen einer Seite. Ghostlab soll ebenso einen Inspektor besitzen, welcher die Analyse des \Gls{DOM}, der 'on the fly'\footnote{Anpassungen in Echtzeit, ohne Neuladen der Seite} Bearbeitung von CSS und der Analyse und Bearbeitung von \Gls{Javascript}dateien. Das \Gls{Framework} gibt an, für alle folgenden \Gls{Webbrowser} zu funktionieren, ohne diese konfigurieren zu müssen:

	\begin{table}[H]
 		\centering
		\rowcolors{1}{white}{lightgray}
			\begin{tabular}{| p{5cm} | p{5cm} |}
			
			\hline
				Browser 	& 	Version\\
			\hline
			\hline
				Firefox	&	latest\\
				Chrome	&	latest\\
				Safari	&	latest\\
				Internet Explorer	&	8/9/10\\
				Opera	&	11\\
				Opera Mobile	&	supportet\\
				Firefox Mobile	&	supportet\\
				Blackberry	&	supportet\\
				Windows Phone	&	supportet\\
				Safari mobile	&	supportet\\	
				Android	&	2.3 - 4.2\\
				\hline
				\end{tabular}
			\caption{von Ghostlab getestete \Gls{Webbrowser} (Stand 10.03.2014, Version 1.2.3)}
	\end{table}

	Der Kostenpunkt der Lizenz liegt bei Erstellung dieser Arbeit bei 49\$ (entspricht in etwa 35,30 € Stand: 20.3.2014). Zur Erstellung dieser Thesis wurde die 7-Tage-Testvollversion genutzt.
	
	\subsection{\mi{Adobe Edge Inspect}}
	Die Anwendung Edge Inspect stammt von Adobe und wird derzeit in der CC\footnote{Creative \Gls{Cloud}} Version vertrieben. Um Adobe Edge Inspect nutzen zu können, bedarf es drei separate Komponenten. Adobe wirbt mit synchronem Aufrufen und Auffrischen von Websites, sowie deren Inspizierung per weinre. Besonders angepriesen wird von Adobe die Nutzung und Verwendung der Adobe Edge Inspect API, welche auf GitHub zur Verfügung gestellt wird. Des Weiteren kann Adobe Edge Inspect in andere Edge Produkte\footnote{zum Beispiel Edge Reflow CC und Edge Code CC} integriert werden. 
	
	\\Adobe Edge Inspect CC steht 30 Tage kostenlos zum Testen bereit. Danach fallen ab 24,59 / Monat für die Nutzung des Einzelprodukt-Abos an.
	
	\\Die Anwendung läuft nur auf mobilen \Gls{moEn}en mit iOS oder Android Betriebssystem.
	
	\subsection{\mi{Remote Preview}}
	Remote Preview ist ein kleines \Gls{Javascript}framework von dem Web Designer Viljami Salminen aus Helsinki, Finnland. Es überprüft alle 1100ms per \Gls{Ajax}-Request ob sich die Quell-URL geändert hat und teilt dies dann den verbundenen Testgeräten mit. Er wirbt mit dem synchronen Aufruf von Webseiten auf einer Vielzahl von Plattformen: 
	
	\begin{table}[H]
 		\centering
		\rowcolors{1}{white}{lightgray}
			\begin{tabular}{| p{13cm} |}
			
			\hline
				Plattform\\
			\hline
			\hline
				Android OS 2.1 - 4.1.2 (Default browser + Chrome)\\
				Blackberry OS 7.0 (Default browser)\\
				iOS 4.2.1 - 6 (Default browser)\\
				Mac OS X (Safari, Chrome, Firefox, Opera)\\
				Maemo 5.0 (Default browser)\\
				Meego 1.2 (Default browser)\\
				Symbian 3 (Default browser)\\
				Symbian Belle (Default browser)\
				WebOS 3.0.5 (Default browser)\\
				Windows Phone 7.5 (Default browser)\\	
				Windows 7 (IE9)\\
				\hline
				\end{tabular}
			\caption{von Remote Preview unterstützte Plattformen (stand 19.03.2014, letzter \Gls{Commit} 7dc48caa84)}
	\end{table}
	Das \Gls{Framework} ist Kostenlos erhältlich und läuft unter der \Gls{MIT}. Zum Zeitpunkt dieser Arbeit scheint das Projekt nicht 		weiter entwickelt zu werden, da seit 5 Monaten auf der Projektseite keinerlei Aktualisierungen vorgenommen wurden.

		
	\subsection{\mi{Browser-Sync}}
	Browser-Sync wurde von Shane Osbourne entwickelt und soll im Zuge dieser Arbeit den Ansprüchen zur Verbesserung der Qualität gerecht werden. Es wirbt mit synchronisierter Steuerung, dem Entwickeln an CSS Styles und anderen Projektdateien in Echtzeit, der Installation unter Windows, MacOS und Linux und einer umfangreichen Palette an unterstützen Plattformen. Jedoch unterstützt das \Gls{Framework} im Gegensatz zu Ghostlab oder Adobe Edge Inspect keine Remoteinspection des \Gls{DOM} und den Netzwerkaktivitäten.
	
		\begin{table}[H]
 		\centering
		\rowcolors{1}{white}{lightgray}
			\begin{tabular}{| p{5cm} | p{5cm} |}
			
			\hline
				Browser 	& 	Version\\
			\hline
			\hline
				Firefox	&	latest\\
				Chrome	&	latest\\
				Safari	&	latest\\
				Internet Explorer	&	7/8/9/10\\
				Opera	&	latest\\
				Opera Mobile	&	supportet\\
				Firefox Mobile	&	supportet\\
				Blackberry	&	supportet\\
				Windows Phone	&	supportet\\
				Safari mobile	&	supportet\\	
				Android	&	supportet\\
				iOS		&	supportet\\
				\hline
				\end{tabular}
			\caption{von Browser-Sync getestete \Gls{Webbrowser} (Stand 21.03.2014, Version 0.7.2)}
	\end{table}
	
	Das \Gls{Framework} basiert auf dem \Gls{NodeJS} \Gls{Framework} und besitzt dadurch ein hohes Erweiterungspotential. Eine parallel zu Browser-Sync entwickelte Erweiterung kombiniert Browser-Sync mit \Gls{Grunt}, was automatisierte Abläufe ermöglicht. Diese fördert die Produktivität durch das Einbinden des \Gls{Framework}s in bestehende Arbeitsabläufe. Die Software ist kostenlos erhältlich und steht unter der \Gls{MIT}. Das Projekt befindet sich zum Zeitpunkt dieser Arbeit in der Version 0.7.2 und wird täglich weiterentwickelt.

\pagebreak
\section{\Gls{Framework}s als Teilkomponente}
	Als Teilkomponenten werden hier \Gls{Framework}s spezifiziert, welche dazu beitragen, das in der Thesis geforderte Werkzeug selbst zu entwickeln. Diese decken verschiedene spezielle Funktionen ab, wie zum Beispiel die Clientverwaltung, Steuerbefehle oder setzen die Voraussetzung für eigene \Gls{Test}szenarien.
	
	\subsection{\mi{NodeJS}}
	Node.JS Aufgabe besteht darin, anstelle von zum Beispiel \Gls{Apache}, einen Webserver zur Verfügung zu stellen, welcher nur auf \Gls{Javascript} basiert. Alle notwendigen serverseitigen Anfragen und Funktionen erfolgen in \Gls{Javascript}. Entwickelt wird Node.JS von der Kalifornischen Firma Joyent und befindet sich derzeit in Version 0.10.26. Geführt wird Node.JS unter der \Gls{MIT} und steht kostenlos auf nodejs.org oder unter GitHub zum Download bereit.
	
	\subsection{NPM \mi{socket.io}}
	socket.io ist ein \Gls{Framework} welches die WebSocket Technologie aktueller \Gls{Webbrowser} auf \Gls{Javascript} Ebene abbildet. Der Gedanke der Technologie dahinter verfolgt den Ansatz nicht in regelmäßigen Abständen Anfragen an den Server zu stellen und damit unnötig viel Datenvolumen zu generieren, sondern eine permanente Verbindung zum Server aufrecht zu halten um auf Statusänderungen am Server zu reagieren. socket.io wurde von Guillermo Rauch unter der \Gls{MIT} entwickelt und steht derzeit in der Version 0.9.16 auf GitHub oder per \Gls{NPM} zur Verfügung.
	
	\subsection{\mi{Zombie.js}}
	Das \Gls{Framework} Zombie.js ist ein Open-Source Projekt einer ganzen Gruppe von Entwicklern\footnote{https://github.com/assaf/	zombie/graphs/contributors}, welches von dem in Kalifornien sitzenden Assaf Arkin ins Leben gerufen wurde. Zombie.js wirbt mit 	seiner Einfachheit \Gls{Test}s zu erstellen und in \Gls{Test}suiten zu integrieren. Zombie.js emuliert einen sogenannten headless\footnote{Kopflos - ohne Gerüst das ihn umschließt, oder auch virtuell} \Gls{Webbrowser}. Dies hat zur Folge, dass natürlich nur nonvisuelle Aspekte in \Gls{Test}s integriert werden können, wie etwa das Ausfüllen von Formularen, das Navigieren durch den Navigationsbaum oder das Testen von Links.

	\subsection{\mi{Phantom Limb}}
	Phantom Limb ist ein von Brian Carstensen entwickeltes Werkzeug welches es ermöglichen soll, die \Gls{Computer}maus generierten Bewegungen in äquivalente \mbox{Touchevents} umwandelt. Das \Gls{Framework} läuft unter der \Gls{AL2} und kann kostenlos 	verwendet werden. Es kam in die Auswahl der \Gls{Framework}s, da seine Funktionalität zur Generierung von Steuerbefehlen geeignet ist.
	
	\subsection{\mi{jQuery UI Touch Punch}}
	jQuery UI Touch Punch ist eine Erweiterung zu der UI Bibliothek von jQuery die David Furfero entwickelt hat. Diese erlaubt von 	Hause aus nicht die Nutzung von Touchevents auf mobilen \Gls{moEn}en. Die Erweiterung hebt diese Restriktion auf, ohne weiter konfiguriert werden zu müssen. An Quellcode kommen lediglich weitere 584 Bytes hinzu. Die \Gls{Framework}erweiterung läuft unter der MIT und der \Gls{GPL}, wodurch es dem Endnutzer frei steht die Bibliothek unter den Lizenzen des eigenen Projektes zu verwenden.
	
	\subsection{\mi{jQuery Touchit}}
	Das von Daniel Glyde entwickelte \Gls{Framework} jQuery Touchit, wandelt Berührungen in äquivalente Mousevents um und ermittelt deren relative Position in Bezug auf den \Gls{Viewport}. Des Weiteren löst es das Problem bei bereits bestehenden jQuery Anwendungen und deren Darstellung auf mobilen \Gls{moEn}en, wo verschiedene Funktionalitäten, wie zum Beispiel die Verwendung von Slidern, nicht nutzbar sind.
	
	
	
	

%%
%% ############# Evaluation
%%	
%%
%% ############# Evaluation
%%
\chapter{Evaluation der Frameworks}
\section{Auflistung des Evaluationsschlüssels}

Um die einzelnen Frameworks zu Evaluieren habe ich einen Schlüssel aufgelistet, welcher messbare Aspekte abdeckt, die ich im Vorfeld der Arbeit bereits als wichtig erachtet habe um die aufgestellte Thesis in Zahlen darzustellen. Ergänzend kamen Punkte hinzu die bei der Installation und der Nutzung auffielen und sich als wichtig erwiesen. 

\\Ich habe den Schlüssel in 7 Hauptkategorien aufgeteilt. Die Abschnitte Installation und Konfiguration befassen sich in erster Linie mit der Verfügbarkeit, dem Zugang zu dem Framework, deren Installation und Dokumentation sowie der Voraussetzung andere Technologien um es zu nutzen.

\subsubsection{Installation}
\begin{table}[H]
 	\vspace{-30pt}
 	\centering
	\rowcolors{1}{white}{lightgray}
		\begin{tabular}{| p{12cm} | c|}
			\hline
				Kriterium		 &	Punktezahl\\
			\hline
			\hline
				Notwendigkeit von anderen Technologien				&4\\
				Nutzbar direkt nach der Installation			&	2	\\
				Installationsanleitung vorhanden			&	2	\\
				FAQ vorhanden				&	2	\\
				\hline
		\end{tabular}
	\caption{Kriterienübersicht: Installation}
\end{table}

\subsubsection{Konfiguration}
\begin{table}[H]
 	\vspace{-30pt}
 	\centering
	\rowcolors{1}{white}{lightgray}
		\begin{tabular}{| p{12cm} | c|}
			\hline
				Kriterium		 &	Punktezahl\\
			\hline
			\hline
				Nutzbar ohne Konfiguration			&4\\
				Konfigurierbarkeit(IP und Ports)			&	1	\\
				Konfigurierbare Workspaces			&	2	\\
				Support vorhanden (Wiki, Helpdesk, EMail, Forum)				&	2	\\
				Intuitive Benutzeroberfläche			&	1	\\
				\hline
		\end{tabular}
	\caption{Kriterienübersicht: Konfiguration}
\end{table}

\\Der Teilschlüssel Funktion wurde dupliziert und in einen Mobilteil und einen Desktopteil aufgeteilt. Das hat den Grund, dass das testen von mobilen Seiten andere Schwerpunkte der Bewertung haben sollte, als das testen von Desktopseiten. So ist eine funktionierende synchrone Gestenkontrolle beim testen von mobilen Seiten zum Beispiel unerlässlich, wohingegen sie auf Desktops durch den Einsatz einer Maus tendenziell eher unhandlich in der Nutzung ist.
\subsubsection{Funktion: Desktop}
\begin{table}[H]
 	\vspace{-30pt}
 	\centering
	\rowcolors{1}{white}{lightgray}
		\begin{tabular}{| p{12cm} | c|}
			\hline
				Kriterium		 &	Punktezahl\\
			\hline
			\hline
				Darstellung : normale Seiten			&2\\
				Darstellung : gesicherte Seiten		&	2	\\
				 Darstellung: normale Reaktionsgeschwindigkeit ( < 1 Sekunde)	&	2	\\
				Funktion: Seitensteuerung			&	3	\\
				Funktion: Javascript			&	1	\\
				\hline
		\end{tabular}
	\caption{Kriterienübersicht: Desktop}
\end{table}

\subsubsection{Funktion: Mobil}
\begin{table}[H]
 	\vspace{-30pt}
 	\centering
	\rowcolors{1}{white}{lightgray}
		\begin{tabular}{| p{12cm} | c|}
			\hline
				Kriterium		 &	Punktezahl\\
			\hline
			\hline
				Darstellung : normale Seiten			&1\\
				Darstellung : gesicherte Seiten		&	1	\\
				 Darstellung: normale Reaktionsgeschwindigkeit ( < 1 Sekunde)	&	2	\\
				Funktion: Seitensteuerung			&	4	\\
				Funktion: Javascript			&	1	\\
				Funktion: Gestenkontrolle			&	1	\\
				\hline
		\end{tabular}
	\caption{Kriterienübersicht: Mobil}
\end{table}

\\Da es in der Praxis kein Framework gab, welches zum Zeitpunkt dieser Arbeit in der Lage war alle gewünschten Aspekte abzudecken, war es wichtig das Werkzeug um eigene Funktionen oder externe Frameworks zu ergänzen. Dies wurde unter Berücksichtigung der API, dessen Lizenz und Dokumentation bewertet.
\subsubsection{Erweiterbarkeit}
\begin{table}[H]
 	\vspace{-30pt}
 	\centering
	\rowcolors{1}{white}{lightgray}
		\begin{tabular}{| p{12cm} | c|}
			\hline
				Kriterium		 &	Punktezahl\\
			\hline
			\hline
				API Zugang			&5\\
				API lizenzteschnich gesichert	&	2	\\
				API Dokumentation	&	3	\\
				\hline
		\end{tabular}
	\caption{Kriterienübersicht: Erweiterbarkeit}
\end{table}

\\Ein weiterer wichtiger Aspekt ist die Unterstützung möglichst vieler verschiedener Browser auf dem Desktop, auf dem Mobilgerät und in der virtuellen Umgebung.

\subsubsection{Browser Support (aktuelle Versionen)} 
\begin{table}[H]
 	\vspace{-30pt}
 	\centering
	\rowcolors{1}{white}{lightgray}
		\begin{tabular}{| p{12cm} | c|}
			\hline
				Kriterium		 &	Punktezahl\\
			\hline
			\hline
				mobile Plattformen (iOS, Android, Windows)			&3\\
				Virtuelle Browser	&	2	\\
				Chrome				&	1	\\
				Opera				&	1	\\
				Firefox				&	1	\\
				Safari				&	1	\\
				Internet Explorer		&	1	\\
				\hline
		\end{tabular}
	\caption{Kriterienübersicht: Browser}
\end{table}

\\Die Aktivität einer Software lässt darauf schließen, ob und gegebenenfalls wie diese sich in Zukunft Entwickeln kann. So sind von einem inaktiven Entwicklungsstand von über einem halben Jahr keine neuen Ergebnisse mehr zu erwarten und man muss davon ausgehen das die Software um keine neuen Features erweitert werden wird.
\subsubsection{Aktivität}
\begin{table}[H]
 	\vspace{-30pt}
 	\centering
	\rowcolors{1}{white}{lightgray}
		\begin{tabular}{| p{12cm} | c|}
			\hline
				Kriterium		 &	Punktezahl\\
			\hline
			\hline
				Noch in der Entwicklung (letzes Release, Commit Häufigkeit)			&5\\
				Aktives Forum	&	5	\\
				\hline
		\end{tabular}
	\caption{Kriterienübersicht: Aktivität}
\end{table}



%%
%% ############# Ghostlab
%%
	\pagebreak
	\section{\mi{Ghostlab} Version 1.2.3}
		\subsection {Einrichtung der Testumgebung}
		Ghostlab kommt von Hause aus mit einer 7-Tage-Testversion. Die Installation verlief einfach und ereignislos. Nachdem das 		Tool Installiert wurde erfolgte die Zuweisung einer Website zu dem Ghostlabserver. Es wurden in diesem Fall sowohl eine 		Seite auf einem lokalen Apache Server getestet, als auch die mitgelieferte Demoseite von Ghostlab. Nach dem Start des 			Ghostlabservers ist dieser über den localhost\footnote{IP-Adresse des lokalen Rechners} auf Port 8005 (Default) von allen 		zu testenden Geräten erreichbar.
		\ig{../pictures/ghostlab/startbildschirm}{Startbildschirm Ghostlab}{Startbildschirm von Ghostlab nach der Installation}
		
		\subsection{Testen von Desktopbrowsern}
		Durch aufrufen der IP-Adresse des Rechners auf dem der Ghostlabserver läuft verbindet sich der Browser als Client und 			wird fortan durch gesendete Signale beeinflusst. Hierzu zählen auch virtuelle Browser. Jeder Client wird nun gleichzeitig 			Sender und Empfänger für Signale, dass bedeutet das jede Aktion parallel-synchron auf allen anderen Clients gespiegelt 			wird. Hierzu zählen Javascriptevents, das ausfüllen eines Formulars oder das neuladen der gesamten Seite.
		\ig{../pictures/ghostlab/workspaces}{Übersicht Clients}{Darstellung von 4 verschiedenen Clients } 
		
		Über den Übersichtsbildschirm kann jeder verbundene Client einzeln inspiziert werden. Hier ist der Nutzer in der Lage sich 		durch das DOM zu navigieren oder temporäre CSS Anpassungen vorzunehmen. Die Handhabung ist intuitiv, was jedoch an 		dem verwendeten Framework \mi{Weinre} liegt.
		\ig{../pictures/ghostlab/weinre}{Exemplarisch Weinreansicht}{ausgewähltes DOM-Element in Weinre}
		
		\pagebreak
		\subsection{Testen von mobilen Browsern}
		
		Das einrichten zum testen auf mobilen Endgeräten verläuft synchron zu den Desktopbrowsern. Man ruft innerhalb des 			Browsers die IP-Adresse des Ghostlabrechners auf und ist schon nach wenigen Sekunden\footnote{abhängig von der 			Geschwindigkeit des Testgerätes} in der Clientliste aufgenommen.
		
		\\Bei dem Testen auf mobilen Browserns ist es bei Ghostlab\footnote{Version 1.2.3} Notwendig ausreichend Zeit zwischen 		den Eingaben zu lassen, da es sonst bei unterschiedlich schnellen Geräten zu einem Effekt kommt, bei dem die 				langsameren Geräte beim ausführen des Letzen Signals gleichzeitig wieder zum Sender für alle anderen Geräte wird.
		\ig{../pictures/ghostlab/uebersicht_mobil}{Übersicht mobile Clients Ghostlab}{Ghostlabübersicht der verbundenen Clients}
		
		\pagebreak
		
		\subsection{Fazit zu Ghostlab}
		Zum Stand dieser Arbeit wurde Version 1.2.3 von Ghostlab genutzt. Zu diesem Zeitpunkt verfügte die Software noch über 		keinen Master/Slave-Modus\footnote{ein Gerät dient als Steuergerät, alle anderen folgen ihm}, dadurch kam es bei meinen 		Testgeräten bereits nach wenigen Minuten zu dem Problem, dass die Geräte sich in einer 								Endlosschelife von Senden und Empfangen der Steuerbefehle befanden. Für kommende Versionen ist ein solcher Modus 		laut den Entwicklern aber geplant. Das Problem rührt daher, dass einige Geräte schneller auf die übermittelten Befehle 			reagieren als andere. Das führt dazu, dass die langsam ladenden Geräte in dem Augenblick wo sie das Signal umsetzen, 		für die schnelleren Geräte bereits wieder als Sender fungieren. Dieses Problem sehe ich bei einer bereits kleinen Anzahl 			von Geräten als kritisch an. 

		\\Das testen in mehreren Browsern auf einem Rechner lief hingegen sehr gut. Das ausführen von Javascript läuft 				einwandfrei. Das ausfüllen von Inputs, Checkboxen, Radioboxen und das absenden des Formulars funktionierte bis auf die 		Kalenderauswahl im Firefox Browsers anstandslos. Ein Problem scheint das Werkzeug mit Passwortgeschützten Seiten zu 		haben. Diese lassen sich erst nach mehrfacher, abhängig vom jeweiligen Browser, Eingabe des Passwortes aufrufen. 			Diese Prozedur wiederholt sich für jede weitere Unterseite erneut. 

		\\Das arbeiten in einer Virtuellen Umgebung\footnote{es wurde VirtualBox von Oracle genutzt} wird problemlos unterstützt. 		Das einzige Problem was ich analysieren konnte war, dass sich virtuelle Browser nicht in einen Workspace integrieren 			lassen.

		\\Ghostlab unterstützt die Funktion von Workspaces\footnote{Arbeitsumgebung oder auch Arbeitsumfeld}, welche sich die 		Position und Größe der verschiedenen Browserfenster speichert. Per Knopfdruck lassen diese sich dann im Kollektiv öffnen 		sofern in den Browsereinstellungen die Popups aktiviert sind für die zu testende Seite. Dieses Feature\footnote{Funktion 			welche ein Teil der Anwendung ist} bewerte ich als Positiv in Hinsicht der Zeitersparnis, diesen Vorgang immer wieder von 		Hand auszuführen.

		\\Als Kritikpunkt bewerte ich die nicht existente Möglichkeit die Anwendung um eigene Funktionalität zu erweitern.

		\subsection{Tabellarische Evaluation}
		\met{Gewichtungstabelle Evaluation von Ghostlab}{10}{10}{8}{5}{0}{10}{5}
	
%%
%% ############# Adobe Edge Inspect
%%
	\pagebreak
	\section{\mi{Adobe Edge Inspect} CC }
		\subsection {Einrichtung der Testumgebung}
		Es sind 3 Schritte Notwendig Adobe Edge Inspect zum Einsatz bereit zu machen. Als erstes benötigen wir den Client aus 		der Adobe Creative Cloud (CC) Kollektion. Diese gibt es zum Zeitpunkt dieser Arbeit in verschiedenen Modellen und 			beginnt bei der kostenlose 30-Tage Testversion, geht über die Einzellizenz, für ausschliesslich Adobe Edge Inspect, von 			24,59€ / Monat bis hin zum Komplett-Abo was dann mit 61,49€ / Monat zu Buche schlägt. Dieser wird gestartet und läuft 		ab diesem Zeitpunkt als Deamon im Hintergrund. 
		\iga{../pictures/adobeedgeinspect/icon}{Adobe Edge Inspect Deamon Icon}{Der laufende Deamon von Adobe Edge Inspect}
		
		\\Als zweiten Schritt benötigen wir die zugehörige Chrome Extension von Adobe Edge Inspect. Diese wird über den Chrome 		Appstore installiert und kann nach einem Browserneustart aktiviert werden.
		
		\\Als letzes benötigen wir noch die kostenlos erhältliche App aus dem jeweiligen Shop. hier gilt für Android der Play Store, 		für iOS Geräte der AppStore. Windowsgeräte werden derzeit nicht unterstützt.
		
		\\Sind diese 3 Schritte erfolgreich durchgeführt worden, müssen nun die Geräte mit dem Server verbunden werden. Hierzu 		wird die App gestartet (der folgende Prozess verläuft unter Android wie auch unter iOS identisch) und per IP-Adresse mit 			der Adobe Edge Inspect Chrome Extension verbunden werden. Diese verlangt im Gegenzug einen Identifikationscode, 			welcher auf dem jeweiligen Gerät generiert wurde. Nach erfolgreicher Synchronisation wird das Gerät im Gerätemanager 		angezeigt.
		\igp{../pictures/adobeedgeinspect/iphone_2}{Adobe Edge Inspect App Client hinzufügen}{Eingabe der IP-Adresse zum 			Edge Inspect Rechner}{200}{350}
		\iga{../pictures/adobeedgeinspect/desktop_2}{Adobe Edge Inspect Chrome Extension}{Eingabe des Sicherheitscodes in die 		Chrome Extension}

		\\Dieser hat mehrere Funktionen. Er liefert eine Übersicht aller verbundenen Clients und ermöglicht das aufrufen von 			\mi{Weinre} um z.B. das DOM zu inspizieren, verwendete Ressourcen zu inspizieren oder Javascript auszuführen. Über 			den Gerätemanager lassen sich auch verbundene Geräte wieder durch einen Klick entfernen. Desweiteren kann man über 		dieses Interface Screenshot von allen verbundenen Geräten im aktuellen Zustand aufnehmen und anzeigen lassen. 			Weiterhin besteht die Möglichkeit den Darstellungsmodus auf den verbundenen Clients von der Appdarstellung in den 			Vollbildmodus zu wechseln.
		\iga{../pictures/adobeedgeinspect/desktop_3}{Adobe Edge Inspect Gerätemanager}{Übersicht der verbundenen Clients}
		
		\subsection{Testen von Desktopbrowsern}
		Es gibt zum Zeitpunkt der Erstellung dieser Arbeit keine Möglichkeiten Desktopseiten mit Adobe Edge Inspect zu testen.
		
		\subsection{Testen von mobilen Browsern}
		Die Funktionalität zum testen von mobilen Seiten beschränkt sich derzeit nur auf den synchronen Aufruf von Seiten über 			den Chromebrowser mit installierter Extension als Steuergerät. Die verbundenen Geräte erkennen den Aufruf von Links 			und das wechseln von Tabs innerhalb des Browsers. Es besteht wie bereits beschrieben die Option die einzelnen Clients 			per \mi{Weinre} zu untersuchen.
		
		\\Die simulierung eines Scrollevents oder das ausfüllen eines Formulars ist nicht möglich. Es werden lediglich die 				Informationen Dargestellt die am Steuergerät aufgerufen wurden. Jedoch wird der Client, sofern vorhanden auf die mobile 		Seite weitergeleitet. Während des Testens in der App wird das Display aktiv gehalten, wodurch es sich nicht von selbst 			abschaltet. Ein gutes Feature von Adobe Edge Inspect ist die Möglichkeit aus dem Gerätemanager des Browsers 				Screenshots der verbundenen Geräte anzufordern. Diese werden zusammen mit einer Beschreibung des Geräts, dessen 		Modellbezeichnung , die Auflösung sowie Pixeldichte, dem Betriebssystems, der aufgerufenen URL sowie der aktuellen 			Ausrichtung des Bildschirms ausgeliefert.
		\igp{../pictures/adobeedgeinspect/iphone_3}{Adobe Edge Inspect App Content Darstellung}{Darstellung von Content in der 		Adobe Edge Inspect App unter iOS}{200}{350}
		
		Während meiner Versuche ist mir aufgefallen, dass Adobe Edge Inspect unter iOS 6.1.3, Seiten die durch htaccess 				gesichert Sind nicht darstellen kann. Auf den anderen Testgeräten verlief der Prozess der Authentifizierung problemlos. 
		
		\pagebreak
		\subsection{Fazit zu Adobe Edge Inspect}
		Adobe Edge Inspect bedarf viel Aufwand für ein relativ geringes Ergebnis. Man muss an 3 verschiedenen Punkten 				Installationen vornehmen, die dann jedoch ohne Probleme miteinander harmoniert haben. Als besonders Positiv möchte ich 		die Screenshotfunktion bewerten. In Zusammenspiel mit der öffentlich zugänglichen API lassen sich hierüber Screenshots 		im Landschafts, als auch im Portraitmodus anfordern und durch eine externe Applikationen auswerten. 
		
		\\Der Nutzen des Werkzeugs liegt am ehesten bei One-Page-Sites\footnote{Webseiten dessen Inhalt sich füllend auf die 			gesamte Seite erstrecken} oder für Fehlersuche innerhalbs des DOM oder CSS Anpassungen mit \mi{Weinre}. Unter dem 		Aspekt 	des parrallel-synchronen Testens ist Adobe Edge Inspect leider nicht sinnvoll zu verwenden, da wed er 				Steuerbefehle oder 		andere Gesten umgesetzt werden, noch werden die Nutzereingaben in Eingabefeldern mit 			anderen verbundenen Clients geteilt. 	Alle verbunden Clients sind nur Empfänger und besitzen keine Möglichkeit 			als 			Sender zu fungieren. Folglich gehen alle 			Steuerbefehle vom Edge-Server aus.
	
	\subsection{Tabellarische Evaluation}
		\met{Gewichtungstabelle Evaluation von Adobe Edge Inspect}{10}{7}{0}{4}{8}{3}{10}


%%
%% ############# Remote Preview
%%
	\pagebreak			
	\section{\mi{Remote Preview}}
		\subsection {Einrichtung der Testumgebung}
		Es gibt zwei Möglichkeiten dieses Werkzeug zu nutzen. Die eine ist die Installation auf einem lokalen Apache-Server mit 			PHP. Die andere ist die Installation auf einem Cloud-Dienst wie z.B. Dropbox. Die Ergebnisse dieser Arbeit hab ich mit der 		lokalen Apache Installation erzielt. Die Installation sieht lediglich vor das Framework in einen lokalen Entwicklungszweig zu 		entpacken.
		
		\subsection{Testen von Desktopseiten}
		Alle Clients die in die Testumgebung eingebunden werden sollen müssen lediglich die IP-Adresse des Servers eingeben.
		Die Steuerung der Seiten erfolgt sowohl für Desktopseiten als auch für die mobilen Vertreter über die Browsermaske des 			Frameworks. In das untere der beiden Eingabefelder gibt man die aufzurufende URL inklusive Präfix\footnote{http://} ein. 			Diese wird dann auf allen verbundenen Clients innerhalb eines iFrames dargestellt. 
		\ig{../pictures/remotepreview/eingabemaske}{Remote Preview Steuerungsmaske}{Steuerungsmaske zur Eingabe der 			aufzurufenden URL}
				
		 \subsection{Testen von mobilen Browsern}
		 Das Testen der mobilen Browser funktioniert parallel zum testen von Desktopseiten. Positiv möchte ich hier erwähnen, 			dass das Framework auch wenn es dafür nicht ausgelegt ist, dennoch unter aktuellen Windowsgeräten funktioniert.			
		
		\subsection{Fazit zu Remote Preview}
		Ein positiver Punkt ist die Möglichkeit letztendlich jeden Browser unabhängig von dessen Betriebssystems in die 				Testumgebung zu integrieren, da diese einfach nur auf den ApacheServer oder die Dropbox zugreifen müssen. Als Negativ 		führe ich hier die Tatsache auf das es ähnlich Adobe Edge Inspect lediglich dem Aufruf von Seiten dient, jedoch nicht 			dessen Bedienung. So ist es nicht möglich weiteren Verlinkungen zu folgen ohne diese von Hand in der Eingabemaske 			einzutragen oder Formulare auszufüllen. Bedingt funktioniert das Darstellen von Seiten mit Ankern. Das Aufrufen von 			gesicherten Seiten gelang mir nicht. Ebenfalls war es  mir nicht möglich zertifizierte Webseiten aufzurufen, was den 				Nutzungsgrad des Frameworks stark einschränkt. Gut finde ich die Tatsache das Quellcode komplett zugänglich ist, da er 		unter der MIT Lizenz steht und	jederzeit in eigene Projekte eingebunden oder um eigene Funktionalität erweitert werden 			kann. Somit kann man Remote Preview nutzen und es um erweiterte Funktionalität erweitern kann. Das Projekt 				scheint zum Zeitpunkt dieser Arbeit nicht weiter entwickelt zu werden. Für den Aufruf einer einfachen Seite auf n-Geräten 		ist dieses Projekt eine kostenlose Alternative zu Adobe Edge Inspect mit geringerem Funktionsumfang.
		
				
		\subsection{Tabellarische Evaluation}
		\met{Gewichtungstabelle Evaluation von Remote Preview}{8}{6}{3}{3}{8}{10}{0}
		
	
%%
%% ############# Browser-Sync
%%
\pagebreak
	\section{\mi{Browser-Sync}}	
	\subsection{Einrichtung der Testumgebung}
	Um Remote-Sync nutzen zu können wird zu Beginn erst einmal eine NodeJS Implementation benötigt. Diese kann entweder über die Konsole installiert werden oder per Installationstool von der NodeJS Homepage.
	
	\\Nach der NodeJS Installation wird per NPM das Paket von Browser-Sync per Konsole einmalig installiert:
	\iga{../pictures/browser-sync/install}{Browser-Sync Installation per Konsole}{Konsolenbefehl um Browser-Sync zu installieren}
	
	Nun muss für jedes neue oder bestehende Projekt einmalig im Projektordner Browser-Sync initialisiert werden. Browser-Sync legt in dem aktuellen Verzeichnis eine Konfigurationsdatei ab, in welcher man einzelne Optionen wie die zu beobachtenden Dateien oder Einstellungen zum Synchronisationsverhalten. Dies geschieht ebenfalls über die Konsole:
	\igp{../pictures/browser-sync/init}{Browser-Sync Initiierung per Konsole}{Konsolenbefehl um Browser-Sync zu imitieren}{450}{200}
	
	Nach der Initiierung des Servers startet man diesen mit dem Befehl :
	\iga{../pictures/browser-sync/start}{Browser-Sync Starten per Konsole}{Konsolenbefehl um Browser-Sync zu starten}
	
	Um nun die Kommunikation zwischen dem Server und dem Projekt zu gewährleisten muss vor dem Ende des Body Elements der Indexdatei zusätzlicher Scriptcode eingefügt werden, welcher jedoch zum Release entfernt werden sollte. Der einzufügende Code  wird anhand der Konfigurationsdatei und der IP-Adresse des Servers generiert und per Konsole dem Nutzer mitgeteilt.
	\igp{../pictures/browser-sync/start2}{Browser-Sync Script-Tag}{Konsolenausgabe mit einzufügendem Quellcode}{450}{200}
	
	In künftigen Versionen wird es laut dem Entwickler nicht mehr Notwendig sein die Versionsnummer mit anzugeben.
	
	\subsection{Testen von Desktopseiten}
	Das Testen erfolgt durch Aufruf der Seite, in die der Steuercode eingetragen wurde, über den Browser. Die parallele Steuerung erfolgt direkt und synchron. Ist ein Browser erfolgreich verbunden, wird dies in der Konsole des Servers angezeigt.
	\iga{../pictures/browser-sync/connected}{Browser-Sync verbundener Client}{Konsolenausgabe bei erfolgreich verbundenem Client}
	Das folgen von internen Links funktioniert nur Unidirektional, sofern der Steuercode nicht mittels Framework oder von Hand in die verlinkten Dateien eingefügt wurde. Das folgen externer Links erfolgt nur Unidirektional. Auch das aufrufen zertifizierter oder gesicherter Seiten mit Passworteingabe funktioniert Problemlos. Beim nutzen von Steuerbefehlen traten nur bedingt Probleme auf. So gibt es zum Zeitpunk dieser Arbeit Defizite im Umgang mit dem Javascriptframework jQuery. So lassen sich z.B. Lightboxen öffnen, jedoch werden dann Befehle zum Schliessen des Fensters nicht mehr erkannt und übermittelt. Das ausfüllen von Formularen verlief bis auf eine Mehrfachauswahl fehlerfrei. Das erkennen von Hoverevents funktionierte in der Version 0.7.2 noch nicht. Auch das parallele Verwenden von Sliderelementen war zu diesem Zeitpunkt noch nicht implementiert.
	
	\subsection{Testen von mobilen Browsern}
	Das testen von mobilen Browsern verläuft parallel zu Desktopseiten. Ein Aufruf über den internen Browser genügt um den Client am Server zu registrieren. Als zusätzliches Problem trat bei den mobilen Geräten ein Verzug an Elementen auf. Die Geräte richten sich anhand der gescrollten Entfernung aus und bieten zum Zeitpunkt dieser Thesis nicht die Möglichkeit der Ausrichtung an Elementen der Internetseite. So kommt es bei den Testgeräten zu Unstimmigkeiten im dargestellten Inhalt, welche durch die Unterschiedlichen Auflösungen und Ausrichtungen der Geräte zu Stande kommen.
	
	\subsection{Fazit zu Browser-Sync}
	Browser-Sync ist ein viel Versprechendes Framework, welches die zu untersuchenden Aspekte vollkommen abdeckt. Es bestehen noch relativ viele unausgereifte Komponenten, jedoch werden diese bei auftreten, Zeitnah von den Entwicklern behoben. Generell scheint das Framework zum Zeitpunkt dieser Arbeit eine hohe Entwicklungsgeschwindigkeit zu besitzen. Es trat gelegentlich ein Fehler auf bei dem ein verbundener Client, selbst nach mehrfacher Neuverbindung, nicht mehr auf die Steuersignale reagierte. Dieser Fehler trat bei meistens bei mehr als 6 verbundenen Klienten auf. Das Framework ist zum validieren von Websites gedacht, die sich noch in der Entwicklung befinden. Das testen ist aufgrund der notwendingen Testumgebung nur zum lokalen Arbeiten vorgesehen. Als Pluspunkt wird das injizieren von geändertem Code zur Entwicklungszeit gewertet. So ist es Möglichen zum Beispiel vorgenommene Änderungen am Styling oder dem DOM ohne weitere Handgriffe direkt auf allen Testgeräten zu begutachten.
	
	\subsection{Tabellarische Evaluation}
		\met{Gewichtungstabelle Evaluation von Remote Preview}{5}{2}{9}{7}{8}{10}{10}
	
	
	\section{Eigenes \mi{Framework}}
	Der Ursprüngliche Gedanke dieser Arbeit verfolgte dn Ansatz ein eigenes Framework zu entwickeln, was die parallel-synchrone Steuerung auf mehreren Endgeräten insbesondere auf mobilen Geräten ermöglicht. Diesen Gedanken berücksichtigend erfolgte eine Validierung verschiedener Einzeltechnologien die nur gewisse Aspekte abdecken. Untersucht wurden diese in Hinsicht auf ihre tatsächliche Funktionalität, ihre Installation und Kombinierbarkeit mit anderen verwendeten Frameworks.
	
\\Die Bibliotheken werden insbesondere auf ihre Implementation in einen Node.JS Server überprüft.

	\subsection{Installation eines Node.JS Servers}
	Die Installation des Node.JS Servers erfolgt einfach über die Konsole unter Mac oder den Installer\footnote{erhältlich unter Nodejs.org}. Alleinstehend erfüllt dieser Server keinerlei der gewünschten Funktionen, jedoch dient dieser als Grundlage für einige nachfolgende Frameworks. Node.JS ist eine gute Wahl aufgrund der hohen Verarbeitungsgeschwindigkeit sowohl Client als auch Server seitig. Weitere Pluspunkte sind die rasche Entwicklungsgeschwindigkeit, die hohe Vielfalt an Erweiterungen und Plugins, sowie eine sehr große aktive Entwicklergemeinde.
	\subsection{Einbinden von socket.io}
	socket.io lässt sich einfach über den NPM installieren. Es ermöglicht das herstellen einer permanenten Verbindung mit dem Server über einen Socket. Der Vorteil liegt hierbei darin, dass keine zyklischen Anfragen an den Server gesendet werden. Stattdessen wird hier das Observer-Pattern umgesetzt und alle verbundenen Clients werden vom Server informiert sobald eine Änderung des Status stattgefunden hat. 
	
	\subsection{Generierung von Steuerbefehlen über socket.io}
	Der generelle Aufbau von socket.io sieht vor, dass der Client sicht mit dem Server verbindet und eine permanente Verbindung mit diesem aufrecht erhällt. Identifizierbar bleibt diese über eine generierte, einzigartige, alphanumerische Session ID. socket.io funktioniert nach den Observer-Pattern, dass bedeutet das der Client nicht in zyklischen Abständen Anfragen an den Server sendet, sondern bei einer Änderung der Modelle oder zum Beispiel einem Funktionsaufruf vom Server mittels eines Broadcasts informiert wird. So entsteht das Problem, dass wenn ein Client eine Nachricht an den Server sendet, dieser allen Clients (auch dem Auslöser) diese Nachricht sendet. 
	
	\igp{../pictures/broadcast}{socket.io vereinfachte Darstellung eines Broadcasts}{socket.io Broadcast \cite{3}}{300}{300}
	
Für die entwicklung eines eigenen Frameworks wirft dies einige Probleme auf.

\\ Ein  Beispiel: Ein Nutzer klickt auf einer Seite auf einen Button. Das senden-Event wird an den Server gesendet und an alle per socket verbundenen Clients dupliziert. Somit würde der ursprüngliche Sender des Signals, erneut das selbe Event erhalten. Das Resultat wäre, dass dieser Den Button zweimalig drückt. Das kann zu Problemen führen, weil beispielsweise eine Clientseitie Aktion mehrfach ausgeführt wird. Ein weiteres Problem kann durch rekursives Aufrufen einer Methode einen Dead-Lock erzeugen. Clients die empfangen Signale langsamer als andere verarbeiten, können in dem Moment vom Empfänger direkt wieder zum Sender werden.
	
	\\Daher ist der Ansatz ein Master-Slave-Pattern umzusetzen denkbar sinnvoll. Es wird ein Steuergerät definiert, welches seine Aktionen dem Server mitteilt und dieser die Events dann an alle verbundenen Clients innerhalb eines Aktionsraumes sendet.
	
	\igp{../pictures/broadcastroom}{socket.io vereinfachte Darstellung eines Broadcasts mit einem Aktionraum}{socket.io Broadcast mit Aktionsraum}{350}{350}
	
	\subsection{Implementierung eines einfachen Broadcast}
	Das Beispiel soll veranschaulichen wie ein einfacher Broadcast ohne Aktionsraum implementiert werden kann. Das Beispiel soll das Scrollevent des Clients abfangen und auf allen verbundenen Clients an die selbe Position auf dem Bildschirm scrollen.
	
	\subsubsection{Serverseitig}
	Zu Begin werden die notwendigen Bibliotheken eingebunden um einen NodeJS Server starten zu können (Zeilen: 1-2). Im Anschluss wird  eine Serverinstanz von NodeJS erstellt (Zeile: 4) und gestartet auf Port 8001 (Zeile: 6). Dieser wird nun mit der socket.io Bibliothek verknüpft (Zeile: 7). Sofern nun ein Client sich mit dem Server verbindet wird das \frqq connection\flqq-Event gefeuert (Zeile: 9) und der Client wartet auf ein selbstdefinierten Methodenaufruf vom  \frqq scroll\flqq (Zeile: 11).
	
	\\Wenn am Server ein Srollevent eingegangen ist sendet dieser dies per Broadcast an alle verbundenen Clienten (Zeile: 12).
	
	\igp{../pictures/socketio/server_1}{socket.io Quellcode Scrollbeispiel Serverseitig}{minifizierter Quellcode Serverseitig}{350}{350}
	\subsubsection{Clientseitig}
	Auf der Seite des Clients müssen zwei Methoden implementiert werden. Zum einen die Methode die das Scrollevent des Browsers, was hier über jQuery erfolgt, abfängt und über die Socketverbindung die Servermethode \frqq scroll\flqq  aufruft und die aktuelle Scrollposition zum oberen Bildschirmrand übergibt.
	
	\igp{../pictures/socketio/client_1}{socket.io Quellcode Scrollbeispiel Clientseitig}{minifizierter Quellcode Clientseitig}{350}{350}
	
	Zum anderen muss die Methode implementiert werden, welche vom Server gesendete Events abfängt und verarbeitet. In diesem Beispiel wartet der Client auf ein Event vom Typ \frqq scroll \flqq. Dieses bekommt einen Datensatz, die Scrollposition, mitgeliefert. Nach erfolgreichem Eventaufruf wird per jQuery die Position des Bildausschnittes an den des mitgelieferten Datensatzes angepasst.
	
	\igp{../pictures/socketio/client_2}{socket.io Quellcode Scrollbeispiel Clientseitig}{minifizierter Quellcode Clientseitig}{250}{250}

	\subsection{Einschätzung zur Umsetzung eines eigenen Frameworks}
	Der realisierung eines eigenen Frameworks zur parallel-synchronen Steuerung von Webseiten steht nichts im wege. Die sehr schnelle Datenübertragung in nahezu Echtzeit mittels Node.JS, Vorraussetzung ist hier, dass die Geräte sich im gleichen Lokalen Netz befinden, die einfache Implementierung von Steuersignalen über socket.io und die modulare Grundstruktur der Frameworks ermöglichen einen einfachen Einstieg in die Materie.

\\Die Struktur ermöglicht es sämtliche Events abzufangen, egal ob mit jQuery oder anderen Frameworks zur Eventermittlung, um diese dann in entsprechende Funktionen umzuwandeln und an alle Clients weiterzugeben. Der Einsatz eines Mastergerätes und das nutzen von Aktionsräumen verhindern die irreführende Rückkopplungen innerhalb des Nachrichtenzyklus. 

\\Der Einsatz in einer virtuellen Umgebung erfolgt Problemlos, da keine weitere Software installiert werden muss. DieVerwendung von Socket.io ermöglicht die Unterstützung aller alten Browser Plattformen da das Framework mit einer Reihe von Fallbacks sich gegen Funktionsverlust absichert. Sollte keine WebSocket-Technologie verfügbar sein greift das Framework zuerst auf Adobe Flash Sockets zurück und sollte dies auch nicht Verfügbar sein auf eine Reihe verschiedener Long-Polling-Ansätzen um die Kommunikation weiterhin zu gewährleisten.

\chapter{Zusammenfassung und Ausblick}
Zum Abschluss der Arbeite werde ich in einem Fazit auf gesetzte Ziele eingehen und die Vorgehensweise der Evaluierung schildern. Im Anschluss wird ein kleiner Ausblick gewährt, was basierend auf dem aktuellen Stand der Framework eventuell verbessert oder noch entwickelt werden könnte.

\section{Zusammenfassung}
Das Ziel dieser Arbeit war die Evaluierung von Techniken zur
parallel-synchronen Bedienung einer
Web-Applikation auf verschiedenen
mobilen Endgeräten. Zu Beginn wurden erst einmal Frameworks ermittelt, welche die gesetzten Kriterien versprechen ganz oder zu großen Teilen abzudecken. 
Derzeit gibt es nur sehr wenige Anbieter von Produkten für parallele Webseitentest und das Angebot wird durch den Wunsch diese Test auf mobile Geräte zu erweitern eingeschränkt. Auf Grund dessen wurden auch Segmentframeworks analysiert, welche in Kombination miteinander die Möglichkeit bieten die geforderten Kriterien zu erfüllen. Im nächsten Schritt wurden diese kurz Vorgestellt.

\\Als Nachfolgender Schritt wurde ein Evaluationsschlüssel festgelegt, welcher zum einen Teil aus geforderten Kriterien abgeleitet wurde und zum anderen im Laufe dieser Arbeit um Kriterien erweitert, welche beim einrichten und verwenden der einzelnen Werkzeuge als für die Evaluierung wichtig empfunden wurden. Anhand des Schlüssels konnten die Komplettframeworks miteinander, in ihren Stärken und Schwächen, gewertet werden.

\\Im Anschluss wurden die Frameworks installiert, konfiguriert, mit dem Desktopbrowser sowie mit einer Vielzahl von mobilen Endgeräten getestet, wobei auftretende Fehler oder Lob für eine gute problemlösende Funktion dokumentiert wurde. Am Ende jedes Frameworktestes erfolgte eine tabellarische Evaluierung in welcher die Pro und Contra ersichtlich wurden. Außerdem wurde ein Wert anhand der einzelnen Unterkategorien errechnet, welcher in der Gesamtwertung einen Vergleich der Frameworks untereinander ermöglicht.

\\Das Ziel der Arbeit war es durch das effiziente Testen von Web-Applikationen mehr Qualität zu erreichen und dies in weniger Zeit als im herkömmlichen Sinne notwendig ist. Dieses Ziel erreichte leider keins der getesteten Frameworks vollends, da sie nie alle Kriterien abdeckten und somit nach wie vor von Hand nachgetestet werden musste. 
Die einzelnen Frameworks haben in der Regel einen Schwerpunkt gut abgedeckt, jedoch dafür andere Aspekte vernachlässigt. Positiv ist das open-source Projekt Browser-Sync aufgefallen. Es erfüllte am besten die gesetzten Kriterien und ist zusätzlich um eigene Funktionalitäten erweiterbar. Zusätzlich basiert Browser-Sync auf Node.JS was es dem Entwickler ermöglicht auf die umfassende Paketdatenbank von NPM zuzugreifen und sie in das Framework zu implementieren. 

\\Auch das kommerzielle Ghostlab Produkt hat ein sehr solides Grundgerüst, jedoch gibt es grade im Bereich des mobilen Testen eine Schwachstelle die das Arbeiten, zumindest mit älteren Generationen von Mobilgeräten, fast unmöglich macht. Da es leider keine Möglichkeit bietet das Programm um eigenen Code zu erweitern und derzeit noch über keinen Master-Slave-Modus verfügt, verfängt es sich sehr schnell in einem zyklischen Deadloop.

\\Abschließend ist zu sagen, dass nach den gesetzten Kriterien zum Zeitpunkt der Erstellung der Arbeit keines der Komplettframeworks in der Lage ist den Qualitätssicherungsprozess effizient zu optimieren.

\subsection{Ausblick}

In kommenden



%%
%% ############# Glossarausgabe
%%
\clearpage
\printglossary[style=altlist,title=Glossar]

%% 
%% ############# Indexverzeichnis
\printindex
%%
%% ############# Literaturverzeichnis
%%

\begin{thebibliography}{999}

\bibitem[RJBF88]{1}Ralph E. Johnson, Brian Foote: “Designing Reusable Classes” im "Journal of Object-Oriented Programming" (1988)
\bibitem[Wiki01]{2} \url{http://de.wikipedia.org/w/index.php?title=Ajax\_(Programmierung)&oldid=129355492}
\bibitem[ScVi]{3}\url{http://irlnathan.github.io/sailscasts/blog/2013/10/10/building-a-sails-application-ep21-integrating-socket-dot-io-and-sails-with-custom-controller-actions-using-real-time-model-events/}
\bibitem[Wiki02]{4} \url{http://de.wikipedia.org/w/index.php?title=JavaScript&oldid=129548016}
\bibitem[Wiki03]{5}\url{http://de.wikipedia.org/w/index.php?title=Bildaufl%C3%B6sung&oldid=129700697}


\end{thebibliography}

\end{document}



