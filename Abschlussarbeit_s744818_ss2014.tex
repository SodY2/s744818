\documentclass[13pt,a4paper,oneside]{scrbook} %#try: report, article, book, amsart

%%
%%------------------- Eigne Kommandos
%%
%Kommando zum erzeugen eines deutlichen REMOVE Blocks
\newcommand{\tr}[1]{TOREMOVE-->\linebreak{#1} \linebreak <--TOREMOVE}

%Kommando zum erzeugen eines Index Wortes
\newcommand{\mi}[1]{\index{#1}#1}

%Kommando zum erzeugen einer freizeile
\renewcommand{\\}{\bigskip}

%Kommando zum einfügen und skalieren eines Bildes
\newcommand{\ig}[3]{
\begin{figure}[htbp]
	\centering
	\includegraphics[width=400,keepaspectratio]{#1}%
	\caption[#2]{#3}%
\end{figure}
}

\newcommand{\iga}[3]{
\begin{figure}[H]
	\centering
	\includegraphics[keepaspectratio]{#1}%
	\caption[#2]{#3}%
\end{figure}
}

\newcommand{\igp}[5]{
\begin{figure}[H]
	\centering
	\includegraphics[width=#4, height=#5,,keepaspectratio]{#1}%
	\caption[#2]{#3}%
\end{figure}
}

%Kommando zur Erzeugung der evaluationstablle
\usepackage{fp} 
\usepackage[table]{xcolor}

\newcommand{\met}[9]{

	\begin{table}[H]
	 \vspace{-20pt}
 		\centering
		\rowcolors{1}{white}{lightgray}
			\begin{tabular}{| p{8cm} | p{2cm} | p{2cm} |}
			\hline
				Komponente		 	&	Punkte	&	Wertigkeit\\
			\hline
			\hline
				Installation			&	#2	&	10 \%\\
				Konfiguration			&	#3	&	10 \%\\
				Funktion: Desktop		&	#4	&	25 \%\\
				Funktion: Mobil			&	#5	&	25 \%\\
				Erweiterbarkeit			&	#6	&	10 \%\\
				unterst{\"u}tze Browser				&	#7	&	10 \%\\
				Aktivit{\"a}t			&	#8	&	10 \%\\
				\hline
				\hline
				Gesamt				&	\FPeval\resultfp{clip((#2*1)+(#3*1)+(#4*2.5)+(#5*2.5)+(#6*1)+(#7*1)+(#8*1))} 												\resultfp 	&	100 \%\\
				\hline
				\end{tabular}
			\caption{#1}
	\end{table}

}

\newcommand{\mmet}[6]{
	\subsubsection{#1}
	 \vspace{-20pt}
	\begin{table}[H]
 		\centering
		\rowcolors{1}{white}{lightgray}
			\begin{tabular}{| p{8cm} | p{5cm} |}
			\hline
				Komponente		 &	\\
			\hline
			\hline
				Betriebssystem				&	#2	\\
				Versionsnummer			&	#3	\\
				Bildschirmdiagonale			&	#4	\\
				Aufl{\"o}sung				&	#5	\\
				priem{\"a}re Ausrichtung		&	#6	\\
				\end{tabular}
			\caption{{\"U}bersicht #1}
	\end{table}

}



%%
%%------------------- Imports
%%
\RequirePackage{ifpdf}
\usepackage{fancyunits}
\usepackage[entwurf]{bhtThesis}
\usepackage{makeidx}
\usepackage{graphicx}
\usepackage{float}

\makeindex 

%%
%% Pfad zu den Bildern
%%
\graphicspath{
  {pictures/}
}

%\usepackage{makeidx}
%\makeindex


%%
%% Titel, Autor und Betreuer
%%
\version{0.1$\alpha$}
\datum{\today}
\fachbereich{VI -- Informatik und Medien --} 
\studiengang{Medieninformatik}
\autor{Adrian Randhahn}
\edvnr{744818}
\titel{Evaluierung von Techniken zur parallel-synchronen Bedienung einer Web-Applikation auf verschiedenen mobilen Endgeräten} 
\abschluss{Bachelor of Science (B.Sc.)}

\betreuerFeld{
  \begin{tabular}{lr}
    \multicolumn{2}{l}{\textbf{Gutachter}}\\
    Prof.~Knabe & Beuth Hochschule für Technik\\
    Prof.~Dr. Wambach & Beuth Hochschule für Technik
  \end{tabular}
}

\begin{document}
\pagestyle{fancy}

%%
%% ############# Deckblatt
%%
\maketitle

\pagenumbering{roman}

%%
%% ############# Erklärung
%%
\chapter*{Erklärung}
Ich  versichere, dass  ich diese  Abschlussarbeit ohne  fremde  Hilfe selbstständig
verfasst und  nur die  angegebenen Quellen und  Hilfsmittel benutzt  habe. Wörtlich
oder dem  Sinn nach  aus anderen  Werken entnommene Stellen  sind unter  Angabe der
Quellen kenntlich gemacht.
Ich erkläre weiterhin, dass die vorliegende Arbeit noch nicht im Rahmen eines anderen Prüfungsverfahrens eingereicht wurde.
\vspace{10ex}\\
\hrule
{\small{Datum}}\hfill{\small{Unterschrift}}

%%
%% ############# Datenschutz
%%
\chapter*{Sperrvermerk}
Die vorliegende Arbeit beinhaltet interne und vertrauliche Informationen der Firma New Image Systems GmbH. Die Weitergabe des Inhalts der Arbeit im Gesamten oder in Teilen sowie das Anfertigen von Kopien oder Abschriften - auch in digitaler Form - sind grundsätzlich untersagt. Ausnahmen bedürfen der schriftlichen Genehmigung der Firma New Image Systems GmbH.

%%
%% ############# Abstract
%%
\section*{Kurzfassung}
Während der Entstehung einer Webapplikation durchläuft diese wiederholt die Qualitätskontrolle. Innerhalb dieser Arbeit werden bestehende Technologien auf deren Nutzen hin untersucht die Qualitätssicherung qualitativ zu verbessern. Die \Gls{Framework}s werden auf ihre Kompatibilität für Desktop-, sowie Mobilbrowser untersucht. Des Weiteren werden alleinstehende Frameworks, dahin gehend untersucht, ob sie in kombinierter Form in der Lage sind ein eigenständiges \Gls{Framework} zu erschaffen.


\section*{Abstract}
During the origin of a web application this runs through repeats the high-class control. Within this work existing technologies are examined for their use to improve the quality assurance qualitatively. Besides become single Frameworks, passing examined whether they in combined form in the situation are to be fulfilled the demanded criteria.

%%
%% ############# Inhaltsverzeichnis
%%
\tableofcontents

%%
%% ############# Abbildungsverzeichnis
%%
\listoffigures

%%
%% ############# Tabellenverzeichnis
%%
\listoftables

%%
%% ############# Maincontent
%%
\pagenumbering{arabic}

%%
%% ############# Einleitung
%%
\chapter{Einleitung}
%\tr{Eine Einleitung bietet die Möglichkeiten den Sinn und Zweck der Diplomarbeit für einen Durchschnittsinformatiker (ohne die Spezialkenntnisse, die Sie jetzt haben) verständlich zu beschreiben. Hier können Sie Hintergründe darstellen, wie die Arbeit und das Thema entstanden und selbstverständlich für Ihre Arbeit werben. Interessierte Leser entscheiden hier, ob diese Arbeit für sie fachlich interessant ist.}

%\tr{	Eine kurze Beschreibung des allgemeinen Forschungsgebietes in ein bis zwei Absätzen. Die Einleitung sollte am Ende in ein bis zwei Sätzen die eigentlich untersuchte Fragestellung benennen.}
%Thematische Hinführung (2 Absätze); diese besteht im Idealfall aus dem Einstieg (etwas, was an die allgemeine Erfahrung anknüpft und unmittelbar ersichtlich ist) und einem weiteren Absatz, in dem – ausgehend vom 		Einstieg – auf das eigentliche Thema fokussiert wird. In einer Arbeit über Online-Marketing mit Facebook beispielsweise würde es im 1. Absatz um Online-Marketing allgemein gehen und im 2. Absatz auf die besonderen 		Anforderungen im Zusammenhang mit Facebook verwiesen. (Kontrollfrage: „Worum geht es hier?“)


%\tr{Hier sollte der Hintergrund und die Motivation der Arbeit kurz angerissen werden. Nach Möglichkeit sollte man alle Aspekte, die im zweiten Kapitel ("Grundlagen") besprochen werden, hier schon einmal ansprechen, damit 		diese nicht später aus heiterem Himmel fallen. Insbesondere sollte der Hintergrund quasi "zwingend" den nächsten Teil der Einleitung motivieren:

%\section{Hintergrund}	
In der modernen Webentwicklung durchläuft eine Anwendung verschiedene Etappen eines Entwicklungszykluses. Er beginnt bei einem Auftrag oder einer Idee, darauf folgt dann die Spezifikation einzelner \mi{Usecases}\footnote{Szenario oder auch Anwendungsfall}. Im Anschluss folgt in der Regel die Entwicklung und Implementation\footnote{Einbindung} der einzelnen Komponenten. Am Ende der jeweiligen Implementationsphase durchläuft das Produkt\footnote{hier: einzelne Softwarekomponente} die Qualitätskontrolle. Sollten in diesem Abschnitt Fehler auftreten wird das Produkt dem Entwickler zur erneuten Bearbeitung vorgelegt. 
\\
Dieser Vorgang kann sich beliebig oft wiederholen. Bei großen und komplexen Softwaresystemen ist es trotz zeitgemäßer Implementierung nicht immer Ausgeschlossen, dass \mi{Kaskadierungsfehler}\footnote{Fehler die nicht im eigentlichen Segment auftreten, sondern eine oder mehr Ebenen weiter unten in der Systemhirarchie} entstehen. Aus Sicht der Qualitätssicherung ist dies ein lästiges Problem, da diese nach jedem erneuten Modifikationsvorganges eines Softwaresegments einen größeren Segmentblock, wenn nicht sogar das gesamte Softwareystem erneut testen muss.
\\
Bei der Entwicklung auf und für mobile Endgeräte\footnote{Smartphones, Tabletts  oder Ähnliche} kommt noch ein erschwerender Faktor hinzu, nämlich die diversen, verschiedenen Bildschirmauflösungen. Diese können nicht nur die Darstellung des Inhaltes beeinflussen, sondern auch daraus folgend die Interaktionskonformität beeinflussen.

\ig{../pictures/Entwicklungsprozess}{Entwicklungsprozess}{Vereinfachte Darstellung eines Softwareentwicklungsprozesses}

Im Optimalfall wird die Software erst nach vollständiger Homogenität auf allen unterstützen Geräten freigegeben.

%\section{Problemstellung}
\\
Dieser zyklisch wiederkehrende Prozessablauf ist sehr Zeitintensiv und nimmt linear mit der Anzahl der zu testenden Geräte zu.

%\section{Operationalisierung der Fragestellung}
\\
Das Ergebnis dieser Forschungsarbeit soll zeigen, wie verschiedene \mi{Softwareframeworks} die Zeit, die in die Qualitätssicherung investiert wird, beeinflussen können, indem sie die Steuerung diverser Geräte parallel-synchron steuern. Die Evaluierung soll zeigen wo die Vorteile und Nachteile der einzelnen Werkzeuge liegen. Weiterhin soll gezeigt werden ob aktuelle \mi{Frameworks} erweiterbar sind um Beispielsweise automatisierte \mi{Testunits} zu implementieren.
%\section{Untersuchungsverlauf}

\subsubsection{Anmerkung}
Aus Gründen der besseren Lesbarkeit wird für alle Personen und Funktionsbezeichnungen durchgängig das generische Maskulinum angewendet und bezieht in gleicher Weise Frauen und Männer ein.

%%
%% ############# Aufgabenstellung
%%
\chapter{Aufgabenstellung}
%\tr{Durch eine klare Beschreibung der Aufgabenstellung wird die zu lösende Aufgabe deutlich. Vorhandene Teillösungen oder -systeme können hier ebenfalls dargestellt werden. In vielen Fällen ist es auch hilfreich Sachverhalte oder Problemstellungen zu beschreiben, die nicht zur Aufgabenstellung gehören (Abgrenzung).}
Die Aufgaben dieser Thesis ist die Evaluierung von Techniken zur parallel-synchronen Steuerung von Webapplikationen auf mobilen Endgeräten, um damit die Produktivität der Qualitätssicherung zu optimieren.
	%%
	%% ############# Annahmen und Einschränkungen
	%%
	\section{Problemstellung}
	%\tr{Aus dem Hintergrund sollte die Wissenslücke klar werden, die durch die Abschlussarbeit geschlossen werden soll. Kurz sollte beschrieben werden, mit welchen Methoden die Arbeit versuchen will, diese zu schließen 			(empirische 	Untersuchung, Loganalyse, neuartige Programmkomponenten, etc.). Schließlich sollte man herausstellen, warum es wichtig ist, diese Wissenslücke zu schließen (wie profitiert die Welt davon).}
	Ein Problem in der aktuellen Softwareentwicklung ist die immer mehr wachsende Anzahl an Endgeräten, welche mit verschiedenen Bildschirmauflösungen und eigenen Betriebsystemen in unterschiedlichen Versionen auftreten. Ein Qualitätsprüfer der einen hohen Qualitätsstandard hat investiert daher linear zu der Anzahl der zu testenden Geräte ansteigend Zeit, lediglich um vereinzelte Testszenarien durchzuarbeiten. Solch ein Testszenraio kann Navigationsabläufe\footnote{ein Nutzerspezifischer Gang durch die Webseite}, das ausfüllen und validieren eines Formular oder auch das überprüfen funktionaler\footnote{aktive Links und deren Aufruf} Links sein. Bereits an dieser Stelle ist die zu investierende Zeit, und dies wiederholt, enorm.
\\
Wenn der Qualitätsprüfer innerhalb eines Testszenarios einen schwerwiegenden Fehler bei einem der Geräte entdeckt, muss dieser den Vorgang beenden. Abgebrochen werden muss deshalb, da bei korrigierter Implementierung der Qualitätsprüfer nicht davon ausgehen darf, das bereits kontrollierte Abschnitte immernoch voll funktionsfähig sind, da eventuell neue Fehler in bereits Kontrollierten Segmenten auftreten können.
\\
Sollte ein Szenario aufgrund eines Fehler abgebrochen worden sein, wird dem Entwickler das Problem möglichst konkret geschildert. Dessen Aufgabe ist es nun das Problem zu beheben. Ist dies geschehen startet der Prüfer einen erneuten Durchgang des Szenarios. Ein generelles Problem was hier noch zusätzlich entstehen kann, ist der Umstand, dass sich grade bei nur kleineren fixes\footnote{Problemlösungen, Codeanpassungen} und immer wieder auftretenden Testszenarioschleifen eine gewisse Routine einschleichen kann, worunter die Qualität des Produkts leidet.

\ig{../pictures/Testszenario}{Qualitätssicherung Testszenario}{Darstellung eines Qualitätssicherungsablaufes in der mobilen Anwendungsentwicklung}}
\pagebreak

	%%
	%% ############# Annahmen und Einschränkungen
	%%
	\section{Annahmen und Einschränkungen}
	%\tr{Wenn die Arbeit wichtige Annahmen trifft, unter denen die Untersuchungsergebnisse gültig sind, oder die Allgemeinheit der getroffenen Aussagen wichtigen Einschränkungen unterliegt, sollten diese ebenfalls in der 		Einleitung 	beschrieben werden.}
	
	%%
	%% ############# Zielsetzung
	%%
	\section{Zielsetzung}
	         Das Ziel dieser Arbeit ist es, bestehende \mi{Frameworks} auf ihre Tauglichkeit in Bezug auf die parallel-synchrone Steuerung von mobilen Endgeräten zur Durchführung von Testszenarien zu evaluieren.
Hierzu werden auf mobilen Endgeräten die internen Browser getestet. Hinzu kommen auf Desktopgeräten die aktuellen Versionen \tr{Versionsnummern} von Firefox, Chrome, Safari(nur für Mac-Desktopgeräte) und der Internet Explorer(nur für Windows-Desktops). Um eine Allgemeine Testbarkeit zu gewährleisten werden die Frameworks auch auf Genauigkeit in virtuellen Umgebungen analysiert. Dabei können Abweichungen, seien sie noch so klein, entstehen. Bereits 1 Pixel Abweichung kann bereits ausschlaggebend sein einen Umbruch zu erzeugen und damit das Layout negativ zu verändern.
	%%
	%% ############# Abgrenzungskriterien
	%%
	\section{Abgrenzungskriterien}
	%\tr{Hier werden die Grundlagen für das zu entwickelnde Softwaresystem definiert. Zwar noch aus fachtechnischer Sicht werden hier die Anforderungen an das geplante Softwaresystem in möglichst formaler Form spezifiziert. 	Es 	sollen hier keine Lösungen präsentiert werden, sondern möglichst präzise die Anforderungen (Sollkonzept) an das geplante Softwaresystem mit seinen Schnittstellen, Informationsflüssen und Systemfunktionen 			dokumentiert 	werden. Verwendete Methoden können z.B. SA, SADT, Petri-Netze oder andere sein. Das Ergebnis ist ein für die Systementwicklung verwendbares Pflichtenheft. Über Art und Umfang des Pflichtenhefts sollten 	Sie mit Ihrem Betreuer sprechen.}
	- Einarbeitungszeit
	\newline
	- Erweiterbarkeit in Hinsicht auf mehrerer Geräte
	\newline
	- Erweiterbarkeit des verwendeten Frameworks durch eigene Funktionen
	\newline
	- Browsersupport
	\newline
	- Virtuelle Umgebung
	- ...

%%
%% ############# Grundlagen
%%
\chapter{Grundlagen}
In diesem Abschnitt behandle ich spezifische Definitionen wie zum Beispiel verwendetes Fachvokabular, allgemeine technische Abläufe die Notwendig sind um diese Arbeit und die darin verwendetet Techniken zu verstehen, sowie verwendete Hardwarekomponenten.
%\tr{Dieser Teil beschreibt das fachliche Umfeld der Aufgabenstellung. Hier werden die wesentlichen fachlichen Begrifflichkeiten, die für die Aufgabe relevanten Problemstellungen und Lösungsansätze des Fachgebietes vorgestellt. Der Sprachgebrauch sollte einen direkten Bezug zum Fachgebiet haben. Die notwendigen Darstellungsmethoden, die Art und der Umfang der Beschreibung hängen wesentlich von der jeweiligen Fachdisziplin ab und sollten im Dialog mit dem Betreuer entschieden werden. Beispielsweise wird sich die Beschreibung eines Hotelreservierungssystems sehr von einer Beschreibung mathematischen Transformationen auf Grafikobjekte unterscheiden.Dies ist oft vor der Einleitung das erste Kapitel, das man schreibt, und sollte einen Überblick über die Literatur und existierende Arbeiten im Bereich der Arbeit liefern (Welche Grundlagen gibt es in diesem Bereich? Haben andere Autoren schon etwas zu verwandten Themen veröffentlicht?). Die hier vorgestellten Konzepte sollten in der Einleitung zumindest schon einmal angesprochen worden sein. Bei der Vorstellung verwandter Arbeiten sollten neuere Erkenntnisse bevorzugt werden. Bei jeder in das Grundlagenkapitel aufgenommenen Arbeit gilt es herauszustellen, was die Ergebnisse der Arbeit waren und warum diese Ergebnisse (oder Einschränkungen der vorgestellten Arbeit) eben noch keine Schließung der Wissenslücke oder keine Lösung der Aufgabenstellung darstellen? Am Ende erfolgt eine kurze Zusammenfassung der Grundlagen, die begründete Schlussfolgerung, dass das zu untersuchende Problem noch ungelöst ist, und ggf. wieder eine Vorschau auf das folgende Kapitel.}

	%%
	%% ############# Begriffsklärung
	%%
	\section{Begriffsklärung}	
		\subsection{parallel-synchron}
		\subsection{Web-Applikation}
		\subsection{HTML}
		Die Hypertext Markup Language ist eine Auszeichnungsprache zur Beschreibung von Inhalten. Sie dient der Strukturierung 		von Texten, Links\footnote{Verweise zu anderen Inhalten}, Listen und Bildern eines Dokumentes. Eine HTML Seite wird von 		einem Webbrowser interpretiert und anschließend dargestellt. Die Entwicklung von HTML geschieht durch das World Wide 		Web Consortium(W3C) und den Web Hypertext Application Technology Working Group (WHATWG). 

		\subsection{Webbrowser}
		\subsection{Desktopcomputer / Desktops}
		In dieser Arbeit werden gängige Modelle von Personal Computern oder Macs mit einem festen Arbeitsumfeld als Desktops 		bezeichnet. Hierzu zählen auch tragbare Modelle und Laptops. Im Sinne der Thesis umschließe ich nachfolgend mit dem 		Begriff Desktop oben genannte Komponenten. Dies dient später der Differenzierung ob es sich um ein mobiles Endgerät 			handelt oder einem Computer .

		\subsection{Mobiles Endgerät}
		Im Nachfolgenden werden Komponenten mit primärer mobiler Nutzung umfassend als mobile Endgeräte gruppiert. Hierzu 		zählen Smarthphones und Tablets.

		\subsection{Javascript}
		\subsection{Framework}
		\subsection{Nodejs}
		\subsection{PHP}
		\subsection{NPM}
		\subsection{Qualitätssicherung}
		\subsection{VirtualBox / virtuelle Umgebung}
		\subsection{Smartphone}
		\subsection{Tablet}
		\subsection{Panorama / Portrait View}
		\subsection{Pixel}
		\subsection{Auflösung}
		\subsection{Event}
		\subsection{DOM}
		\subsection{Apache}
		\subsection{Form, Checkbox, Radiobox, Inputs}
		\subsection{Workspace}
	
	%%
	%% ############# technischer Aufbau
	%%
	\section{technischer Aufbau}
	
	%%
	%% ############# Komponenten
	%%
	\section{Komponenten}
		\subsection{\mi{Raspberry Pi}}
		\subsection{Hardware}



%%
%% ############# Technologien
%%
\chapter{Technologien}
	\section{\mi{Ghostlab}}
	Ghostlab ist ein Framework des Schweizer Unternehmens Vanamco. Es verspricht das synchrone Testen von Websiten in Echtzeit. Weiterhin wirbt das Unternehmen mit einem umfangreichen Repertoire an nützlichen Fähigkeiten. Der Funktionsumfang umschliesst das Scrollen innerhalb einer Seite, das ausfüllen von  Formularen, das wahrnehmen und reproduzieren von Click-Events sowie dem neuladen einer Seite. Ghostlab soll ebenso einen Inspektor besitzen, welcher die Analyse des DOMs, der on the fly Bearbeitung der CSS und der Analyse und Bearbeitung von Javascriptdateien. Das Framework gibt an für alle folgenden Browser zu funktionieren ohne diese Konfigurieren zu müssen:

	\begin{table}[h]
 		\centering
		\rowcolors{1}{white}{lightgray}
			\begin{tabular}{| p{5cm} | p{5cm} |}
			
			\hline
				Browser 	& 	Version\\
			\hline
			\hline
				Firefox	&	latest\\
				Chrome	&	latest\\
				Safari	&	latest\\
				Internet Explorer	&	8/9/10\\
				Opera Mobile	&	supportet\\
				Opera	&	11\\
				FireFox Mobile	&	supportet\\
				Blackberry	&	supportet\\
				Windows Phone	&	supportet\\
				Safari mobile	&	supportet\\	
				Android	&	2.3 - 4.2\\
				\hline
				\end{tabular}
			\caption{von Ghostlab getestete Browser (stand 10.03.2014, Version 1.2.3)}
	\end{table}

	Der Kostenpunkt der Lizenz liegt zur Erstellung dieser Arbeit bei 49\$ (entspricht 35,30€ beim aktuellen Umrechnungswert). Zur 	Erstellung dieser Thesis wurde die 7-Tage-Testvollversion genutzt.
	
	\section{\mi{Adobe Edge Inspect}}
	Die Anwendung Edge Inspect stammt von Adobe und wird derzeit in der CC\footnote{Creative Cloud} Version vertrieben. Um 		Adobe Edge Inspect nutzen zu können bedarf es 3 separaten Komponenten. Adobe wirbt mit synchronem aufrufen und 			auffrischen von Websites, sowie deren Inspizierung per Weinre. Besonders angepriesen wird von Adobe die Nutzung und 		Verwendung der Adobe Edge Inspect API, welche auf GitHub zur Verfügung gestellt wird. Des weiteren kann Adobe Edge 		Inspect in andere Edge Produkte\footnote{zum Beispiel Edge Reflow CC und Edge Code CC} innegriert werden. 
	
	\\Adobe Edge Inspect CC steht 30 Tage kostenlos zum testen bereit. Danach fallen ab 24,59/ Monat für die Nutzung des 			Einzelprodukt-Abos an.
	
	\\Die Anwendung läuft nur auf mobilen Endgeräten mit iOS oder Android Betriebssystem.
	
	\section{\mi{Remote Preview}}
	Remote Preview ist ein kleines Javascript Framework von dem Web Designer Viljami Salminen aus Helsinki, Finnland. Es überprüft alle 1100ms per AJAX-Request ob sich die Quellurl geändert hat und teilt dies dann den verbundenen Testgeräten mit. Er wirbt 	mit dem synchronen Aufruf von Websiten auf einer Vielzahl von Plattformen: 
	
	\begin{table}[h]
 		\centering
		\rowcolors{1}{white}{lightgray}
			\begin{tabular}{| p{13cm} |}
			
			\hline
				Plattform\\
			\hline
			\hline
				Android OS 2.1 - 4.1.2 (Default browser + Chrome)\\
				Blackberry OS 7.0 (Default browser)\\
				iOS 4.2.1 - 6 (Default browser)\\
				Mac OS X (Safari, Chrome, Firefox, Opera)\\
				Maemo 5.0 (Default browser)\\
				Meego 1.2 (Default browser)\\
				Symbian 3 (Default browser)\\
				Symbian Belle (Default browser)\
				WebOS 3.0.5 (Default browser)\\
				Windows Phone 7.5 (Default browser)\\	
				Windows 7 (IE9)\\
				\hline
				\end{tabular}
			\caption{von Remote Preview unterstützte Plattformen (stand 19.03.2014, letzter Commit 7dc48caa84)}
	\end{table}
	Das Framework ist Kostenlos erhältlich und läuft unter der MIT Lizenz. Zum Zeitpunkt dieser Arbeit scheint das Projekt nicht weiter entwickelt zu werden, da seit 5 Monaten auf der Projektseite keinerlei Aktualisierungen vorgenommen wurden.


	\section{\mi{NodeJS}}
	\section{\mi{Zombie.js}}
	\section{\mi{W3C Touch Events Extensions}}
	\section{\mi{Phantom Limb}}
	\section{\mi{jQuery} \mi{UI Touch Punch}}
	\section{\mi{jQuery} \mi{Touchit}}
	\section{\mi{NPM} \mi{touchit}}
	
	\section{\mi{Adobe Edge Inspect}}
	\section{\mi{Remote Preview}}
	\section{\mi{Browser-Sync}}
	\section{Eigenes \mi{Framework}}

		
%%
%% ############# Evaluation
%%
\chapter{Evaluation der Techniken}
\section{Auflistung des Evaluationsschlüssels}

\begin{enumerate}
\item Installation 
	\begin{enumerate}
		\item sind zusatzinstallationen notwendig (basiert das Werkzeug auf anderen Technologien) 4Pt
		\item kann nach der Installation die Software direkt genutzt werden ? 2Pt
		\item gibt es eine zur Version passende installationsanleitung? 2Pt
		\item gibt es eine FAQ? 2Pt
	\end{enumerate}
	
	\item Konfiguration
	\begin{enumerate}
		\item mann die Software out-of-the-Box\footnote{ohne weitere Konfiguration nach der Installation} genutzt werden ? 4Pt
		\item ist die Software Konfigurierbar in Hinsicht auf IP-Adressen und Ports? 1Pt
		\item kann man verschiedene Konfigurationen abspeichern ? (für z.B. verschiedene Arbeitsumgebungen) 2Pt
		\item gibt es Support (Wiki, Helpdesk, EMail, Forum)? 2Pt
		\item Intuitive Benutzeroberfläche 1Pt
	\end{enumerate}
	
	\item Funktion: Desktop
	\begin{enumerate}
		\item Darstellung : normale Seiten 2Pt
		\item Darstellung : gesicherte Seiten 2Pt
		\item Darstellung: normale( < 1 Sekunde) Reaktionsgeschwindigkeit 2Pt
		\item Funktion: Seitensteuerung 3Pt
		\item Funktion: Javascript 1Pt
	\end{enumerate}
	
	\item Funktion: Mobil
	\begin{enumerate}
		\item Darstellung : normale Seiten 1Pt
		\item Darstellung : gesicherte Seiten 1Pt
		\item Darstellung: normale( < 1 Sekunde) Reaktionsgeschwindigkeit 2Pt
		\item Funktion: Seitensteuerung 4Pt
		\item Funktion: Gestenkontrolle 1Pt
		\item Funktion: Javascript 1Pt
	\end{enumerate}
	
	\item Erweiterbarkeit
	\begin{enumerate}
		\item gibt es eine API zur Implementierung in eigenen Entwicklungen ? 5Pt
		\item Ist die API rechtlich durch Lizenzen geschützt? 2Pt
		\item ist die API Dokumentiert? 3Pt
	\end{enumerate}
	
	\item unterstützte Browser (aktuelle Version zum Zeitpunkt der Erstellung dieser Thesis)
	\begin{enumerate}
		\item mobile Plattformen (iOS, Android, Windows) 3Pt
		\item Unterstützung von Browser innerhalb einer virtuellen Umgebung 2Pt
		\item Chrome 1Pt
		\item Opera 1Pt
		\item Firefox 1Pt
		\item Safari 1Pt
		\item Internet Explorer 1Pt
	\end{enumerate}
	
	\item Aktivität
	\begin{enumerate}
		\item wird die Software noch entwickelt? (letzes Release, Commit Häufigkeit) 5Pt
		\item gibt es ein aktives Forum (letzter Beitrag jünger 14 Tage) 5Pt
	\end{enumerate}

\end{enumerate}

%%
%% ############# Ghostlab
%%
	\pagebreak
	\section{\mi{Ghostlab} Version 1.2.3}
		\subsection {Einrichtung der Testumgebung}
		Ghostlab kommt von Hause aus mit einer 7-Tage-Testversion. Die Installation verlief einfach und ereignislos. Nachdem das 		Tool Installiert wurde erfolgte die Zuweisung einer Website zu dem Ghostlabserver. Es wurden in diesem Fall sowohl eine 		Seite auf einem lokalen Apache Server getestet, als auch die mitgelieferte Demoseite von Ghostlab. Nach dem Start des 			Ghostlabservers ist dieser über den localhost\footnote{IP-Adresse des lokalen Rechners} auf Port 8005 (Default) von allen 		zu testenden Geräten erreichbar.
		\ig{../pictures/ghostlab/startbildschirm}{Startbildschirm Ghostlab}{Startbildschirm von Ghostlab nach der Installation}
		
		\subsection{Testen von Desktopbrowsern}
		Durch aufrufen der IP-Adresse des Rechners auf dem der Ghostlabserver läuft verbindet sich der Browser als Client und 			wird fortan durch gesendete Signale beeinflusst. Hierzu zählen auch virtuelle Browser. Jeder Client wird nun gleichzeitig 			Sender und Empfänger für Signale, dass bedeutet das jede Aktion parallel-synchron auf allen anderen Clients gespiegelt 			wird. Hierzu zählen Javascriptevents, das ausfüllen eines Formulars oder das neuladen der gesamten Seite.
		\ig{../pictures/ghostlab/workspaces}{Übersicht Clients}{Darstellung von 4 verschiedenen Clients } 
		
		Über den Übersichtsbildschirm kann jeder verbundene Client einzeln inspiziert werden. Hier ist der Nutzer in der Lage sich 		durch das DOM zu navigieren oder temporäre CSS Anpassungen vorzunehmen. Die Handhabung ist intuitiv, was jedoch an 		dem verwendeten Framework \mi{Weinre} liegt.
		\ig{../pictures/ghostlab/weinre}{Exemplarisch Weinreansicht}{ausgewähltes DOM-Element in Weinre}
		
		\pagebreak
		\subsection{Testen von mobilen Browsern}
		
		Das einrichten zum testen auf mobilen Endgeräten verläuft synchron zu den Desktopbrowsern. Man ruft innerhalb des 			Browsers die IP-Adresse des Ghostlabrechners auf und ist schon nach wenigen Sekunden\footnote{abhängig von der 			Geschwindigkeit des Testgerätes} in der Clientliste aufgenommen.
		
		\\Bei dem Testen auf mobilen Browserns ist es bei Ghostlab\footnote{Version 1.2.3} Notwendig ausreichend Zeit zwischen 		den Eingaben zu lassen, da es sonst bei unterschiedlich schnellen Geräten zu einem Effekt kommt, bei dem die 				langsameren Geräte beim ausführen des Letzen Signals gleichzeitig wieder zum Sender für alle anderen Geräte wird.
		\ig{../pictures/ghostlab/uebersicht_mobil}{Übersicht mobile Clients Ghostlab}{Ghostlabübersicht der verbundenen Clients}
		
		\pagebreak
		
		\subsection{Fazit zu Ghostlab}
		Zum Stand dieser Arbeit wurde Version 1.2.3 von Ghostlab genutzt. Zu diesem Zeitpunkt verfügte die Software noch über 		keinen Master/Slave-Modus\footnote{ein Gerät dient als Steuergerät, alle anderen folgen ihm}, dadurch kam es bei meinen 		Testgeräten bereits nach wenigen Minuten zu dem Problem, dass die Geräte sich in einer 								Endlosschelife von Senden und Empfangen der Steuerbefehle befanden. Für kommende Versionen ist ein solcher Modus 		laut den Entwicklern aber geplant. Das Problem rührt daher, dass einige Geräte schneller auf die übermittelten Befehle 			reagieren als andere. Das führt dazu, dass die langsam ladenden Geräte in dem Augenblick wo sie das Signal umsetzen, 		für die schnelleren Geräte bereits wieder als Sender fungieren. Dieses Problem sehe ich bei einer bereits kleinen Anzahl 			von Geräten als kritisch an. 

		\\Das testen in mehreren Browsern auf einem Rechner lief hingegen sehr gut. Das ausführen von Javascript läuft 				einwandfrei. Das ausfüllen von Inputs, Checkboxen, Radioboxen und das absenden des Formulars funktionierte bis auf die 		Kalenderauswahl im Firefox Browsers anstandslos. Ein Problem scheint das Werkzeug mit Passwortgeschützten Seiten zu 		haben. Diese lassen sich erst nach mehrfacher, abhängig vom jeweiligen Browser, Eingabe des Passwortes aufrufen. 			Diese Prozedur wiederholt sich für jede weitere Unterseite erneut. 

		\\Das arbeiten in einer Virtuellen Umgebung\footnote{es wurde VirtualBox von Oracle genutzt} wird problemlos unterstützt. 		Das einzige Problem was ich analysieren konnte war, dass sich virtuelle Browser nicht in einen Workspace integrieren 			lassen.

		\\Ghostlab unterstützt die Funktion von Workspaces\footnote{Arbeitsumgebung oder auch Arbeitsumfeld}, welche sich die 		Position und Größe der verschiedenen Browserfenster speichert. Per Knopfdruck lassen diese sich dann im Kollektiv öffnen 		sofern in den Browsereinstellungen die Popups aktiviert sind für die zu testende Seite. Dieses Feature\footnote{Funktion 			welche ein Teil der Anwendung ist} bewerte ich als Positiv in Hinsicht der Zeitersparnis, diesen Vorgang immer wieder von 		Hand auszuführen.

		\\Als Kritikpunkt bewerte ich die nicht existente Möglichkeit die Anwendung um eigene Funktionalität zu erweitern.

		\subsection{Tabellarische Evaluation}
			\met{Gewichtungstabelle Evaluation von Ghostlab}{10}{10}{8}{5}{0}{10}{5}
	
	\pagebreak
	\section{\mi{Adobe Edge Inspect} CC }
		\subsection {Einrichtung der Testumgebung}
		Es sind 3 Schritte Notwendig Adobe Edge Inspect zum Einsatz bereit zu machen. Als erstes benötigen wir den Client aus 		der Adobe Creative Cloud (CC) Kollektion. Diese gibt es zum Zeitpunkt dieser Arbeit in verschiedenen Modellen und 			beginnt bei der kostenlose 30-Tage Testversion, geht über die Einzellizenz, für ausschliesslich Adobe Edge Inspect, von 			24,59€ / Monat bis hin zum Komplett-Abo was dann mit 61,49€ / Monat zu Buche schlägt. Dieser wird gestartet und läuft 		ab diesem Zeitpunkt als Deamon im Hintergrund. 
		\iga{../pictures/adobeedgeinspect/icon}{Adobe Edge Inspect Deamon Icon}{Der laufende Deamon von Adobe Edge Inspect}
		
		\\Als zweiten Schritt benötigen wir die zugehörige Chrome Extension von Adobe Edge Inspect. Diese wird über den Chrome 		Appstore installiert und kann nach einem Browserneustart aktiviert werden.
		
		\\Als letzes benötigen wir noch die kostenlos erhältliche App aus dem jeweiligen Shop. hier gilt für Android der Play Store, 		für iOS Geräte der AppStore. Windowsgeräte werden derzeit nicht unterstützt.
		
		\\Sind diese 3 Schritte erfolgreich durchgeführt worden, müssen nun die Geräte mit dem Server verbunden werden. Hierzu 		wird die App gestartet (der folgende Prozess verläuft unter Android wie auch unter iOS identisch) und per IP-Adresse mit 			der Adobe Edge Inspect Chrome Extension verbunden werden. Diese verlangt im Gegenzug einen Identifikationscode, 			welcher auf dem jeweiligen Gerät generiert wurde. Nach erfolgreicher Synchronisation wird das Gerät im Gerätemanager 		angezeigt.
		\igp{../pictures/adobeedgeinspect/iphone_2}{Adobe Edge Inspect App Client hinzufügen}{Eingabe der IP-Adresse zum 			Edge Inspect Rechner}{200}{350}
		\iga{../pictures/adobeedgeinspect/desktop_2}{Adobe Edge Inspect Chrome Extension}{Eingabe des Sicherheitscodes in die 		Chrome Extension}

		\\Dieser hat mehrere Funktionen. Er liefert eine Übersicht aller verbundenen Clients und ermöglicht das aufrufen von 			\mi{Weinre} um z.B. das DOM zu inspizieren, verwendete Ressourcen zu inspizieren oder Javascript auszuführen. Über 			den Gerätemanager lassen sich auch verbundene Geräte wieder durch einen Klick entfernen. Desweiteren kann man über 		dieses Interface Screenshot von allen verbundenen Geräten im aktuellen Zustand aufnehmen und anzeigen lassen. 			Weiterhin besteht die Möglichkeit den Darstellungsmodus auf den verbundenen Clients von der Appdarstellung in den 			Vollbildmodus zu wechseln.
		\iga{../pictures/adobeedgeinspect/desktop_3}{Adobe Edge Inspect Gerätemanager}{Übersicht der verbundenen Clients}
		
		\subsection{Testen von Desktopbrowsern}
		Es gibt zum Zeitpunkt der Erstellung dieser Arbeit keine Möglichkeiten Desktopseiten mit Adobe Edge Inspect zu testen.
		
		\subsection{Testen von mobilen Browsern}
		Die Funktionalität zum testen von mobilen Seiten beschränkt sich derzeit nur auf den synchronen Aufruf von Seiten über 			den Chromebrowser mit installierter Extension als Steuergerät. Die verbundenen Geräte erkennen den Aufruf von Links 			und das wechseln von Tabs innerhalb des Browsers. Es besteht wie bereits beschrieben die Option die einzelnen Clients 			per \mi{Weinre} zu untersuchen.
		
		\\Die simulierung eines Scrollevents oder das ausfüllen eines Formulars ist nicht möglich. Es werden lediglich die 				Informationen Dargestellt die am Steuergerät aufgerufen wurden. Jedoch wird der Client, sofern vorhanden auf die mobile 		Seite weitergeleitet. Während des Testens in der App wird das Display aktiv gehalten, wodurch es sich nicht von selbst 			abschaltet. Ein gutes Feature von Adobe Edge Inspect ist die Möglichkeit aus dem Gerätemanager des Browsers 				Screenshots der verbundenen Geräte anzufordern. Diese werden zusammen mit einer Beschreibung des Geräts, dessen 		Modellbezeichnung , die Auflösung sowie Pixeldichte, dem Betriebssystems, der aufgerufenen URL sowie der aktuellen 			Ausrichtung des Bildschirms ausgeliefert.
		\igp{../pictures/adobeedgeinspect/iphone_3}{Adobe Edge Inspect App Content Darstellung}{Darstellung von Content in der 		Adobe Edge Inspect App unter iOS}{200}{350}
		
		Während meiner Versuche ist mir aufgefallen, dass Adobe Edge Inspect unter iOS 6.1.3, Seiten die durch htaccess 				gesichert Sind nicht darstellen kann. Auf den anderen Testgeräten verlief der Prozess der Authentifizierung problemlos. 
		
		\pagebreak
		\subsection{Fazit zu Adobe Edge Inspect}
		Adobe Edge Inspect bedarf viel Aufwand für ein relativ geringes Ergebnis. Man muss an 3 verschiedenen Punkten 				Installationen vornehmen, die dann jedoch ohne Probleme miteinander harmoniert haben. Als besonders Positiv möchte ich 		die Screenshotfunktion bewerten. In Zusammenspiel mit der öffentlich zugänglichen API lassen sich hierüber Screenshots 		im Landschafts, als auch im Portraitmodus anfordern und durch eine externe Applikationen auswerten. 
		
		\\Der Nutzen des Werkzeugs liegt am ehesten bei One-Page-Sites\footnote{Webseiten dessen Inhalt sich füllend auf die 			gesamte Seite erstrecken} oder für Fehlersuche innerhalbs des DOM oder CSS Anpassungen mit \mi{Weinre}. Unter dem 		Aspekt 	des parrallel-synchronen Testens ist Adobe Edge Inspect leider nicht sinnvoll zu verwenden, da wed er 				Steuerbefehle oder 		andere Gesten umgesetzt werden, noch werden die Nutzereingaben in Eingabefeldern mit 			anderen verbundenen Clients geteilt. 	Alle verbunden Clients sind nur Empfänger und besitzen keine Möglichkeit 			als 			Sender zu fungieren. Folglich gehen alle 			Steuerbefehle vom Edge-Server aus.
	
	\subsection{Tabellarische Evaluation}
		\met{Gewichtungstabelle Evaluation von Adobe Edge Inspect}{10}{7}{0}{4}{8}{3}{10}

				
	\section{\mi{Remote Preview}}
		\subsection {Einrichtung der Testumgebung}
		Es gibt zwei Möglichkeiten dieses Werkzeug zu nutzen. Die eine ist die Installation auf einem lokalen Apache-Server mit 			PHP. Die andere ist die Installation auf einem Cloud-Dienst wie z.B. Dropbox. Die Ergebnisse dieser Arbeit hab ich mit der 		lokalen Apache Installation erzielt. Die Installation sieht lediglich vor das Framework in einen lokalen Entwicklungszweig zu 		entpacken.
		
		\subsection{Testen von Desktopseiten}
		Alle Clients die in die Testumgebung eingebunden werden sollen müssen lediglich die IP-Adresse des Servers eingeben.
		Die Steuerung der Seiten erfolgt sowohl für Desktopseiten als auch für die mobilen Vertreter über die Browsermaske des 			Frameworks. In das untere der beiden Eingabefelder gibt man die aufzurufende URL inklusive Präfix\footnote{http://} ein. 			Diese wird dann auf allen verbundenen Clients innerhalb eines iFrames dargestellt. 
		\ig{../pictures/remotepreview/eingabemaske}{Remote Preview Steuerungsmaske}{Steuerungsmaske zur Eingabe der 			aufzurufenden URL}
				
		 \subsection{Testen von mobilen Browsern}
		 Das Testen der mobilen Browser funktioniert parallel zum testen von Desktopseiten. Positiv möchte ich hier erwähnen, 			dass das Framework auch wenn es dafür nicht ausgelegt ist, dennoch unter aktuellen Windowsgeräten funktioniert.			
		
		\subsection{Fazit zu Remote Preview}
		Ein positiver Punkt ist die Möglichkeit letztendlich jeden Browser unabhängig von dessen Betriebssystems in die 				Testumgebung zu integrieren, da diese einfach nurauf den ApacheServer oder die Dropbox zugreifen müssen. Als Negativ 		führe ich hier die Tatsache auf das es ähnlich Adobe Edge Inspect lediglich dem Aufruf von Seiten dient, jedoch nicht 			dessen Bedienung. So ist es nicht möglich weiteren Verlinkungen zu folgen ohne diese von Hand in der Eingabemaske 			einzutragen oder Formulare auszufüllen. Bedingt funktioniert das Darstellen von Seiten mit Ankern. Das Aufrufen von 			gesicherten Seiten gelang mir nicht. Ebenfalls war es  mir nicht möglich zertifizierte Webseiten aufzurufen, was den 				Nutzungsgrad des Frameworks stark einschränkt. Gut finde ich die Tatsache das Quellcode komplett zugänglich ist und 			jederzeit in eigene Projekte eingebunden oder um eigene Funktionalität erweitert werden kann.	
		
				
		\subsection{Tabellarische Evaluation}
		\met{Gewichtungstabelle Evaluation von Remote Preview}{8}{6}{3}{3}{8}{10}{0}
	
		
		
		
	
	\section{\mi{Browser-Sync}}
	\section{Eigenes \mi{Framework}}
		\subsection{\mi{Systementwurf}}
		\subsubsection{\mi{Ablaufdiagram}}
		\subsubsection{\mi{Klassendiagramm}}

\chapter{Ausblick}

\printindex

\end{document}



