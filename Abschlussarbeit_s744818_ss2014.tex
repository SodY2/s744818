\documentclass[13pt,a4paper,oneside]{scrbook} %#try: report, article, book, amsart

%%
%%------------------- Eigne Kommandos
%%
%Kommando zum erzeugen eines deutlichen REMOVE Blocks
\newcommand{\tr}[1]{TOREMOVE-->\linebreak{#1} \linebreak <--TOREMOVE}

%Kommando zum erzeugen eines Index Wortes
\newcommand{\mi}[1]{\index{#1}#1}

%Kommando zum erzeugen einer freizeile
\renewcommand{\\}{\bigskip}

%Kommando zum einfügen und skalieren eines Bildes
\newcommand{\ig}[3]{
\begin{figure}[htbp]
	\centering
	\includegraphics[width=400,keepaspectratio]{#1}%
	\caption[#2]{#3}%
\end{figure}
}

%Kommando zur erstellung eines Zitats
\newcommand{\z}[2]{
	\begin{quote}
			" #1 " \cite{#2}
	\end{quote}
} 

\newcommand{\iga}[3]{
\begin{figure}[H]
	\centering
	\includegraphics[keepaspectratio]{#1}%
	\caption[#2]{#3}%
\end{figure}
}

\newcommand{\igp}[5]{
\begin{figure}[H]
	\centering
	\includegraphics[width=#4, height=#5,,keepaspectratio]{#1}%
	\caption[#2]{#3}%
\end{figure}
}

%Kommando zur Erzeugung der evaluationstablle
\usepackage{fp} 
\usepackage[table]{xcolor}

\newcommand{\met}[9]{

	\begin{table}[H]
	 \vspace{-20pt}
 		\centering
		\rowcolors{1}{white}{lightgray}
			\begin{tabular}{| p{8cm} | p{2cm} | p{2cm} |}
			\hline
				Komponente		 	&	Punkte	&	Wertigkeit\\
			\hline
			\hline
				Installation			&	#2	&	10 \%\\
				Konfiguration			&	#3	&	10 \%\\
				Funktion: Desktop		&	#4	&	25 \%\\
				Funktion: Mobil			&	#5	&	25 \%\\
				Erweiterbarkeit			&	#6	&	10 \%\\
				unterst{\"u}tzte Browser				&	#7	&	10 \%\\
				Aktivit{\"a}t			&	#8	&	10 \%\\
				\hline
				\hline
				Gesamt				&	\FPeval\resultfp{clip((#2*1)+(#3*1)+(#4*2.5)+(#5*2.5)+(#6*1)+(#7*1)+(#8*1))} 												\resultfp 	&	100 \%\\
				\hline
				\end{tabular}
			\caption{#1}
	\end{table}

}

\newcommand{\mmet}[6]{
	\subsubsection{#1}
	 \vspace{-20pt}
	\begin{table}[H]
 		\centering
		\rowcolors{1}{white}{lightgray}
			\begin{tabular}{| p{8cm} | p{5cm} |}
			\hline
				Komponente		 &	\\
			\hline
			\hline
				Betriebssystem				&	#2	\\
				Versionsnummer			&	#3	\\
				Bildschirmdiagonale			&	#4	\\
				Aufl{\"o}sung				&	#5	\\
				prim{\"a}re Ausrichtung		&	#6	\\
				\end{tabular}
			\caption{{\"U}bersicht #1}
	\end{table}

}

%%
%%------------------- Imports
%%
\RequirePackage{ifpdf}
\usepackage{fancyunits}
\usepackage[entwurf]{bhtThesis}
\usepackage{makeidx}
\usepackage{graphicx}
\usepackage{float}
\usepackage[toc]{glossaries}

\makeindex 


 
\makeglossaries
 
 
%%
%% ======================= GLOSSAR ANFANG
%% setzen
%% makeglossaries Abschlussarbeit_s744818_ss2014
%% setzen
%% 
 
\newglossaryentry{Ajax}
{
    name=AJAX,
    description={\z{Ajax (Apronym von engl. Asynchronous JavaScript and XML) bezeichnet ein Konzept der asynchronen 				Datenübertragung zwischen einem Browser und dem Server. Dieses ermöglicht es, HTTP-Anfragen durchzuführen, während 		eine HTML-Seite angezeigt wird und die Seite zu verändern, ohne sie komplett neu zu laden}{2}
	}
}

 \newglossaryentry{moEn}
{
    name=Endgerät,
    description={Komponenten mit primärer mobiler Nutzung werden umfassend als mobile Endgeräte gruppiert. Hierzu zählen 			Smarthphones und \Gls{Tablet}s}
}

 \newglossaryentry{Computer}
{
    name=Computer,
    description={In dieser Arbeit werden gängige Modelle von Personal Computern oder Macs mit einem festen Arbeitsumfeld als Computer 	bezeichnet. Hierzu zählen auch tragbare Modelle und Laptops. Im Sinne der Thesis umschließe ich nachfolgend mit dem 		Begriff Desktop oben genannte Komponenten. Dies dient später der Differenzierung ob es sich um ein mobiles Endgerät 			handelt oder einem Computer}
}


 \newglossaryentry{Framework}
{
    name=Framework,
    description={\z{Ein Framework ist eine semi-vollständige \Gls{App}likation. Es stellt für \Gls{App}likationen eine wiederverwendbare, 			gemeinsame Struktur zur Verfügung. Die Entwickler bauen das Framework in ihre eigene \Gls{App}likation ein, und erweitern es 		derart, dass es ihren spezifischen Anforderungen entspricht. Frameworks unterscheiden sich von Toolkits dahingehend, dass sie 	eine kohärente Struktur zur Verfügung stellen, anstatt einer einfachen Menge von Hilfsklassen.}{1}
    Der Einfachheit halber wurden Sammlungen die nach dieser Definition eventuell unter den Begriff eines Toolkits fallen, ebenfalls als Framework betitelt
	}
}
 
\newglossaryentry{Webbrowser}
{
    name=Browser,
    description={\Gls{Computer}programm zur Darstellung von Inhalten des World Wide Web}
}

\newglossaryentry{HTML}
{
    name=HTML,
    description={Die Hypertext Markup Language ist eine Auszeichnungsprache zur Beschreibung von Inhalten. Sie dient der Strukturierung von Texten, Links\footnote{Verweise zu anderen Inhalten}, Listen und Bildern eines Dokumentes. Eine HTML Seite wird von einem Webbrowser interpretiert und anschließend dargestellt. Die Entwicklung von HTML geschieht durch das World Wide Web Consortium (W3C) und den Web Hypertext \Gls{App}lication Technology Working Group (WHATWG) }
}

\newglossaryentry{Javascript}
{
    name=Javascript,
    description={\z{JavaScript (kurz JS) ist eine Skriptsprache, die ursprünglich für dynamisches HTML in Webbrowsern entwickelt wurde, um Benutzerinteraktionen auszuwerten, Inhalte zu verändern, nachzuladen oder zu generieren und so die Möglichkeiten von HTML und CSS zu erweitern. Heute findet JavaScript auch außerhalb von Browsern Anwendung, so etwa auf Servern und in Microcontrollern}{4}}
}

\newglossaryentry{NodeJS}
{
    name=node.js,
    description={Plattform um serverseitige eventgesteuerte Javascriptanwendungen zu entwickeln}
}

\newglossaryentry{PHP}
{
    name=PHP,
    description={Eine an C und Perl angelehnte Scriptsprache für dynamische Webseiten}
}

\newglossaryentry{Devicelab}
{
    name=Devicelab,
    description={Ein physikalischer Ort, an dem es eine Vielzahl an miteinander verbundenen Mobilgeräten und \Gls{Computer}n gibt}
}

\newglossaryentry{NPM}
{
    name=NPM,
    description={Node Packaged Modules, eine Software zur Installation von NodeJS Bibliotheken}
}

\newglossaryentry{qs}
{
    name=Qualitätssicherung,
    description={Station, welche ein Produkt (hier die Anwendung) durchlaufen und bestehen muss, um eine gewisse Qualität zu gewährleisten}
}

\newglossaryentry{VirtualBox}
{
    name=VirtualBox,
    description={Virtuelle Desktopumgebung von Oracle. Wird genutzt um zum Beispiel ein anderes Betriebssystem, als das eigentlich genutzte zu emulieren}
}

\newglossaryentry{Smartphone}
{
    name=Smartphone,
    description={Mobiltelefon mit \Gls{Computer}ähnlicher Struktur, meist mit Touchdisplay ausgestattet}
}

\newglossaryentry{Tablet}
{
    name=Tablet,
    description={Tragbarer flacher \Gls{Computer} mit einem Touchscreen}
}

\newglossaryentry{PoW}
{
    name=Portraitmodus,
    description={Vertikalausrichtung des Bildschirms eines mobilen Endgerätes}
}

\newglossaryentry{PaW}
{
    name=Panorama View,
    description={Horizontalausrichtung des Bildschirms eines mobilen Endgerätes}
}

\newglossaryentry{Pixel}
{
    name=Pixel,
    description={Auch bekannt als Bildpunkt. Farbwert einer digitalen Rastergrafik}
}

\newglossaryentry{BA}
{
    name=Auflösung,
    description={\z{Die Bildauflösung ist ein umgangssprachliches Maß für die Bildgröße einer Rastergrafik. Sie wird durch die Gesamtzahl der Bildpunkte oder durch die Anzahl der Spalten (Breite) und Zeilen (Höhe) einer Rastergrafik angegeben.}{5}}
}

\newglossaryentry{Viewport}
{
    name=Viewport,
    description={Der Viewport bezeichnet hier den Sichtbereich eines Bildes}
}

\newglossaryentry{Event}
{
    name= Event,
    description={Übersetzbar als Ereignis. In der Ereignisgesteuerten Programmierung löst ein Event einen Funktionsaufruf aus}
}

\newglossaryentry{DOM}
{
    name=DOM,
    description={\z{Document Object Model (DOM) ist eine Spezifikation einer Schnittstelle für den Zugriff auf HTML- oder XML-Dokumente. Sie wird vom World Wide Web Consortium definiert.}{6}}
}

\newglossaryentry{Apache}
{
    name= Apache,
    description={\z{Der Apache HTTP Server ist ein quelloffenes und freies Produkt der Apache Software Foundation und der meistbenutzte Webserver im Internet.}{7}}
}

\newglossaryentry{Input}
{
    name= Input,
    description={Verschiedene Eingabefelder in einem HTML-Dokument, zum Beispiel Texteingabe oder \Gls{Checkbox}en}
}

\newglossaryentry{Checkbox}
{
    name= Checkbox,
    description={Eingabefeld zum Anhaken. Besitzt zwei Zustände, wahr und falsch}
}

\newglossaryentry{Radiobox}
{
    name= Radiobox,
    description={Eingabefeld für Einfachauswahl von mehreren Möglichkeiten. Besitzt zwei Zustände, wahr und falsch}
}

\newglossaryentry{Anker}
{
    name= Anker,
    description={\z{Ein Anker bezeichnet eine Sprungmarke innerhalb eines HTML-Dokuments.}{8} Diese wird in der Regel durch ein Hashtag (\#) gekennzeichnet}
}

\newglossaryentry{gesichert}
{
    name= gesichert,
    description={\z{Ein digitales Zertifikat ist ein digitaler Datensatz, der bestimmte Eigenschaften von Personen oder Objekten bestätigt und dessen Authentizität und Integrität durch kryptografische Verfahren geprüft werden kann. Das digitale Zertifikat enthält insbesondere die zu seiner Prüfung erforderlichen Daten.}{9}}
}

\newglossaryentry{Workspace}
{
    name= Workspace,
    description={Als Workspace wird eine Arbeitsumgebung im physischen aber auch im virtuellen Bereich bezeichnet}
}

\newglossaryentry{Commit}
{
    name= Commit,
    description={Ein Begriff aus der Versionsverwaltung, welcher eine bestätigte Änderung an einem (Software)projekt bezeichnet}
}

\newglossaryentry{Deamon}
{
    name= Daemon,
    description={Ein Daemon ist ein im Hintergrund laufendes Programm, welches verschiedene Funktionalitäten bereitstellt}
}

\newglossaryentry{App}
{
    name= App,
    description={Eine Applikation bezeichnet eine Anwendung oder auch ein \Gls{Computer}programm, welche gestellte Probleme beheben soll}
}

\newglossaryentry{htaccess}
{
    name= htaccess,
    description={\z{.htaccess-Dateien sind Server-Konfigurationsdateien für Verzeichnisse, die zu Ihrem Web-Angebot gehören. So ist die .htaccess-Technik beispielsweise der übliche Weg, um nur bestimmten Benutzern den Zugriff auf bestimmte Daten zu erlauben.}{10}}
}

\newglossaryentry{GPL}
{
    name= GNU General Public License,
    description={\z{Die GNU General Public License (auch GPL oder GNU GPL) ist die am weitesten verbreitete Software-Lizenz, welche den Endnutzern (Privatpersonen, Organisationen, Firmen) die Freiheiten garantiert, die mit ihr lizenzierte Software nutzen, studieren, verbreiten (kopieren) und ändern zu dürfen. Software, die diese Freiheitsrechte gewährt, wird Freie Software genannt. Die Lizenz wurde ursprünglich von Richard Stallman der Free Software Foundation (FSF) für das GNU-Projekt geschrieben.}{11}}
}

\newglossaryentry{MIT}
{
    name= MIT Lizenz,
    description={\z{Die MIT-Lizenz, auch X-Lizenz oder X11-Lizenz genannt, ist eine aus dem Massachusetts Institute of Technology stammende Lizenz für die Benutzung verschiedener Arten von \Gls{Computer}software. Sie erlaubt die Wiederverwendung der unter ihr stehenden Software sowohl für Software, deren Quelltext frei einsehbar ist, als auch für Software, deren Quelltext nicht frei einsehbar ist.}{12}}
}

\newglossaryentry{AL2}
{
    name= Apache Lizenz 2,
    description={\z{Die Apache-Lizenz wird von der Free Software Foundation als Lizenz für freie Software anerkannt und ist zu der GNU General Public License Version 3, nicht jedoch zur Version 2, kompatibel.}{13}}
}

\newglossaryentry{Cloud}
{
    name= Cloud,
    description={Umgangssprachliche Bezeichnung für eine verteilte IT- bzw. \Gls{Computer}struktur}
}

\newglossaryentry{iFrame}
{
    name= iframe,
    description={HTML-Element welches den Inhalt einer externen Seite anzeigen kann, auch Inlineframe}
}

\newglossaryentry{Grunt}
{
    name= Grunt,
    description={Ein Javascript-Softwareframework zur automatisierten Job-Verwaltung. Die Software wrd genutzt um zum Beispiel nach jedem erfolgreichen Build, eine Reihe an Test absolvieren zu lassen}
}

\newglossaryentry{Test}
{
    name= Test,
    description={Ein Test wird im Idealfall vor der Implementierung einer Funktion geschrieben und soll gewährleisten, dass nach jeder Änderung im Programmcode diese Funktion ihren Zweck noch erfüllt. Eine Testsuite ist eine Ansammlung von Tests die nacheinander absolviert werden}
}

\newglossaryentry{parallel-synchron}
{
    name= parallel-synchron,
    description={Simulierte synchrone Bedienung mehrerer \Gls{moEn}e innerhalb des gleichen Zeitintervalls}
}

%%
%% ======================= GLOSSAR ENDE
%%


\usepackage{tikz}
\usetikzlibrary{calc}

%%
%% Pfad zu den Bildern
%%
\graphicspath{
  {pictures/}
}

%%
%% Titel, Autor und Betreuer
%%
\version{0.1$\alpha$}
\datum{\today}
\fachbereich{VI -- Informatik und Medien --} 
\studiengang{Medieninformatik}
\autor{Adrian Randhahn}
\edvnr{744818}
\titel{Evaluierung von Techniken zur parallel-synchronen Bedienung einer Web-Applikation auf verschiedenen mobilen Endgeräten} 
\abschluss{Bachelor of Science (B.Sc.)}

\betreuerFeld{
  \begin{tabular}{lr}
    \multicolumn{2}{l}{\textbf{Gutachter}}\\
    Prof.~Knabe & Beuth Hochschule für Technik\\
    Prof.~Dr. Wambach & Beuth Hochschule für Technik
  \end{tabular}
}

\begin{document}
\pagestyle{fancy}

%%
%% ############# Deckblatt
%%
\maketitle
\chapter* { }
\pagenumbering{roman}

%%
%% ############# Erklärung
%%
\chapter*{Erklärung}
Ich  versichere, dass  ich diese  Abschlussarbeit ohne  fremde  Hilfe selbstständig
verfasst und  nur die  angegebenen Quellen und  Hilfsmittel benutzt  habe. Wörtlich
oder dem  Sinn nach  aus anderen  Werken entnommene Stellen  sind unter  Angabe der
Quellen kenntlich gemacht.
Ich erkläre weiterhin, dass die vorliegende Arbeit noch nicht im Rahmen eines anderen Prüfungsverfahrens eingereicht wurde.
\vspace{10ex}\\
\hrule
{\small{Datum}}\hfill{\small{Unterschrift}}

%%
%% ############# Datenschutz
%%
\chapter*{Sperrvermerk}
Die vorliegende Arbeit beinhaltet interne und vertrauliche Informationen der Firma New Image Systems GmbH. Die Weitergabe des Inhalts der Arbeit im Gesamten oder in Teilen sowie das Anfertigen von Kopien oder Abschriften - auch in digitaler Form - sind grundsätzlich untersagt. Ausnahmen bedürfen der schriftlichen Genehmigung der Firma New Image Systems GmbH.

%%
%% ############# Abstract
%%
\section*{Kurzfassung}
Während der Entstehung einer Webapplikation durchläuft diese wiederholt die Qualitätskontrolle. Innerhalb dieser Arbeit werden bestehende Technologien auf deren Nutzen hin untersucht die Qualitätssicherung qualitativ zu verbessern. Die \Gls{Framework}s werden auf ihre Kompatibilität für Desktop-, sowie Mobilbrowser untersucht. Des Weiteren werden alleinstehende Frameworks, dahin gehend untersucht, ob sie in kombinierter Form in der Lage sind ein eigenständiges \Gls{Framework} zu erschaffen.


\section*{Abstract}
During the origin of a web application this runs through repeats the high-class control. Within this work existing technologies are examined for their use to improve the quality assurance qualitatively. Besides become single Frameworks, passing examined whether they in combined form in the situation are to be fulfilled the demanded criteria.

%%
%% ############# Inhaltsverzeichnis
%%
\tableofcontents

%%
%% ############# Abbildungsverzeichnis
%%
\listoffigures

%%
%% ############# Tabellenverzeichnis
%%
\listoftables

%%
%% ############# Maincontent
%%
\pagenumbering{arabic}

%%
%% ############# Einleitung
%%
\chapter{Einleitung}
%\tr{Eine Einleitung bietet die Möglichkeiten den Sinn und Zweck der Diplomarbeit für einen Durchschnittsinformatiker (ohne die Spezialkenntnisse, die Sie jetzt haben) verständlich zu beschreiben. Hier können Sie Hintergründe darstellen, wie die Arbeit und das Thema entstanden und selbstverständlich für Ihre Arbeit werben. Interessierte Leser entscheiden hier, ob diese Arbeit für sie fachlich interessant ist.}

%\tr{	Eine kurze Beschreibung des allgemeinen Forschungsgebietes in ein bis zwei Absätzen. Die Einleitung sollte am Ende in ein bis zwei Sätzen die eigentlich untersuchte Fragestellung benennen.}
%Thematische Hinführung (2 Absätze); diese besteht im Idealfall aus dem Einstieg (etwas, was an die allgemeine Erfahrung anknüpft und unmittelbar ersichtlich ist) und einem weiteren Absatz, in dem – ausgehend vom 		Einstieg – auf das eigentliche Thema fokussiert wird. In einer Arbeit über Online-Marketing mit Facebook beispielsweise würde es im 1. Absatz um Online-Marketing allgemein gehen und im 2. Absatz auf die besonderen 		Anforderungen im Zusammenhang mit Facebook verwiesen. (Kontrollfrage: „Worum geht es hier?“)


%\tr{Hier sollte der Hintergrund und die Motivation der Arbeit kurz angerissen werden. Nach Möglichkeit sollte man alle Aspekte, die im zweiten Kapitel ("Grundlagen") besprochen werden, hier schon einmal ansprechen, damit 		diese nicht später aus heiterem Himmel fallen. Insbesondere sollte der Hintergrund quasi "zwingend" den nächsten Teil der Einleitung motivieren:

%\section{Hintergrund}	
In der modernen Webentwicklung durchläuft eine Anwendung verschiedene Etappen eines Entwicklungszykluses. Er beginnt bei einem Auftrag oder einer Idee, darauf folgt dann die Spezifikation einzelner \mi{Usecases}\footnote{Szenario oder auch Anwendungsfall}. Im Anschluss folgt in der Regel die Entwicklung und Implementation\footnote{Einbindung} der einzelnen Komponenten. Am Ende der jeweiligen Implementationsphase durchläuft das Produkt\footnote{hier: einzelne Softwarekomponente} die Qualitätskontrolle. Sollten in diesem Abschnitt Fehler auftreten wird das Produkt dem Entwickler zur erneuten Bearbeitung vorgelegt. 
\\
Dieser Vorgang kann sich beliebig oft wiederholen. Bei großen und komplexen Softwaresystemen ist es trotz zeitgemäßer Implementierung nicht immer Ausgeschlossen, dass \mi{Kaskadierungsfehler}\footnote{Fehler die nicht im eigentlichen Segment auftreten, sondern eine oder mehr Ebenen weiter unten in der Systemhirarchie} entstehen. Aus Sicht der Qualitätssicherung ist dies ein lästiges Problem, da diese nach jedem erneuten Modifikationsvorganges eines Softwaresegments einen größeren Segmentblock, wenn nicht sogar das gesamte Softwareystem erneut testen muss.
\\
Bei der Entwicklung auf und für mobile Endgeräte\footnote{Smartphones, Tabletts  oder Ähnliche} kommt noch ein erschwerender Faktor hinzu, nämlich die diversen, verschiedenen Bildschirmauflösungen. Diese können nicht nur die Darstellung des Inhaltes beeinflussen, sondern auch daraus folgend die Interaktionskonformität beeinflussen.

\ig{../pictures/Entwicklungsprozess}{Entwicklungsprozess}{Vereinfachte Darstellung eines Softwareentwicklungsprozesses}

Im Optimalfall wird die Software erst nach vollständiger Homogenität auf allen unterstützen Geräten freigegeben.

%\section{Problemstellung}
\\
Dieser zyklisch wiederkehrende Prozessablauf ist sehr Zeitintensiv und nimmt linear mit der Anzahl der zu testenden Geräte zu.

%\section{Operationalisierung der Fragestellung}
\\
Das Ergebnis dieser Forschungsarbeit soll zeigen, wie verschiedene \mi{Softwareframeworks} die Zeit, die in die Qualitätssicherung investiert wird, beeinflussen können, indem sie die Steuerung diverser Geräte parallel-synchron steuern. Die Evaluierung soll zeigen wo die Vorteile und Nachteile der einzelnen Werkzeuge liegen. Weiterhin soll gezeigt werden ob aktuelle \mi{Frameworks} erweiterbar sind um Beispielsweise automatisierte \mi{Testunits} zu implementieren.
%\section{Untersuchungsverlauf}

\subsubsection{Anmerkung}
Aus Gründen der besseren Lesbarkeit wird für alle Personen und Funktionsbezeichnungen durchgängig das generische Maskulinum angewendet und bezieht in gleicher Weise Frauen und Männer ein.

%%
%% ############# Aufgabenstellung
%%
\chapter{Aufgabenstellung}
%\tr{Durch eine klare Beschreibung der Aufgabenstellung wird die zu lösende Aufgabe deutlich. Vorhandene Teillösungen oder -systeme können hier ebenfalls dargestellt werden. In vielen Fällen ist es auch hilfreich Sachverhalte oder Problemstellungen zu beschreiben, die nicht zur Aufgabenstellung gehören (Abgrenzung).}
Die Aufgaben dieser Thesis ist die Evaluierung von Techniken zur parallel-synchronen Steuerung von Webapplikationen auf mobilen Endgeräten, um damit die Produktivität der Qualitätssicherung zu optimieren.
	%%
	%% ############# Annahmen und Einschränkungen
	%%
	\section{Problemstellung}
	%\tr{Aus dem Hintergrund sollte die Wissenslücke klar werden, die durch die Abschlussarbeit geschlossen werden soll. Kurz 		sollte beschrieben werden, mit welchen Methoden die Arbeit versuchen will, diese zu schließen 			(empirische 		Untersuchung, Loganalyse, neuartige Programmkomponenten, etc.). Schließlich sollte man herausstellen, warum es wichtig ist, 	diese Wissenslücke zu schließen (wie profitiert die Welt davon).}
	Ein Problem in der aktuellen Softwareentwicklung ist die immer mehr wachsende Anzahl an Endgeräten, welche mit 			verschiedenen Bildschirmauflösungen und eigenen Betriebsystemen in unterschiedlichen Versionen auftreten. Ein 				Qualitätsprüfer der einen hohen Qualitätsstandard hat investiert daher linear zu der Anzahl der zu testenden Geräte ansteigend 	Zeit, lediglich um vereinzelte Testszenarien durchzuarbeiten. Solch ein Testszenraio kann Navigationsabläufe\footnote{ein 		Nutzerspezifischer Gang durch die Webseite}, das ausfüllen und validieren eines Formular oder auch das überprüfen 			funktionaler\footnote{aktive Links und deren Aufruf} Links sein. Bereits an dieser Stelle ist die zu investierende Zeit, und dies 		wiederholt, enorm.
	\\
	Wenn der Qualitätsprüfer innerhalb eines Testszenarios einen schwerwiegenden Fehler bei einem der Geräte entdeckt, muss 		dieser den Vorgang beenden. Abgebrochen werden muss deshalb, da bei korrigierter Implementierung der Qualitätsprüfer nicht 	davon ausgehen darf, das bereits kontrollierte Abschnitte immernoch voll funktionsfähig sind, da eventuell neue Fehler in bereits 	Kontrollierten Segmenten auftreten können.
	\\
	Sollte ein Szenario aufgrund eines Fehler abgebrochen worden sein, wird dem Entwickler das Problem möglichst konkret 		geschildert. Dessen Aufgabe ist es nun das Problem zu beheben. Ist dies geschehen startet der Prüfer einen erneuten 			Durchgang des Szenarios. Ein generelles Problem was hier noch zusätzlich entstehen kann, ist der Umstand, dass sich grade 		bei nur kleineren fixes\footnote{Problemlösungen, Codeanpassungen} und immer wieder auftretenden Testszenarioschleifen 		eine gewisse Routine einschleichen kann, worunter die Qualität des Produkts leidet.

	\ig{../pictures/Testszenario}{Qualitätssicherung Testszenario}{Darstellung eines Qualitätssicherungsablaufes in der mobilen 		Anwendungsentwicklung}}
	\pagebreak
	
	%%
	%% ############# Annahmen und Einschränkungen
	%%
	
	%\section{Annahmen und Einschränkungen}
	%\tr{Wenn die Arbeit wichtige Annahmen trifft, unter denen die Untersuchungsergebnisse gültig sind, oder die Allgemeinheit der 	getroffenen Aussagen wichtigen Einschränkungen unterliegt, sollten diese ebenfalls in der 		Einleitung 	beschrieben 		werden.}
	
	%%
	%% ############# Zielsetzung
	%%
	\section{Zielsetzung}
	 Das Ziel dieser Arbeit ist es, bestehende \mi{Frameworks} auf ihre Tauglichkeit in Bezug auf die parallel-synchrone Steuerung 	von mobilen Endgeräten zur Durchführung von Testszenarien zu evaluieren.
	Hierzu werden auf mobilen Endgeräten die internen Browser getestet. Hinzu kommen auf Desktopgeräten die aktuellen 			Versionen von Firefox, Chrome, Safari(nur für Mac-Desktopgeräte) und der Internet Explorer(nur für Windows-Desktops). Um 		eine Allgemeine Testbarkeit zu gewährleisten werden die Frameworks auch auf Genauigkeit in virtuellen Umgebungen analysiert. 	Dabei können Abweichungen, seien sie noch so klein, entstehen. Bereits 1 Pixel Abweichung kann bereits ausschlaggebend 		sein einen Umbruch zu erzeugen und damit das Layout negativ zu verändern.
	%%
	%% ############# Abgrenzungskriterien
	%%
	\section{Abgrenzungskriterien}
	%\tr{Hier werden die Grundlagen für das zu entwickelnde Softwaresystem definiert. Zwar noch aus fachtechnischer Sicht werden 	hier die Anforderungen an das geplante Softwaresystem in möglichst formaler Form spezifiziert. 	Es 	sollen hier keine 		Lösungen präsentiert werden, sondern möglichst präzise die Anforderungen (Sollkonzept) an das geplante Softwaresystem mit 	seinen Schnittstellen, Informationsflüssen und Systemfunktionen 			dokumentiert 	werden. Verwendete Methoden 	können z.B. SA, SADT, Petri-Netze oder andere sein. Das Ergebnis ist ein für die Systementwicklung verwendbares 			Pflichtenheft. Über Art und Umfang des Pflichtenhefts sollten 	Sie mit Ihrem Betreuer sprechen.}
	\subsubsection{Zeit}
	Als eins der wichtigsten Abgrenzungskriterien gilt es die Einarbeitungszeit zu bewerten. Hier gilt je kürzer desto besser, immer gesehen in Relation 	zu dem Umfang des Frameworks. So ist ein Framework qualitativer zu bewerten, wenn es exponentielle Lernkurve in Relation zur Zeit vorweist.

\begin{figure}[H]
	\centering
	\begin{tikzpicture}
		
	\draw[thick,->] (0,0) -- (10,0) node[right]{$Zeit$};
	\draw[thick,->] (0,0) -- (0,10) node[above]{$Lernfortschritt$};

	\draw[red,id=test1,samples=100,domain=0.0:9.0] plot(\x,{1.3*ln(\x+1)}); 
	\draw[blue,domain=0:3.3] plot (\x,{-1+exp(ln(2)*\x)});
	\draw[red,thick] (9,5) -- +(0.3,0) node[anchor=mid west,black] {schlechte Lernkurve};
	\draw[blue,thick] (9,5.5) -- +(0.3,0) node[anchor=mid west,black] {gute Lernkurve};

	\end{tikzpicture}
	\vspace{-25pt}
	\caption[Darstellung der Lernkurve für Frameworks]{Lernkurve für Frameworks}
\end{figure}
\vspace{-40pt}


	\pagebreak
	 \subsubsection{Erweiterbarkeit}
	 in Hinsicht auf mehrere Endgeräte gibt es mehrere Aspekte zum analysieren. Zum einen ob das Framework auf verschiedene 	Arten an System gebunden sind, wie etwa Android oder iOS. Zum anderen ob es eine limitierung der Anzahl der 				anzuschliessenden Geräte gibt. 

	\\Unter den Aspekt der Erweiterbarkeit fällt auch die Möglichkeit das Framework um eigene Funktionalitäten zu erweitern. So 		sollte im Idealfall ein Framework ein Grundgerüst liefern, auf das der Entwickler mit eigenen Erweiterungen aufbauen kann um 	das gewünscht Ziel zu erreichen.

	\subsubsection{Unterstütze Browser}
	Ein wichtiger Aspekt der durchzuführenden Tests wird die unterstützung möglichst verschiedener Browser beinhalten. Dies hat 	den Grund, dass leidglich ein Framework was eine große Spanne an Endgeräten abdeckt in der Lage ist effektiv genutzt werden 	zu können.

	\subsubsection{Virtuelle Umgebung}
	Ein positiv in die Validierung einfliessender Aspekt ist die Einbindung oder Verwendung des Frameworks innerhalb einer 			virtuellen Umgebung. Das wird durch den Fakt begründet das der Tester nicht immer im Besitz aller Testgeräte oder 				Umgebungen ist. So ist es ohne eine virtuelle Maschiene zum Beispiel nicht möglich eien Seite im Internet Explorer innerhalb 		einer MacOS-Umgebung zu testen, da diese ihn nicht unterstützt.



%%
%% ############# Grundlagen
%%
\chapter{Grundlagen}
In diesem Abschnitt behandle ich spezifische Definitionen wie zum Beispiel verwendetes Fachvokabular, allgemeine technische Abläufe die Notwendig sind um diese Arbeit und die darin verwendetet Techniken zu verstehen, sowie verwendete Hardwarekomponenten.
%\tr{Dieser Teil beschreibt das fachliche Umfeld der Aufgabenstellung. Hier werden die wesentlichen fachlichen Begrifflichkeiten, die für die Aufgabe relevanten Problemstellungen und Lösungsansätze des Fachgebietes vorgestellt. Der Sprachgebrauch sollte einen direkten Bezug zum Fachgebiet haben. Die notwendigen Darstellungsmethoden, die Art und der Umfang der Beschreibung hängen wesentlich von der jeweiligen Fachdisziplin ab und sollten im Dialog mit dem Betreuer entschieden werden. Beispielsweise wird sich die Beschreibung eines Hotelreservierungssystems sehr von einer Beschreibung mathematischen Transformationen auf Grafikobjekte unterscheiden.Dies ist oft vor der Einleitung das erste Kapitel, das man schreibt, und sollte einen Überblick über die Literatur und existierende Arbeiten im Bereich der Arbeit liefern (Welche Grundlagen gibt es in diesem Bereich? Haben andere Autoren schon etwas zu verwandten Themen veröffentlicht?). Die hier vorgestellten Konzepte sollten in der Einleitung zumindest schon einmal angesprochen worden sein. Bei der Vorstellung verwandter Arbeiten sollten neuere Erkenntnisse bevorzugt werden. Bei jeder in das Grundlagenkapitel aufgenommenen Arbeit gilt es herauszustellen, was die Ergebnisse der Arbeit waren und warum diese Ergebnisse (oder Einschränkungen der vorgestellten Arbeit) eben noch keine Schließung der Wissenslücke oder keine Lösung der Aufgabenstellung darstellen? Am Ende erfolgt eine kurze Zusammenfassung der Grundlagen, die begründete Schlussfolgerung, dass das zu untersuchende Problem noch ungelöst ist, und ggf. wieder eine Vorschau auf das folgende Kapitel.}

	%%
	%% ############# Begriffsklärung
	%%
	\section{Begriffsklärung}	
		\subsection{parallel-synchron}
		\subsection{Web-Applikation}
		\subsection{HTML}
		Die Hypertext Markup Language ist eine Auszeichnungsprache zur Beschreibung von Inhalten. Sie dient der Strukturierung 		von Texten, Links\footnote{Verweise zu anderen Inhalten}, Listen und Bildern eines Dokumentes. Eine HTML Seite wird von 		einem Webbrowser interpretiert und anschließend dargestellt. Die Entwicklung von HTML geschieht durch das World Wide 		Web Consortium(W3C) und den Web Hypertext Application Technology Working Group (WHATWG). 

		\subsection{Webbrowser}
		\subsection{Desktopcomputer / Desktops}
		In dieser Arbeit werden gängige Modelle von Personal Computern oder Macs mit einem festen Arbeitsumfeld als Desktops 		bezeichnet. Hierzu zählen auch tragbare Modelle und Laptops. Im Sinne der Thesis umschließe ich nachfolgend mit dem 		Begriff Desktop oben genannte Komponenten. Dies dient später der Differenzierung ob es sich um ein mobiles Endgerät 			handelt oder einem Computer .

		\subsection{Mobiles Endgerät}
		Im Nachfolgenden werden Komponenten mit primärer mobiler Nutzung umfassend als mobile Endgeräte gruppiert. Hierzu 		zählen Smarthphones und Tablets.

		\subsection{Javascript}
		\subsection{Framework}
		\subsection{Nodejs}
		\subsection{PHP}
		\subsection{NPM}
		\subsection{Qualitätssicherung}
		\subsection{VirtualBox / virtuelle Umgebung}
		\subsection{Smartphone}
		\subsection{Tablet}
		\subsection{Panorama / Portrait View}
		\subsection{Pixel}
		\subsection{Auflösung}
		\subsection{Event}
		\subsection{DOM}
		\subsection{Apache}
		\subsection{Form, Checkbox, Radiobox, Inputs}
		\subsection{Workspace}
	
	%%
	%% ############# technischer Aufbau
	%%
	\section{technischer Aufbau}
	
	%%
	%% ############# Komponenten
	%%
	\section{Komponenten}
		\subsection{\mi{Raspberry Pi}}
		\subsection{Hardware}


%%
%% ############# Technologien
%%
\chapter{Auswahl der Frameworks}

\section{Frameworks als Komplettlösung}
Unter diesem Punkt werden alle Frameworks gelistet, welche damit werben das parallel-synchrone Testen von mobilen Seiten zu ermöglichen und somit den Entwicklungsprozess zu optimieren. Im Gegensatz zu den Einzelkomponenten sollten diese im Idealfall keine weiteren Technologien benötigen um genutzt zu werden.
	\subsection{\mi{Ghostlab}}
	Ghostlab ist ein Framework des Schweizer Unternehmens Vanamco. Es verspricht das synchrone Testen von Webseiten in 		Echtzeit. Weiterhin wirbt das Unternehmen mit einem umfangreichen Repertoire an nützlichen Fähigkeiten. Der 				Funktionsumfang umschließt das Scrollen innerhalb einer Seite, das ausfüllen von  Formularen, das wahrnehmen und 			reproduzieren von Click-Events sowie dem neu-laden einer Seite. Ghostlab soll ebenso einen Inspektor besitzen, welcher die 		Analyse des DOMs, der on the fly Bearbeitung der CSS und der Analyse und Bearbeitung von Javascriptdateien. Das 			Framework gibt an für alle folgenden Browser zu funktionieren ohne diese Konfigurieren zu müssen:

	\begin{table}[H]
 		\centering
		\rowcolors{1}{white}{lightgray}
			\begin{tabular}{| p{5cm} | p{5cm} |}
			
			\hline
				Browser 	& 	Version\\
			\hline
			\hline
				Firefox	&	latest\\
				Chrome	&	latest\\
				Safari	&	latest\\
				Internet Explorer	&	8/9/10\\
				Opera	&	11\\
				Opera Mobile	&	supportet\\
				FireFox Mobile	&	supportet\\
				Blackberry	&	supportet\\
				Windows Phone	&	supportet\\
				Safari mobile	&	supportet\\	
				Android	&	2.3 - 4.2\\
				\hline
				\end{tabular}
			\caption{von Ghostlab getestete Browser (Stand 10.03.2014, Version 1.2.3)}
	\end{table}

	Der Kostenpunkt der Lizenz liegt zur Erstellung dieser Arbeit bei 49\$ (entspricht 35,30€ beim aktuellen Umrechnungswert). Zur 	Erstellung dieser Thesis wurde die 7-Tage-Testvollversion genutzt.
	
	\subsection{\mi{Adobe Edge Inspect}}
	Die Anwendung Edge Inspect stammt von Adobe und wird derzeit in der CC\footnote{Creative Cloud} Version vertrieben. Um 		Adobe Edge Inspect nutzen zu können bedarf es 3 separaten Komponenten. Adobe wirbt mit synchronem aufrufen und 			auffrischen von Websites, sowie deren Inspizierung per Weinre. Besonders angepriesen wird von Adobe die Nutzung und 		Verwendung der Adobe Edge Inspect API, welche auf GitHub zur Verfügung gestellt wird. Des weiteren kann Adobe Edge 		Inspect in andere Edge Produkte\footnote{zum Beispiel Edge Reflow CC und Edge Code CC} integriert werden. 
	
	\\Adobe Edge Inspect CC steht 30 Tage kostenlos zum testen bereit. Danach fallen ab 24,59/ Monat für die Nutzung des 			Einzelprodukt-Abos an.
	
	\\Die Anwendung läuft nur auf mobilen Endgeräten mit iOS oder Android Betriebssystem.
	
	\subsection{\mi{Remote Preview}}
	Remote Preview ist ein kleines Javascript Framework von dem Web Designer Viljami Salminen aus Helsinki, Finnland. Es 		überprüft alle 1100ms per AJAX-Request ob sich die Quellurl geändert hat und teilt dies dann den verbundenen Testgeräten mit. 	Er wirbt mit dem synchronen Aufruf von Webseiten auf einer Vielzahl von Plattformen: 
	
	\begin{table}[H]
 		\centering
		\rowcolors{1}{white}{lightgray}
			\begin{tabular}{| p{13cm} |}
			
			\hline
				Plattform\\
			\hline
			\hline
				Android OS 2.1 - 4.1.2 (Default browser + Chrome)\\
				Blackberry OS 7.0 (Default browser)\\
				iOS 4.2.1 - 6 (Default browser)\\
				Mac OS X (Safari, Chrome, Firefox, Opera)\\
				Maemo 5.0 (Default browser)\\
				Meego 1.2 (Default browser)\\
				Symbian 3 (Default browser)\\
				Symbian Belle (Default browser)\
				WebOS 3.0.5 (Default browser)\\
				Windows Phone 7.5 (Default browser)\\	
				Windows 7 (IE9)\\
				\hline
				\end{tabular}
			\caption{von Remote Preview unterstützte Plattformen (stand 19.03.2014, letzter Commit 7dc48caa84)}
	\end{table}
	Das Framework ist Kostenlos erhältlich und läuft unter der MIT Lizenz. Zum Zeitpunkt dieser Arbeit scheint das Projekt nicht 		weiter entwickelt zu werden, da seit 5 Monaten auf der Projektseite keinerlei Aktualisierungen vorgenommen wurden.

		
	\subsection{\mi{Browser-Sync}}
	Browser-Sync wurde von Shane Osbourne entwickelt und soll im Zuge dieser Arbeit den Ansprüchen des Themas gerecht 		werden. Es wirbt mit synchronisierter Steuerung, dem Entwickeln an Stiles und anderen Projektdateien in Echtzeit, der 			Installation unter Windows, MacOS und Linux und einer umfangreichen Palette an unterstützen Plattformen. Jedoch unterstützt 	das Framework im Gegensatz zu Ghostlab oder Adobe Edge Inspect keine Remoteinspection des DOM und den 				Netzwerkaktivitäten.
	
		\begin{table}[H]
 		\centering
		\rowcolors{1}{white}{lightgray}
			\begin{tabular}{| p{5cm} | p{5cm} |}
			
			\hline
				Browser 	& 	Version\\
			\hline
			\hline
				Firefox	&	latest\\
				Chrome	&	latest\\
				Safari	&	latest\\
				Internet Explorer	&	7/8/9/10\\
				Opera	&	latest\\
				Opera Mobile	&	supportet\\
				FireFox Mobile	&	supportet\\
				Blackberry	&	supportet\\
				Windows Phone	&	supportet\\
				Safari mobile	&	supportet\\	
				Android	&	supportet\\
				iOS		&	supportet\\
				\hline
				\end{tabular}
			\caption{von Browser-Sync getestete Browser (Stand 21.03.2014, Version 0.7.2)}
	\end{table}
	
	Das Framework basiert auf dem NodeJS Framework und besitzt dadurch ein hohes Erweiterungspotential. Eine parallel zu 		Browser-Sync entwickelte Erweiterung kombiniert Browser-Sync mit Grunt, was automatisierte Abläufe ermöglicht. Diese fördert 	die Produktivität durch das einbinden des Frameworks in bestehende Arbeitsabläufe. Die Software ist kostenlos erhältlich und 		steht unter der MIT Lizenz. Das Projekt befindend sich zum Zeitpunkt dieser Arbeit in der Version 0.7.2 und wird täglich 			weiterentwickelt.

\pagebreak
\section{Frameworks als Einzelkomponente}
	Als Einzelkomponenten werden hier Frameworks spezifiziert, welche dazu beitragen, das in der Thesis geforderte Werkzeug 		selbst zu entwickeln. Diese decken verschiedene Einzelkomponenten ab, wie zum Beispiel die Clientverwaltung, Steuerbefehle 	oder setzen die Voraussetzung für eigene Testszenarien.
	
	\subsection{\mi{NodeJS}}
	Node.JS Aufgabe besteht darin, anstelle von zum Beispiel Apache, einen Webserver zur Verfügung zu stellen, welcher nur auf 		Javascript basiert. Alle Notwendigen serverseitigen Anfragen und Funktionen erfolgen in Javascript. Entwickelt wird Node.JS von 	der Kalifornischen Firma Joyent und befindet sich derzeit in Version 0.10.26. Geführt wird Node.JS unter der MIT Lizenz und 		steht kostenlos auf nodejs.org oder unter GitHub zum Download bereit.
	
	\subsection{NPM \mi{socket.io}}
	socket.io ist ein Framework welches die WebSocket Technologie aktueller Browser auf Javascript Ebene abbildet. Der Gedanke 	der Technologie dahinter verfolgt den Gedanken nicht in regelmäßigen Abständen Anfragen an den Server zu stellen und damit 	unnötig viel Datenvolumen zu generieren, sondern eine permanente Verbindung zum Server aufrecht zu halten um auf 			Statusänderungen am Server zu reagieren. socket.io wurde von Guillermo Rauch unter der MIT Lizenz entwickelt und steht 		derzeit in der Version 0.9.16 auf GitHub oder per NPM zur Verfügung.
	
	\subsection{\mi{Zombie.js}}
	Das Framework Zombie.js ist ein Open-Source Projekt einer ganzen Gruppe von Entwicklern\footnote{https://github.com/assaf/	zombie/graphs/contributors}, welches von dem in Kalifornien sitzenden Assaf Arkin ins Leben gerufen wurde. Zombie.js wirbt mit 	seiner Einfachheit Tests zu erstellen und in Testsuiten zu integrieren. Zombie.js emuliert einen sogenannten headless			\footnote{Kopflos - ohne Gerüst das ihn umschließt, oder auch virtuell} Browser. Dies hat zur Folge das natürlich nur non-		visuelle Aspekte in Tests integriert werden können, wie etwa das ausfüllen von Formularen, das navigieren durch den 			Navigationsbaum oder das testen von Links.

	\subsection{\mi{Phantom Limb}}
	Phantom Limb ist ein von Brian Carstensen entwickeltes Werkzeug welches es ermöglichen soll, die Computermaus generierten 	Bewegungen in äquivalente Touchevents umwandelt. Das Framework läuft unter der Apache Lizenz 2 und kann kostenlos 		verwendet werden. Es kam in die Auswahl der Frameworks, da es von Nutzen ist das Steuergerät für die Tests an einem 			Computer zu verwenden.
	
	\subsection{\mi{jQuery UI Touch Punch}}
	jQuery UI Touch Punch ist eine Erweiterung zu der UI Bibliothek von jQuery die David Furfero entwickelt hat. Diese erlaubt von 	Hause aus nicht die Nutzung von Touch Events auf mobilen Endgeräten. Die Erweiterung hebt diese Restriktion auf ohne weiter 	Konfiguriert werden zu müssen. An Quellcode kommen lediglich weitere 584 Bytes hinzu. Die Frameworkerweiterung läuft unter 	der MIT und der GPL Lizenz, wodurch es dem Endnutzer frei steht die Bibliothek unter den Lizenzen des eigenen Projektes zu 		verwenden.
	
	\subsection{\mi{jQuery Touchit}}
	Das von Daniel Glyde entwickelte Framework jQuery Touchit wandelt Berührungen in äquivalente Mausevents um und ermittelt 	deren relative Position in Bezug auf den Viewport. Des weitere löst es das Problem bei bereits bestehenden jQuery 				Anwendungen und deren Darstellung auf mobilen Endgeräten wo verschiedene Funktionalitäten , wie zum Beispiel die 			Verwendung von Slidern, nicht Nutzbar sind.
	
	
	
	

%%
%% ############# Evaluation
%%	
%%
%% ############# Evaluation
%%
\chapter{Evaluation der \Gls{Framework}s}
\section{Auflistung des Evaluationsschlüssels}

Zur Evaluierung der einzelnen \Gls{Framework}s, wurde ein Schlüssel erstellt, welcher messbare Aspekte abdeckt, die im Vorfeld der Arbeit bereits als wichtig erachtet wurden, um die aufgestellte Thesis in Zahlen darzustellen. Ergänzend kamen Punkte hinzu, die bei der Installation und der Nutzung auffielen und sich als wichtig erwiesen. 

\\Der Schlüssel wurde in sieben Hauptkategorien aufgeteilt. Die Abschnitte 'Installation' und 'Konfiguration' befassen sich in erster Linie mit der Verfügbarkeit, dem Zugang zu dem \Gls{Framework}, dessen Installation und Dokumentation sowie der Voraussetzung anderer Technologien, um es zu nutzen.

\subsubsection{Installation}
\begin{table}[H]
 	\vspace{-30pt}
 	\centering
	\rowcolors{1}{white}{lightgray}
		\begin{tabular}{| p{12cm} | c|}
			\hline
				Kriterium		 &	Punktezahl\\
			\hline
			\hline
				Notwendigkeit von anderen Technologien				&4\\
				Nutzbar direkt nach der Installation			&	2	\\
				Installationsanleitung vorhanden			&	2	\\
				FAQ vorhanden				&	2	\\
				\hline
		\end{tabular}
	\caption{Kriterienübersicht: Installation}
\end{table}

\subsubsection{Konfiguration}
\begin{table}[H]
 	\vspace{-30pt}
 	\centering
	\rowcolors{1}{white}{lightgray}
		\begin{tabular}{| p{12cm} | c|}
			\hline
				Kriterium		 &	Punktezahl\\
			\hline
			\hline
				Nutzbar ohne Konfiguration			&4\\
				Konfigurierbarkeit (IP und Ports)			&	1	\\
				Konfigurierbare \Gls{Workspace}s			&	2	\\
				Support vorhanden (Wiki, Helpdesk, EMail, Forum)				&	2	\\
				Intuitive Benutzeroberfläche			&	1	\\
				\hline
		\end{tabular}
	\caption{Kriterienübersicht: Konfiguration}
\end{table}

\\Der Teilschlüssel 'Funktion' wurde dupliziert und in einen Mobilteil und einen Desktopteil aufgeteilt, da das Testen von mobilen Seiten \mbox{andere} Schwerpunkte der Bewertung haben sollte, als das von \mbox{Desktopseiten}. So ist eine funktionierende synchrone Gestenkontrolle beim Verwenden von mobilen Seiten zum Beispiel unerlässlich, wohingegen sie auf Desktops durch den Einsatz einer Maus tendenziell eher unhandlich in der Nutzung ist.
\subsubsection{Funktion: Desktop}
\begin{table}[H]
 	\vspace{-30pt}
 	\centering
	\rowcolors{1}{white}{lightgray}
		\begin{tabular}{| p{12cm} | c|}
			\hline
				Kriterium		 &	Punktezahl\\
			\hline
			\hline
				Darstellung : normale Seiten			&2\\
				Darstellung : \gls{gesichert}e Seiten		&	2	\\
				 Darstellung: normale Reaktionsgeschwindigkeit ( < 1 Sekunde)	&	2	\\
				Funktion: Seitensteuerung			&	3	\\
				Funktion: \Gls{Javascript}			&	1	\\
				\hline
		\end{tabular}
	\caption{Kriterienübersicht: Desktop}
\end{table}

\subsubsection{Funktion: Mobil}
\begin{table}[H]
 	\vspace{-30pt}
 	\centering
	\rowcolors{1}{white}{lightgray}
		\begin{tabular}{| p{12cm} | c|}
			\hline
				Kriterium		 &	Punktezahl\\
			\hline
			\hline
				Darstellung : normale Seiten			&1\\
				Darstellung : \gls{gesichert}e Seiten		&	1	\\
				 Darstellung: normale Reaktionsgeschwindigkeit ( < 1 Sekunde)	&	2	\\
				Funktion: Seitensteuerung			&	4	\\
				Funktion: \Gls{Javascript}			&	1	\\
				Funktion: Gestenkontrolle			&	1	\\
				\hline
		\end{tabular}
	\caption{Kriterienübersicht: Mobil}
\end{table}

\\Da es in der Praxis kein \Gls{Framework} gab, welches zum Zeitpunkt dieser Arbeit in der Lage war alle gewünschten Aspekte abzudecken, war es wichtig das Werkzeug um eigene Funktionen oder externe \Gls{Framework}s zu ergänzen. Dies wurde unter Berücksichtigung der API, der Lizenz des \Gls{Framework}s und dessen Dokumentation bewertet.
\subsubsection{Erweiterbarkeit}
\begin{table}[H]
 	\vspace{-30pt}
 	\centering
	\rowcolors{1}{white}{lightgray}
		\begin{tabular}{| p{12cm} | c|}
			\hline
				Kriterium		 &	Punktezahl\\
			\hline
			\hline
				API Zugang			&5\\
				API lizenzteschnich gesichert	&	2	\\
				API Dokumentation	&	3	\\
				\hline
		\end{tabular}
	\caption{Kriterienübersicht: Erweiterbarkeit}
\end{table}

\\Ein weiterer wichtiger Aspekt, ist die Unterstützung möglichst vieler verschiedener \Gls{Webbrowser} auf dem Desktop, auf dem Mobilgerät und in der virtuellen Umgebung.

\subsubsection{Browser Support (aktuelle Versionen)} 
\begin{table}[H]
 	\vspace{-30pt}
 	\centering
	\rowcolors{1}{white}{lightgray}
		\begin{tabular}{| p{12cm} | c|}
			\hline
				Kriterium		 &	Punktezahl\\
			\hline
			\hline
				mobile Plattformen (iOS, Android, Windows)			&3\\
				Virtuelle \Gls{Webbrowser}	&	2	\\
				Chrome				&	1	\\
				Opera				&	1	\\
				Firefox				&	1	\\
				Safari				&	1	\\
				Internet Explorer		&	1	\\
				\hline
		\end{tabular}
	\caption{Kriterienübersicht: \Gls{Webbrowser}}
\end{table}

\\Die Aktivität einer Software lässt darauf schließen, ob und gegebenenfalls wie, diese sich in Zukunft entwickeln kann. So sind von einem inaktiven Entwicklungsstand, von über einem halben Jahr, keine neuen Ergebnisse mehr zu erwarten und man muss davon ausgehen, dass die Software um keine neuen Features erweitert werden wird.
\subsubsection{Aktivität}
\begin{table}[H]
 	\vspace{-30pt}
 	\centering
	\rowcolors{1}{white}{lightgray}
		\begin{tabular}{| p{12cm} | c|}
			\hline
				Kriterium		 &	Punktezahl\\
			\hline
			\hline
				Noch in der Entwicklung (letztes Release, \Gls{Commit} Häufigkeit)			&5\\
				Aktives Forum	&	5	\\
				\hline
		\end{tabular}
	\caption{Kriterienübersicht: Aktivität}
\end{table}



%%
%% ############# Ghostlab
%%
	\pagebreak
	\section{\mi{Ghostlab} Version 1.2.3}
		\subsection {Einrichtung der Testumgebung}
		Ghostlab kommt von Hause aus mit einer 7-Tage-Testversion. Die Installation verlief einfach und ereignislos. Nachdem das Tool installiert wurde, erfolgte die Zuweisung einer Website zu dem Ghostlabserver. Es wurden in diesem Fall sowohl eine Seite auf einem lokalen \Gls{Apache} Server getestet, als auch die mitgelieferte Demoseite von Ghostlab. Nach dem Start des Ghostlabservers ist dieser über den localhost\footnote{IP-Adresse des lokalen Rechners} auf Port 8005 (Default) von allen zu testenden Geräten erreichbar.
		\ig{../pictures/ghostlab/startbildschirm}{Startbildschirm Ghostlab}{Startbildschirm von Ghostlab nach der Installation}
		
		\subsection{Testen von Desktopbrowsern}
		Durch Aufrufen der IP-Adresse des Rechners auf dem der Ghostlabserver läuft, verbindet sich der \Gls{Webbrowser} als Client und wird fortan durch gesendete Signale beeinflusst. Hierzu zählen auch virtuelle \Gls{Webbrowser}. Jeder Client wird nun gleichzeitig Sender und Empfänger für Signale, das bedeutet, dass jede Aktion \gls{parallel-synchron} auf allen anderen Clients gespiegelt wird. Hierzu zählen \Gls{Javascript}events, das Ausfüllen eines Formulars oder das Neuladen der gesamten Seite.
		\ig{../pictures/ghostlab/workspaces}{Übersicht Clients}{Darstellung von 4 verschiedenen Clients } 
		
		Über den Übersichtsbildschirm kann jeder verbundene Client einzeln inspiziert werden. Hier ist der Nutzer in der Lage sich durch das \Gls{DOM} zu navigieren oder temporäre CSS Anpassungen vorzunehmen. Die Handhabung ist intuitiv, was jedoch an dem verwendeten \Gls{Framework} \mi{weinre} liegt.
		\ig{../pictures/ghostlab/weinre}{Exemplarisch Weinreansicht}{ausgewähltes \Gls{DOM}-Element in weinre}
		
		\pagebreak
		\subsection{Testen von Mobilbrowsern}
		
		Das Einrichten zum Testen auf mobilen \Gls{moEn}en verläuft synchron zu den Desktopbrowsern. Man ruft innerhalb des \Gls{Webbrowser}s die IP-Adresse des Ghostlabrechners auf und ist schon nach wenigen Sekunden\footnote{abhängig von der Geschwindigkeit des Testgerätes} in die Clientliste aufgenommen.
		
		\\Bei dem Testen auf mobilen \Gls{Webbrowser}n ist es bei Ghostlab\footnote{Version 1.2.3} notwendig ausreichend Zeit zwischen den Eingaben zu lassen, da es sonst bei unterschiedlich schnellen Geräten zu einem Effekt kommt, bei dem die langsameren Geräte beim Ausführen des letzten Signals gleichzeitig wieder zum Sender für alle anderen Geräte werden.
		\ig{../pictures/ghostlab/uebersicht_mobil}{Übersicht mobile Clients Ghostlab}{Ghostlabübersicht der verbundenen Clients}
		
		\pagebreak
		
		\subsection{Fazit zu Ghostlab}
		Für diese Arbeit wurde Version 1.2.3 von Ghostlab genutzt. Zu diesem Zeitpunkt verfügte die Software noch über keinen Master/Slave-Modus\footnote{ein Gerät dient als Steuergerät, alle anderen folgen ihm}, dadurch kam es bei meinen Testgeräten bereits nach wenigen Minuten zu dem Problem, dass die Geräte sich in einer Endlosschleife von Senden und Empfangen der Steuerbefehle befanden. Für kommende Versionen ist ein solcher Modus laut den Entwicklern aber geplant. Das Problem rührt daher, dass einige Geräte schneller auf die übermittelten Befehle reagieren, als andere. Das führt dazu, dass die langsam ladenden Geräte in dem Augenblick wo sie das Signal umsetzen, für die schnelleren Geräte bereits wieder als Sender fungieren. Dieses Problem tritt bereits bei einer kleinen Anzahl von Geräten auf und wird deshalb als kritisch eingestuft. 

		\\Das Testen in mehreren \Gls{Webbrowser}n auf einem Rechner lief hingegen sehr gut. Das Ausführen von \Gls{Javascript} läuft einwandfrei. Das Ausfüllen von \Gls{Input}s, \Gls{Checkbox}en, \Gls{Radiobox}en und das Absenden des Formulars funktionierte bis auf die Kalenderauswahl im Firefox \Gls{Webbrowser}s anstandslos. Ein Problem scheint das Werkzeug mit passwortgeschützten Seiten zu haben. Diese lassen sich erst nach mehrfacher, abhängig vom jeweiligen \Gls{Webbrowser}, Eingabe des Passwortes aufrufen. Diese Prozedur wiederholt sich für jede weitere Unterseite. 

		\\Das Arbeiten in einer virtuellen Umgebung\footnote{es wurde \Gls{VirtualBox} von Oracle genutzt} wird problemlos unterstützt. Das einzige Problem, was ich analysieren konnte, war dass sich virtuelle \Gls{Webbrowser} nicht in einen \Gls{Workspace} integrieren lassen.

		\\Ghostlab unterstützt die Funktion von \Gls{Workspace}s\footnote{Arbeitsumgebung oder auch Arbeitsumfeld}, welche die Position und Größe der verschiedenen \Gls{Webbrowser}fenster speichern. Per Knopfdruck lassen diese sich dann im Kollektiv öffnen, sofern in den \Gls{Webbrowser}einstellungen die Popups aktiviert sind, für die zu testende Seite. Dieses Feature\footnote{Funktion, welche ein Teil der Anwendung ist} bewerte ich als positiv gegenüber dem Zeitaufwand, diesen Vorgang immer wieder manuell auszuführen.

		\\Als Kritikpunkt bewerte ich die nicht existente Möglichkeit, die Anwendung um eigene Funktionalität zu erweitern.

		\subsection{Tabellarische Evaluation}
		\met{Gewichtungstabelle Evaluation von Ghostlab}{10}{10}{8}{5}{0}{10}{5}
	
%%
%% ############# Adobe Edge Inspect
%%
	\pagebreak
	\section{\mi{Adobe Edge Inspect} CC }
		\subsection {Einrichtung der Testumgebung}
		Es sind drei Schritte notwendig Adobe Edge Inspect zum Einsatz vorzubereiten. Als erstes benötigen wir den Client aus der Adobe Creative \Gls{Cloud} (CC) Kollektion. Diese gibt es zum Zeitpunkt dieser Arbeit in verschiedenen Modellen und beginnt bei der kostenlosen 30-Tage Testversion, geht über die Einzellizenz, für ausschliesslich Adobe Edge Inspect, von 24,59 € / Monat bis hin zum Komplett-Abo, was dann mit 61,49 € / Monat zu Buche schlägt. Nach Starten des Clients läuft dieser als \Gls{Deamon} im Hintergrund. 
		\iga{../pictures/adobeedgeinspect/icon}{Adobe Edge Inspect \Gls{Deamon} Icon}{Der laufende \Gls{Deamon} von Adobe Edge Inspect}
		
		\\Als zweiten Schritt benötigen wir die zugehörige Chrome Extension von Adobe Edge Inspect. Diese wird über den Chrome \Gls{App}store installiert und kann nach einem \Gls{Webbrowser}neustart aktiviert werden.
		
		\\Als letztes benötigen wir noch die kostenlos erhältliche \Gls{App} aus dem jeweiligen Shop. Hier gilt für Android der Play Store und für iOS Geräte der \Gls{App}Store. Windowsgeräte werden derzeit nicht unterstützt.
		
		\\Sind diese drei Schritte erfolgreich durchgeführt worden, müssen nun die Geräte mit dem Server verbunden werden. Hierzu wird die \Gls{App} gestartet (der folgende Prozess verläuft unter Android wie auch unter iOS identisch) und per IP-Adresse mit der Adobe Edge Inspect Chrome Extension verbunden. Diese verlangt im Gegenzug einen Identifikationscode, welcher auf dem jeweiligen Gerät generiert wurde. Nach erfolgreicher Synchronisation wird das Gerät im Gerätemanager angezeigt.
		\igp{../pictures/adobeedgeinspect/iphone_2}{Adobe Edge Inspect \Gls{App} Client hinzufügen}{Eingabe der IP-Adresse zum 			Edge Inspect Rechner}{200}{350}
		\iga{../pictures/adobeedgeinspect/desktop_2}{Adobe Edge Inspect Chrome Extension}{Eingabe des Sicherheitscodes in die 		Chrome Extension}

		\\Dieser hat mehrere Funktionen. Er liefert eine Übersicht aller verbundenen Clients und ermöglicht das Aufrufen von \mi{weinre} um z. B. das \Gls{DOM} oder verwendete Ressourcen zu inspizieren oder \Gls{Javascript} auszuführen. Über den Gerätemanager lassen sich auch verbundene Geräte wieder durch einen Klick entfernen. Des Weiteren kann man über dieses Interface Screenshot von allen verbundenen Geräten im aktuellen Zustand aufnehmen und anzeigen lassen. Weiterhin besteht die Möglichkeit den Darstellungsmodus auf den verbundenen Clients von der \Gls{App}darstellung auf Vollbild zu ändern.
		\iga{../pictures/adobeedgeinspect/desktop_3}{Adobe Edge Inspect Gerätemanager}{Übersicht der verbundenen Clients}
		
		\subsection{Testen von Desktopbrowsern}
		Es gibt zum Zeitpunkt der Erstellung dieser Arbeit keine Möglichkeiten Desktopseiten mit Adobe Edge Inspect zu testen.
		
		\subsection{Testen von mobilen \Gls{Webbrowser}n}
		Die Funktionalität zum Testen von mobilen Seiten beschränkt sich derzeit auf den synchronen Aufruf von Seiten über den Chromebrowser, mit installierter Extension als Steuergerät. Die verbundenen Geräte erkennen den Aufruf von Links und das Wechseln von Tabs innerhalb des \Gls{Webbrowser}s. Es besteht wie bereits beschrieben die Option die einzelnen Clients per \mi{weinre} zu untersuchen.
		
		\\Die Simulierung eines Scrollevents oder das Ausfüllen eines Formulars ist nicht möglich. Es werden lediglich die Informationen dargestellt, die am Steuergerät aufgerufen wurden. Jedoch wird der Client, sofern vorhanden, auf die mobile Seite weitergeleitet. Während des Testens in der \Gls{App}, wird das Display aktiv gehalten, wodurch es sich nicht von selbst abschaltet. Ein gutes Feature von Adobe Edge Inspect ist die Möglichkeit aus dem Gerätemanager des \Gls{Webbrowser}s Screenshots der verbundenen Geräte anzufordern. Diese werden zusammen mit einer Beschreibung des Gerätes, dessen Modellbezeichnung, der \Gls{BA} sowie \Gls{Pixel}dichte, dem Betriebssystem, der aufgerufenen URL, sowie der aktuellen Ausrichtung des Bildschirms ausgeliefert.
		\igp{../pictures/adobeedgeinspect/iphone_3}{Adobe Edge Inspect \Gls{App} Content Darstellung}{Darstellung von Content in der 		Adobe Edge Inspect \Gls{App} unter iOS}{200}{350}
		
		Während meiner Versuche ist mir aufgefallen, dass Adobe Edge Inspect unter iOS 6.1.3, Seiten die durch \gls{htaccess} 				gesichert sind, nicht darstellen kann. Auf den anderen Testgeräten verlief der Prozess der Authentifizierung problemlos. 
		
		\pagebreak
		\subsection{Fazit zu Adobe Edge Inspect}
		Adobe Edge Inspect bedarf viel Aufwand für ein unzureichendes Ergebnis. Man muss an drei verschiedenen Punkten Installationen vornehmen, die dann jedoch \mbox{ohne} Probleme miteinander harmonieren. Als besonders positiv möchte ich die Screenshotfunktion bewerten. In Zusammenspiel mit der öffentlich zugänglichen API lassen sich hierüber Screenshots im Landschafts-, als auch im \Gls{PoW} anfordern und durch eine externe \Gls{App}likationen auswerten. 
		
		\\Der Nutzen des Werkzeuges liegt am ehesten bei One-Page-Sites\footnote{Webseiten dessen Inhalt sich füllend auf die gesamte Seite erstrecken} oder für Fehlersuche innerhalb des \Gls{DOM} oder CSS Anpassungen mit \mi{weinre}. Unter dem Aspekt des \gls{parallel-synchron}en Testens ist Adobe Edge Inspect nicht sinnvoll zu verwenden, da weder Steuerbefehle oder andere Gesten umgesetzt werden, noch werden die Nutzereingaben in Eingabefeldern mit anderen verbundenen Clients geteilt. Alle verbundenen Clients sind nur Empfänger und besitzen keine Möglichkeit als Sender zu fungieren. Folglich gehen alle Steuerbefehle vom Edge-Server aus.
	
	\subsection{Tabellarische Evaluation}
		\met{Gewichtungstabelle Evaluation von Adobe Edge Inspect}{10}{7}{0}{4}{8}{3}{10}


%%
%% ############# Remote Preview
%%
	\pagebreak			
	\section{\mi{Remote Preview}}
		\subsection {Einrichtung der Testumgebung}
		Es gibt zwei Möglichkeiten dieses Werkzeug zu nutzen. Die eine ist die Installation auf einem lokalen \Gls{Apache}-Server mit \Gls{PHP}. Die andere ist die Installation auf einem \Gls{Cloud}-Dienst wie z. B. Dropbox. Die Ergebnisse dieser Arbeit wurden mit einer lokalen \Gls{Apache} Installation erzielt. Die Installation sieht lediglich vor, das \Gls{Framework} in einen lokalen Entwicklungszweig zu entpacken.
		
		\subsection{Testen von Desktopseiten}
		Alle Clients, die in die Testumgebung eingebunden werden sollen, müssen lediglich die IP-Adresse des Servers eingeben.
		Die Steuerung der Seiten erfolgt sowohl für Desktopseiten, als auch für die mobilen Vertreter über die \Gls{Webbrowser}maske des \Gls{Framework}s. In das untere der beiden Eingabefelder, gibt man die aufzurufende URL inklusive Präfix\footnote{http://} ein. Diese wird dann auf allen verbundenen Clients innerhalb eines \gls{iFrame}s dargestellt. 
		\ig{../pictures/remotepreview/eingabemaske}{Remote Preview Steuerungsmaske}{Steuerungsmaske zur Eingabe der 			aufzurufenden URL}
				
		 \subsection{Testen von mobilen \Gls{Webbrowser}n}
		 Das Testen der mobilen \Gls{Webbrowser} funktioniert parallel zum Testen von Desktopseiten. Positiv möchte ich hier erwähnen, dass das \Gls{Framework} auch wenn es dafür nicht ausgelegt ist, unter aktuellen Windowsgeräten funktioniert.			
		
		\subsection{Fazit zu Remote Preview}
		Ein positiver Punkt, ist die Möglichkeit, letztendlich jeden \Gls{Webbrowser} unabhängig von dessen Betriebssystems in die 				Testumgebung zu integrieren, da diese einfach nur auf den \Gls{Apache}server oder die Dropbox zugreifen müssen. Als negativ führe ich hier die Tatsache auf, dass es ähnlich Adobe Edge Inspect lediglich dem Aufrufen von Seiten dient, jedoch nicht deren Bedienung. So ist es nicht möglich, weiteren Verlinkungen zu folgen, ohne diese von Hand in die Eingabemaske einzutragn oder Formulare auszufüllen. Das Darstellen von Seiten mit \Gls{Anker}n, funktioniert nur bedingt. Das Aufrufen von \gls{gesichert}en Seiten gelang nicht. Ebenfalls war es nicht möglich, zertifizierte Webseiten aufzurufen, was den Nutzungsgrad des \Gls{Framework}s stark einschränkt. Gut finde ich die Tatsache, dass Quellcode komplett zugänglich ist, da er 		unter der MIT Lizenz steht und	jederzeit in eigene Projekte eingebunden oder um eigene Funktionalität erweitert werden kann. Das Projekt scheint zum Zeitpunkt dieser Arbeit nicht weiterentwickelt zu werden. Für den Aufruf einer einfachen Seite auf n-Geräten ist dieses Projekt eine kostenlose Alternative zu Adobe Edge Inspect mit geringerem Funktionsumfang.
		
				
		\subsection{Tabellarische Evaluation}
		\met{Gewichtungstabelle Evaluation von Remote Preview}{8}{6}{3}{3}{8}{10}{0}
		
	
%%
%% ############# Browser-Sync
%%
\pagebreak
	\section{\mi{Browser-Sync}}	
	\subsection{Einrichtung der Testumgebung}
	Als Vorraussetzung um Remote-Sync nutzen zu können, wird zu Beginn erst einmal eine \gls{NodeJS} Implementation benötigt. Diese kann entweder über die Konsole installiert werden oder per Installationstool von der \gls{NodeJS} Homepage.
	
	\\Nach der \gls{NodeJS} Installation wird per \Gls{NPM} das Paket von Browser-Sync per Konsole einmalig installiert:
	\iga{../pictures/browser-sync/install}{Browser-Sync Installation per Konsole}{Konsolenbefehl um Browser-Sync zu installieren}
	
	Nun muss für jedes neue oder bestehende Projekt einmalig im Projektordner Browser-Sync initialisiert werden. Browser-Sync legt in dem aktuellen Verzeichnis eine Konfigurationsdatei ab, in welcher man einzelne Optionen, wie z. B. die zu beobachtenden Dateien oder Einstellungen zum Synchronisationsverhalten, festlegen kann. Dieser Aufruf erfolgt ebenfalls über die Konsole.
	\igp{../pictures/browser-sync/init}{Browser-Sync Initiierung per Konsole}{Konsolenbefehl um Browser-Sync zu imitieren}{450}{200}
	
	Nach der Initiierung des Servers startet man diesen mit dem Befehl :
	\iga{../pictures/browser-sync/start}{Browser-Sync Starten per Konsole}{Konsolenbefehl um Browser-Sync zu starten}
	
	Um nun die Kommunikation zwischen Server und Projekt zu gewährleisten, muss vor dem Ende des Body Elements der Indexdatei zusätzlicher Scriptcode eingefügt werden, welcher jedoch zum Release entfernt werden sollte. Der einzufügende Code  wird anhand der Konfigurationsdatei und der IP-Adresse des Servers generiert und per Konsole dem Nutzer mitgeteilt.
	\igp{../pictures/browser-sync/start2}{Browser-Sync Script-Tag}{Konsolenausgabe mit einzufügendem Quellcode}{450}{200}
	
	In künftigen Versionen wird es laut dem Entwickler nicht mehr notwendig sein die Versionsnummer mitanzugeben.
	
	\subsection{Testen von Desktopseiten}
	Das Testen erfolgt durch Aufruf der Seite, in die der Steuercode eingetragen wurde, über den Browser. Die parallele Steuerung erfolgt direkt und synchron. Ist ein Browser erfolgreich verbunden, wird dies in der Konsole des Servers angezeigt.
	\iga{../pictures/browser-sync/connected}{Browser-Sync verbundener Client}{Konsolenausgabe bei erfolgreich verbundenem Client}
	Das Folgen interner Links funktioniert nur unidirektional, sofern der Steuercode nicht mittels \Gls{Framework} oder von Hand in die verlinkten Dateien eingefügt wurde. Das Folgen externer Links erfolgt ebenfalls nur unidirektional. Auch das Aufrufen zertifizierter oder \gls{gesichert}er Seiten mit Passworteingabe, funktioniert problemlos. Beim Nutzen von Steuerbefehlen, traten nur bedingt Probleme auf. So gibt es zum Zeitpunkt dieser Arbeit Defizite im Umgang mit dem \Gls{Javascript}framework jQuery. So lassen sich z. B. Lightboxen öffnen, jedoch werden dann Befehle zum \mbox{Schließen} des Fensters nicht mehr erkannt und übermittelt. Das Ausfüllen von Formularen verlief bis auf eine Mehrfachauswahl fehlerfrei. Das Erkennen von Hoverevents funktionierte in der Version 0.7.2 noch nicht. Auch das parallele Verwenden von Sliderelementen war zu diesem Zeitpunkt noch nicht implementiert.
	
	\subsection{Testen von mobilen Browsern}
	Das Testen von mobilen Browsern verläuft parallel zu Desktopseiten. Ein Aufruf über den internen Browser genügt um den Client am Server zu registrieren. Als zusätzliches Problem trat bei den mobilen Geräten ein Verziehen der Elemente auf. Die Geräte richten sich anhand der gescrollten Entfernung aus und nicht der Ausrichtung an der \Gls{HTML}-Struktur. Zum Zeitpunkt dieser Thesis bietet das \Gls{Framework} nicht die Möglichkeit der Ausrichtung an \Gls{DOM}-Elementen der Internetseite. So kommt es bei den Testgeräten zu Unstimmigkeiten in der Darstellung des Inhalts, welche durch die unterschiedlichen \Gls{BA}en und Ausrichtungen der Geräte zu Stande kommen.
	
	\subsection{Fazit zu Browser-Sync}
	Browser-Sync ist ein vielversprechendes \Gls{Framework}, welches die zu untersuchenden Aspekte vollkommen abdeckt. Es bestehen noch relativ viele unausgereifte Komponenten, jedoch werden diese bei Auftreten zeitnah von den Entwicklern behoben. Generell scheint das \Gls{Framework} zum Zeitpunkt dieser Arbeit eine hohe Entwicklungsgeschwindigkeit zu besitzen. Es trat gelegentlich ein Fehler auf, bei dem ein verbundener Client, selbst nach mehrfacher Neuverbindung, nicht mehr auf die Steuersignale reagierte. Dieser Fehler trat meistens bei mehr als sechs verbundenen Clients auf. Das \Gls{Framework} ist zum Validieren von Websites gedacht, die sich noch in der Entwicklung befinden. Das Testen ist, aufgrund der notwendingen Testumgebung, nur als lokales Arbeiten angedacht. Als Pluspunkt wird das Injizieren von geändertem Code zur Entwicklungszeit gewertet. So ist es zum Beispiel möglich, vorgenommene Änderungen am Styling oder dem \Gls{DOM}, ohne weitere Handgriffe direkt auf allen Testgeräten zu begutachten.
	
	\subsection{Tabellarische Evaluation}
		\met{Gewichtungstabelle Evaluation von Remote Preview}{5}{2}{9}{7}{8}{10}{10}
	
	
	\section{Eigenes \mi{\Gls{Framework}}}
	Der ursprüngliche Gedanke dieser Arbeit verfolgte den Ansatz ein eigenes \Gls{Framework} zu entwickeln, was die \gls{parallel-synchron}e Steuerung auf mehreren \Gls{moEn}en insbesondere auf mobilen Geräten ermöglicht. Diesen Gedanken berücksichtigend, erfolgte eine Validierung verschiedener Einzeltechnologien, die nur gewisse Aspekte abdecken. Untersucht wurden diese hinsichtlich auf ihre tatsächliche Funktionalität, ihrer Installation und Kombinierbarkeit mit anderen verwendeten \Gls{Framework}s.
	
\\Die Bibliotheken werden insbesondere auf ihre Fähigkeit untersucht, sie in einen \gls{NodeJS} Server zu implementieren.

	\subsection{Installation eines \gls{NodeJS} Servers}
	Die Installation des \gls{NodeJS} Servers erfolgt einfach über die Konsole unter Mac oder den Installer\footnote{erhältlich unter Nodejs.org}. Alleinstehend erfüllt dieser Server keinerlei der gewünschten Funktionen, jedoch dient dieser als Grundlage für einige nachfolgende \Gls{Framework}s. \gls{NodeJS} ist eine gute Wahl aufgrund der hohen Verarbeitungsgeschwindigkeit, sowohl Client-, als auch Server-seitig. Weitere Pluspunkte sind die rasche Entwicklungsgeschwindigkeit, die hohe Vielfalt an Erweiterungen und Plugins, sowie eine sehr große aktive Entwicklergemeinde.
	\subsection{Einbinden von socket.io}
	socket.io lässt sich einfach über den \Gls{NPM} installieren. Es ermöglicht das Herstellen einer permanenten Verbindung mit dem Server über einen Socket. Der Vorteil liegt hierbei darin, dass keine zyklischen Anfragen an den Server gesendet werden. Stattdessen wird hier das Observer-Pattern umgesetzt und alle verbundenen Clients werden vom Server informiert, sobald eine Änderung des Status stattgefunden hat. 
	
	\subsection{Generierung von Steuerbefehlen über socket.io}
	Der generelle Aufbau von socket.io sieht vor, dass der Client sich mit dem Server verbindet und eine permanente Verbindung mit diesem aufrecht erhält. Identifizierbar bleibt diese über eine generierte, einzigartige, alphanumerische Session ID. socket.io funktioniert nach den Observer-Pattern, was bedeutet, dass der Client nicht in zyklischen Abständen Anfragen an den Server sendet, sondern bei einer Änderung der Modelle oder zum Beispiel einem Funktionsaufruf vom Server mittels eines Broadcasts informiert wird. So entsteht das Problem, dass wenn ein Client eine Nachricht an den Server sendet, dieser allen Clients (auch dem Auslöser) diese Nachricht sendet. 
	
	\igp{../pictures/broadcast}{socket.io vereinfachte Darstellung eines Broadcasts}{socket.io Broadcast \cite{3}}{300}{300}
	
Für die Entwicklung eines eigenen \Gls{Framework}s wirft dies einige Probleme auf.

\\ Ein  Beispiel: Ein Nutzer klickt auf einer Seite auf einen Button. Das 'Senden'-Event wird an den Server gesendet und an alle per Socket verbundenen Clients dupliziert. Somit wird der ursprüngliche Sender des Signals, erneut das gleiche \Gls{Event} erhalten. Das Resultat ist, dass dieser den Button zweimalig drückt. Das kann zu Problemen führen, weil beispielsweise eine clientseitige Aktion mehrfach ausgeführt wird. Ein weiteres Problem kann durch rekursives Aufrufen einer Methode einen Dead-Lock erzeugen. Clients, die die empfangen Signale langsamer als andere verarbeiten, können in dem Moment vom Empfänger direkt wieder zum Sender werden.
	
	\\Daher ist der Ansatz ein Master-Slave-Pattern umzusetzen denkbar sinnvoll. Es wird ein Steuergerät definiert, welches seine Aktionen dem Server mitteilt und dieser die \Gls{Event}s dann an alle verbundenen Clients innerhalb eines Aktionsraumes sendet.
	
	\igp{../pictures/broadcastroom}{socket.io vereinfachte Darstellung eines Broadcasts mit einem Aktionraum}{socket.io Broadcast mit Aktionsraum}{350}{350}
	
	\subsection{Implementierung eines einfachen Broadcast}
	Das Beispiel soll veranschaulichen, wie ein einfacher Broadcast ohne Aktionsraum implementiert werden kann. Das Beispiel soll das Scrollevent des Clients abfangen und auf allen verbundenen Clients zur selben Position auf dem Bildschirm scrollen.
	
	\subsubsection{Serverseitig}
	Zu Beginn werden die notwendigen Bibliotheken eingebunden um einen \gls{NodeJS} Server starten zu können (Zeilen: 1-2). Im Anschluss wird  eine Serverinstanz von \gls{NodeJS} erstellt (Zeile: 4) und gestartet auf Port 8001 (Zeile: 6). Dieser wird nun mit der socket.io Bibliothek verknüpft (Zeile: 7). Sofern nun ein Client sich mit dem Server verbindet wird das 'connection'-Event gefeuert (Zeile: 9) und der Client wartet auf ein selbst-definierten Methodenaufruf vom 'scroll' (Zeile: 11).
	
	\\Wenn am Server ein Srollevent eingegangen ist, sendet dieser dies per Broadcast an alle verbundenen Clienten (Zeile: 12).
	
	\igp{../pictures/socketio/server_1}{socket.io Quellcode Scrollbeispiel Serverseitig}{minifizierter Quellcode Serverseitig}{350}{350}
	\subsubsection{Clientseitig}
	Auf der Seite des Clients müssen zwei Methoden implementiert werden. Zum einen die Methode, die das Scrollevent des Browsers, was hier über jQuery erfolgt, abfängt und über die Socketverbindung die Servermethode 'scroll' aufruft und die aktuelle Scrollposition zum oberen Bildschirmrand übergibt.
	
	\igp{../pictures/socketio/client_1}{socket.io Quellcode Scrollbeispiel Clientseitig}{minifizierter Quellcode Clientseitig}{350}{350}
	
	Zum anderen muss die Methode implementiert werden, welche vom Server gesendete \Gls{Event}s abfängt und verarbeitet. In diesem Beispiel wartet der Client auf ein \Gls{Event} vom Typ 'scroll'. Dieses bekommt einen Datensatz, die Scrollposition, mitgeliefert. Nach erfolgreichem \Gls{Event}aufruf wird per jQuery die Position des Bildausschnitts an den des mitgelieferten Datensatzes angepasst.
	
	\igp{../pictures/socketio/client_2}{socket.io Quellcode Scrollbeispiel Clientseitig}{minifizierter Quellcode Clientseitig}{250}{250}

	\subsection{Einschätzung zur Umsetzung eines eigenen \Gls{Framework}s}
	Der Realisierung eines eigenen \Gls{Framework}s zur \gls{parallel-synchron}en Steuerung von Webseiten steht nichts im Wege. Ein positiver Aspekt ist die sehr schnelle Datenübertragung in nahezu Echtzeit mittels \gls{NodeJS}. Voraussetzung ist hierfür, dass die Geräte sich im gleichen lokalen Netz befinden. Ein weiterer positiver Aspekt ist die einfache Implementierung von Steuersignalen über socket.io. Die modulare Grundstruktur der \Gls{Framework}s ermöglicht einen einfachen Einstieg in den Umgang mit dem \Gls{Framework}. So bietet es die Optionen die Standardfunktionen zu nutzen oder aber um eigene Funktionalität zu erweitern.

\\Die Struktur ermöglicht es sämtliche \Gls{Event}s abzufangen, egal ob mit jQuery oder anderen \Gls{Framework}s zur \Gls{Event}ermittlung, um diese dann in entsprechende Funktionen umzuwandeln und an alle Clients weiterzugeben. Der Einsatz eines Mastergerätes und das Nutzen von Aktionsräumen verhindern die irreführenden Rückkopplungen innerhalb des Nachrichtenzyklus. 

\\Der Einsatz in einer virtuellen Umgebung erfolgt problemlos, da keine weitere Software installiert werden muss. Die Verwendung von socket.io ermöglicht die Unterstützung aller alten Browser Plattformen, da das \Gls{Framework} mit einer Reihe von Fallbacks sich gegen Funktionsverlust absichert. Sollte keine WebSocket-Technologie verfügbar sein, greift das \Gls{Framework} zuerst auf Adobe Flash Sockets zurück und sollten diese auch nicht verfügbar sein, auf eine Reihe verschiedener Long-Polling-Ansätze um die Kommunikation weiterhin zu gewährleisten.













%%
%% ############# Fazit und Ausblick
%%	
\chapter{Zusammenfassung und Ausblick}
Zum Abschluss der Arbeit, werde ich in einem Fazit auf gesetzte Ziele eingehen und die Vorgehensweise der Evaluierung schildern. Im Anschluss wird ein kleiner Ausblick gewährt, was basierend auf dem aktuellen Stand der \Gls{Framework} eventuell verbessert oder noch entwickelt werden könnte.

\section{Zusammenfassung}
Das Ziel dieser Arbeit war die Evaluierung von Techniken zur
parallel-synchronen Bedienung einer
Web-Applikation auf verschiedenen
mobilen \Gls{moEn}en. Zu Beginn wurden erst einmal \Gls{Framework}s ermittelt, welche die gesetzten Kriterien versprechen ganz oder zu großen Teilen abzudecken. 
Derzeit gibt es nur sehr wenige Anbieter von Produkten für parallele Webseitentests und das Angebot wird durch den Wunsch diese Tests auf mobile Geräte zu erweitern eingeschränkt. Auf Grund dessen wurden auch Segmentframeworks analysiert, welche in Kombination miteinander die Möglichkeit bieten, die geforderten Kriterien zu erfüllen. Im nächsten Schritt wurden diese kurz vorgestellt.

\\Als nachfolgender Schritt wurde ein Evaluationsschlüssel festgelegt, welcher zum einen Teil aus geforderten Kriterien abgeleitet wurde und zum anderen im Laufe dieser Arbeit um Kriterien erweitert, welche beim Einrichten und Verwenden der einzelnen Werkzeuge als für die Evaluierung wichtig empfunden wurden. Anhand des Schlüssels konnten die Komplettframeworks miteinander, in ihren Stärken und Schwächen, gewertet werden.

\\Im Anschluss wurden die \Gls{Framework}s installiert, konfiguriert, mit dem Desktopbrowser sowie mit einer Vielzahl von mobilen \Gls{moEn}en getestet, wobei auftretende Fehler oder Lob für eine gute problemlösende Funktion dokumentiert wurde. Am Ende jedes \Gls{Framework}testes erfolgte eine tabellarische Evaluierung in welcher Pro und Contra ersichtlich wurde. Außerdem wurde ein Wert anhand der einzelnen Unterkategorien errechnet, welcher in der Gesamtwertung einen Vergleich der \Gls{Framework}s untereinander ermöglicht.

\\Das Ziel der Arbeit war es durch das effiziente Testen von Web-Applikationen mehr Qualität zu erreichen und dies in weniger Zeit, als es im herkömmlichen Sinne notwendig ist. Dieses Ziel erreichte leider keins der getesteten \Gls{Framework}s vollends, da sie nie alle Kriterien abdeckten und somit nach wie vor von Hand nachgetestet werden musste. 
Die einzelnen \Gls{Framework}s haben in der Regel einen Schwerpunkt gut abgedeckt, jedoch dafür andere Aspekte vernachlässigt. Positiv ist das open-source Projekt Browser-Sync aufgefallen. Es erfüllte am besten die gesetzten Kriterien und ist zusätzlich um eigene Funktionalitäten erweiterbar. Zusätzlich basiert Browser-Sync auf Node.JS was es dem Entwickler ermöglicht auf die umfassende Paketdatenbank von \Gls{NPM} zuzugreifen und sie in das \Gls{Framework} zu implementieren. 

\\Auch das kommerzielle Ghostlab Produkt hat ein sehr solides Grundgerüst, jedoch gibt es gerade im Bereich des mobilen Testen eine Schwachstelle die das Arbeiten, zumindest mit älteren Generationen von Mobilgeräten, fast unmöglich macht. Da es leider keine Möglichkeit bietet das Programm um eigenen Code zu erweitern und dieser derzeit noch über keinen Master-Slave-Modus verfügt, verfängt es sich sehr schnell in einem zyklischen Deadloop.

\\Abschließend ist zu sagen, dass nach den gesetzten Kriterien, zum Zeitpunkt der Erstellung der Arbeit, keines der Komplettframeworks in der Lage ist den \Gls{qs}sprozess effizient zu optimieren.

\\Ich persönlich habe während dieser Arbeit gelernt wie wichtig eine gut strukturierte Planung ist. Diese um Zeitfenster für Eventualitäten zu erweitern und dennoch im Zeitrahmen zu bleiben, war eine große Herausforderung, da Teile dieser Arbeit experimentell waren und daher schwer in Zahlen zu fassen. Das evaluieren der einzelnen \Gls{Framework}s lief hingegen überwiegend innerhalb des gesetzten Rahmens. Eine weitere Herausforderung war es die getesteten \Gls{Framework}s zu vergleichen und dies in Zahlen darzustellen, ohne dabei willkürlich zu wirken. Letzendlich hatte ich auch erhofft ein \Gls{Framework} zu finden, was alle meine Wünsche nach einem verbesserten Workflow erfüllt, jedoch stecken viele der Technologien noch in den Kinderschuhen und einige davon sind aber auf dem richtigen Weg und arbeiten mit Nachdruck daran dieses Ziel auch zu erreichen.


	 \vspace{-20pt}
	 \newcolumntype{C}[1]{>{\centering\arraybackslash}p{#1}}
	\begin{table}[H]
 		\centering
		\rowcolors{1}{white}{lightgray}
			\begin{tabular}{| p{9cm} | C{1cm} | C{1cm} | C{1cm} | C{1cm} |}
			\hline
							 	&	GL	&	AEI	&	RP	&	B-S\\
			\hline
			\hline
				Installation						&		&		&		&	\\
				Notwendigkeit von anderen Technologien	&	4	&	4	&	4	&	2\\
				Nutzbar direkt nach der Installation		&	2	&	2	&	2	&	2\\
				Installationsanleitung vorhanden		&	2	&	2	&	2	&	1\\
				FAQ vorhanden						&	2	&	2	&	0	&	0\\
				\hline
				\hline
				Konfiguration						&		&		&		&	\\
				Nutzbar ohne Konfiguration			&	4	&	4	&	4	&	0\\
				Konfigurierbarkeit (IP und Ports)		&	1	&	0	&	1	&	1\\
				Konfigurierbare \Gls{Workspace}s			&	2	&	0	&	0	&	0\\
				Support vorhanden (Wiki, EMail, Forum)	&	2	&	2	&	0	&	1\\
				Intuitive Benutzeroberfläche			&	1	&	1	&	1	&	0\\
				\hline
				\hline
				Funktion: Desktop					&		&		&		&	\\
				Darstellung : normale Seiten			&	2	&	0	&	1	&	2\\
				Darstellung : \Gls{gesichert}e Seiten			&	2	&	0	&	0	&	2\\
				Darstellung: Reaktionsgeschwindigkeit	&	2	&	0	&	2	&	2\\
				Funktion: Seitensteuerung				&	3	&	0	&	0	&	3\\
				Funktion: \Gls{Javascript}					&	1	&	0	&	0	&	0\\
				\hline
				\hline
				Funktion: Mobil						&		&		&		&	\\
				Darstellung : normale Seiten			&	1	&	1	&	1	&	1\\
				Darstellung : \Gls{gesichert}e Seiten			&	0	&	1	&	0	&	1\\
				Darstellung: Reaktionsgeschwindigkeit	&	2	&	2	&	2	&	2\\
				Funktion: Seitensteuerung				&	2	&	0	&	0	&	3\\
				Funktion: \Gls{Javascript}					&	0	&	0	&	0	&	0\\
				Funktion: Gestenkontrolle				&	0	&	0	&	0	&	0\\
				\hline
				\hline
				Erweiterbarkeit						&		&		&		&	\\
				API Zugang						&	0	&	5	&	5	&	5\\
				API lizenzteschnich gesichert			&	0	&	0	&	2	&	2\\
				API Dokumentation					&	0	&	3	&	1	&	1\\
				\hline
				\hline
				Browser Support (aktuelle Versionen)	&		&		&		&	\\
				mobile Plattformen					&	3	&	2	&	3	&	3\\
				virtuelle \Gls{Webbrowser}					&	2	&	0	&	2	&	2\\
				Chrome							&	1	&	1	&	1	&	1\\
				Opera							&	1	&	0	&	1	&	1\\
				Firefox							&	1	&	0	&	1	&	1\\
				Safari							&	1	&	0	&	1	&	1\\
				Internet Explorer					&	1	&	0	&	1	&	1\\
				\hline
				\hline
				Aktivität							&		&		&		&	\\
				Noch in der Entwicklung				&	5	&	5	&	0	&	5\\
				Aktives Forum						&	0	&	5	&	0	&	5\\
				\hline
				\end{tabular}
			\caption{{\"U}bersicht der \Gls{Framework}s: GL(Ghostlab), AEI(Adobe Edge Inspect), RP(Remote Preview), B-S(Browser-Sync)}
	\end{table}


\subsection{Ausblick}

Derzeit besitzt Browser-Sync wohl das höchste Potential, ein Produkt zu erschaffen, welches die Produktivität in der Web-Applikationsentwicklung steigert. Die lebendige Open-Source-Community ist aktiv und engagiert ein hochwertiges \Gls{Framework} zu erschaffen, welches leichtgewichtig und flexibel einsetzbar ist. 

\\ Ein weiterer Kandidat mit hohem Potential ist Ghostlab, welches zwar schon gute Allroundansätze derzeit vorweisen kann, jedoch leider im Detail nicht ausgereift ist. Die Nachfrage nach einem qualitativ hochwertigem Produkt ist vorhanden, wie es die große Anzahl\footnote{Eine große Auswahl findet sich z. B. unter http://opendevicelab.com/} an \Gls{Devicelab}s Weltweit zeigt.

\\ Auch dem Ansatz ein eigenes Produkt für die spezifischen Anforderungen eines Betriebs zu entwickeln, steht in der Zukunft nichts im Wege, da bereits zu diesem Zeitpunkt Node.JS, Socket.io und darauf aufbauende Technologien ein fundiertes Grundgerüst liefern. 
Dieses erfordert zwar eine leicht erhöhte Einarbeitungszeit, jedoch lassen sich hiermit die eigenen Spezifikationen verfolgen und umsetzen. Weiterhin förderlich, ist auch die sehr große und aktive Community rund um die genannten \Gls{Framework}s, welche auch in Zukunft voraussichtlich viele Einzelaspekte verfolgen, die sich dann einfach über das Node.JS-eigene Paketverwaltungstool implementieren lassen.










%%
%% ############# Glossarausgabe
%%
\clearpage
\printglossary[style=altlist,title=Glossar]

%% 
%% ############# Indexverzeichnis
%\printindex
%%
%% ############# Literaturverzeichnis
%%

\begin{thebibliography}{999}

\bibitem[RJBF88]{1}Ralph E. Johnson, Brian Foote: “Designing Reusable Classes” im "Journal of Object-Oriented Programming" (1988)
\bibitem[Wiki01]{2} \url{http://de.wikipedia.org/w/index.php?title=Ajax\_(Programmierung)&oldid=129355492}
\bibitem[ScVi]{3}\url{http://irlnathan.github.io/sailscasts/blog/2013/10/10/building-a-sails-application-ep21-integrating-socket-dot-io-and-sails-with-custom-controller-actions-using-real-time-model-events/}
\bibitem[Wiki02]{4} \url{http://de.wikipedia.org/w/index.php?title=JavaScript&oldid=129548016}
\bibitem[Wiki03]{5}\url{http://de.wikipedia.org/w/index.php?title=Bildaufl%C3%B6sung&oldid=129700697}
\bibitem[Wiki04]{6}\url{http://de.wikipedia.org/w/index.php?title=Document\_Object\_Model&oldid=127850631}
\bibitem[Wiki05]{7}\url{http://de.wikipedia.org/w/index.php?title=Apache\_HTTP\_Server&oldid=129948624}
\bibitem[Wiki06]{8}\url{http://de.wikipedia.org/w/index.php?title=Anker\_(HTML)&oldid=124884439}
\bibitem[Wiki07]{9}\url{http://de.wikipedia.org/w/index.php?title=Digitales\_Zertifikat&oldid=129208084}
\bibitem[SeHt01]{10}\url{http://de.selfhtml.org/servercgi/server/htaccess.htm}
\bibitem[Wiki08]{11}\url{http://de.wikipedia.org/w/index.php?title=GNU\_General\_Public\_License&oldid=130239300}
\bibitem[Wiki09]{12}\url{http://de.wikipedia.org/w/index.php?title=MIT-Lizenz&oldid=125762366}
\bibitem[Wiki10]{13}\url{http://de.wikipedia.org/w/index.php?title=Apache-Lizenz&oldid=130033063}


\end{thebibliography}

\end{document}



