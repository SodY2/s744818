\chapter{Technologien}
\section{\mi{Ghostlab}}
	Ghostlab ist ein Framework des Schweizer Unternehmens Vanamco. Es verspricht das synchrone Testen von Websiten in Echtzeit. Weiterhin wirbt das Unternehmen mit einem umfangreichen Repertoire an nützlichen Fähigkeiten. Der Funktionsumfang umschliesst das Scrollen innerhalb einer Seite, das ausfüllen von  Formularen, das wahrnehmen und reproduzieren von Click-Events sowie dem neuladen einer Seite. Ghostlab soll ebenso einen Inspektor besitzen, welcher die Analyse des DOMs, der on the fly Bearbeitung der CSS und der Analyse und Bearbeitung von Javascriptdateien. Das Framework gibt an für alle folgenden Browser zu funktionieren ohne diese Konfigurieren zu müssen:

	\begin{table}[h]
 		\centering
		\rowcolors{1}{white}{lightgray}
			\begin{tabular}{| p{5cm} | p{5cm} |}
			
			\hline
				Browser 	& 	Version\\
			\hline
			\hline
				Firefox	&	latest\\
				Chrome	&	latest\\
				Safari	&	latest\\
				Internet Explorer	&	8/9/10\\
				Opera Mobile	&	supportet\\
				Opera	&	11\\
				FireFox Mobile	&	supportet\\
				Blackberry	&	supportet\\
				Windows Phone	&	supportet\\
				Safari mobile	&	supportet\\	
				Android	&	2.3 - 4.2\\
				\end{tabular}
			\caption{von Ghostlab getestete Browser (stand 10.03.2014, Version 1.2.3)}
	\end{table}

Der Kostenpunkt der Lizenz liegt zur Erstellung dieser Arbeit bei 49\$ (entspricht 35,30€ beim aktuellen Umrechnungswert). Zur Erstellung dieser Thesis wurde die 7-Tage-Testvollversion genutzt.
