\chapter{Grundlagen}
In diesem Abschnitt behandle ich spezifische Definitionen wie zum Beispiel verwendetes Fachvokabular, allgemeine technische Abläufe die Notwendig sind um diese Arbeit und die darin verwendetet Techniken zu verstehen, sowie verwendete Hardwarekomponenten.
%\tr{Dieser Teil beschreibt das fachliche Umfeld der Aufgabenstellung. Hier werden die wesentlichen fachlichen Begrifflichkeiten, die für die Aufgabe relevanten Problemstellungen und Lösungsansätze des Fachgebietes vorgestellt. Der Sprachgebrauch sollte einen direkten Bezug zum Fachgebiet haben. Die notwendigen Darstellungsmethoden, die Art und der Umfang der Beschreibung hängen wesentlich von der jeweiligen Fachdisziplin ab und sollten im Dialog mit dem Betreuer entschieden werden. Beispielsweise wird sich die Beschreibung eines Hotelreservierungssystems sehr von einer Beschreibung mathematischen Transformationen auf Grafikobjekte unterscheiden.Dies ist oft vor der Einleitung das erste Kapitel, das man schreibt, und sollte einen Überblick über die Literatur und existierende Arbeiten im Bereich der Arbeit liefern (Welche Grundlagen gibt es in diesem Bereich? Haben andere Autoren schon etwas zu verwandten Themen veröffentlicht?). Die hier vorgestellten Konzepte sollten in der Einleitung zumindest schon einmal angesprochen worden sein. Bei der Vorstellung verwandter Arbeiten sollten neuere Erkenntnisse bevorzugt werden. Bei jeder in das Grundlagenkapitel aufgenommenen Arbeit gilt es herauszustellen, was die Ergebnisse der Arbeit waren und warum diese Ergebnisse (oder Einschränkungen der vorgestellten Arbeit) eben noch keine Schließung der Wissenslücke oder keine Lösung der Aufgabenstellung darstellen? Am Ende erfolgt eine kurze Zusammenfassung der Grundlagen, die begründete Schlussfolgerung, dass das zu untersuchende Problem noch ungelöst ist, und ggf. wieder eine Vorschau auf das folgende Kapitel.}

	%%
	%% ############# Begriffsklärung
	%%
	\section{Begriffsklärung}	
		\subsection{parallel-synchron}
		\subsection{Web-Applikation}
		\subsection{HTML}
		Die Hypertext Markup Language ist eine Auszeichnungsprache zur Beschreibung von Inhalten. Sie dient der Strukturierung 		von Texten, Links\footnote{Verweise zu anderen Inhalten}, Listen und Bildern eines Dokumentes. Eine HTML Seite wird von 		einem Webbrowser interpretiert und anschließend dargestellt. Die Entwicklung von HTML geschieht durch das World Wide 		Web Consortium(W3C) und den Web Hypertext Application Technology Working Group (WHATWG). 

		\subsection{Webbrowser}
		\subsection{Desktopcomputer / Desktops}
		In dieser Arbeit werden gängige Modelle von Personal Computern oder Macs mit einem festen Arbeitsumfeld als Desktops 		bezeichnet. Hierzu zählen auch tragbare Modelle und Laptops. Im Sinne der Thesis umschließe ich nachfolgend mit dem 		Begriff Desktop oben genannte Komponenten. Dies dient später der Differenzierung ob es sich um ein mobiles Endgerät 			handelt oder einem Computer .

		\subsection{Mobiles Endgerät}
		Im Nachfolgenden werden Komponenten mit primärer mobiler Nutzung umfassend als mobile Endgeräte gruppiert. Hierzu 		zählen Smarthphones und Tablets.

		\subsection{Javascript}
		\subsection{Framework}
		\subsection{Nodejs}
		\subsection{PHP}
		\subsection{NPM}
		\subsection{Qualitätssicherung}
		\subsection{VirtualBox / virtuelle Umgebung}
		\subsection{Smartphone}
		\subsection{Tablet}
		\subsection{Panorama / Portrait View}
		\subsection{Pixel}
		\subsection{Auflösung}
		\subsection{Event}
		\subsection{DOM}
		\subsection{Apache}
		\subsection{Form, Checkbox, Radiobox, Inputs}
		\subsection{Workspace}
	
	%%
	%% ############# technischer Aufbau
	%%
	\section{technischer Aufbau}
	
	%%
	%% ############# Komponenten
	%%
	\section{Komponenten}
		\subsection{\mi{Raspberry Pi}}
		\subsection{Hardware}
