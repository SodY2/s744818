\chapter{Grundlagen}
In diesem Abschnitt behandle ich spezifische Definitionen wie zum Beispiel verwendetes Fachvokabular, allgemeine technische Abläufe die Notwendig sind um diese Arbeit und die darin verwendetet Techniken zu verstehen, sowie verwendete Hardwarekomponenten.
%\tr{Dieser Teil beschreibt das fachliche Umfeld der Aufgabenstellung. Hier werden die wesentlichen fachlichen Begrifflichkeiten, die für die Aufgabe relevanten Problemstellungen und Lösungsansätze des Fachgebietes vorgestellt. Der Sprachgebrauch sollte einen direkten Bezug zum Fachgebiet haben. Die notwendigen Darstellungsmethoden, die Art und der Umfang der Beschreibung hängen wesentlich von der jeweiligen Fachdisziplin ab und sollten im Dialog mit dem Betreuer entschieden werden. Beispielsweise wird sich die Beschreibung eines Hotelreservierungssystems sehr von einer Beschreibung mathematischen Transformationen auf Grafikobjekte unterscheiden.Dies ist oft vor der Einleitung das erste Kapitel, das man schreibt, und sollte einen Überblick über die Literatur und existierende Arbeiten im Bereich der Arbeit liefern (Welche Grundlagen gibt es in diesem Bereich? Haben andere Autoren schon etwas zu verwandten Themen veröffentlicht?). Die hier vorgestellten Konzepte sollten in der Einleitung zumindest schon einmal angesprochen worden sein. Bei der Vorstellung verwandter Arbeiten sollten neuere Erkenntnisse bevorzugt werden. Bei jeder in das Grundlagenkapitel aufgenommenen Arbeit gilt es herauszustellen, was die Ergebnisse der Arbeit waren und warum diese Ergebnisse (oder Einschränkungen der vorgestellten Arbeit) eben noch keine Schließung der Wissenslücke oder keine Lösung der Aufgabenstellung darstellen? Am Ende erfolgt eine kurze Zusammenfassung der Grundlagen, die begründete Schlussfolgerung, dass das zu untersuchende Problem noch ungelöst ist, und ggf. wieder eine Vorschau auf das folgende Kapitel.}

	%%
	%% ############# Begriffsklärung
	%%
	\section{Begriffsklärung}	
		\Gls{parallel-synchron}
		\Gls{Web-Applikation}

		\Gls{Computer}

		\Gls{Ajax}
		\Gls{moEn}
		\Gls{Framework}
		\Gls{Webbrowser}
		\Gls{HTML}
		\Gls{Javascript}
		
		\Gls{NodeJS}
		\Gls{PHP}
		\Gls{NPM}
		\Gls{qs}
		\Gls{VirtualBox}
		
		\Gls{Smartphone}
		\Gls{Tablet}
		\Gls{PoW}
		\Gls{PaW}
		\Gls{Pixel}
		\Gls{BA}
		
		\Gls{Viewport}
		\Gls{Event}
		\Gls{DOM}
		\Gls{Apache}
		\Gls{Form, Checkbox, Radiobox, Inputs, Anker, Zertifizierung}
		\Gls{Workspace}
		\Gls{Commit}
		\Gls{Deamon}
		\Gls{App}
		\Gls{htaccess}
		\Gls{GNU General Public License (GPL) Lizenz}
		\Gls{MIT Lizenz}
		\Gls{Apache Lizenz 2}
		\Gls{Cloud-Dienst}
		\Gls{iFrame}
		\Gls{Grunt}
		\Gls{Test/Testsuite}
	
	%%
	%% ############# verwendete Hardware
	%%
	\pagebreak
	\section{verwendete Hardware}
	Alle in dieser Thesis aufgeführten Tests wurden mit den nachfolgenden Geräten und Umgebungen ausgeführt und validiert.
	
	\subsubsection{Apple iMac 27\texttt\dq}
	Zur Durchführung dieser Arbeit und der darin enthaltenen Evaluationsverfahren wurde ein Apple iMac mit folgenden 				Spezifikationen genutzt.
	
	\begin{table}[H]
	 \vspace{-20pt}
 		\centering
		\rowcolors{1}{white}{lightgray}
			\begin{tabular}{| p{4cm} | p{8cm}  |}
			\hline
				Prozessor			&	3,4GHz Intel Core i7 \\
				Speicher			&	8GB 1600Mhz DDR3\\
				Grafikkarte		&	NVIDIA GeForce GTX 675MX 1024 MB\\
				Betriebssystem		&	OS X 10.8.5 (12F45)\\

				\hline
				\end{tabular}
			\caption{verwendete Hardware}
	\end{table}

	\subsubsection{mobile Endgeräte}
	Die in dieser Arbeit durchgeführten Tests nutzen folgende Endgeräte. 
	
	\mmet{Nokia Lumina 920}{Windows Phone}{8.0}{11,4 cm (4,5 Zoll)}{768x1280}{Portrait}
	\mmet{LG Nexus 4}{Android}{4.4.2 (KitKat)}{11,9 cm (4,7 Zoll)}{768x1280}{Portrait}
	\mmet{Apple iPhone4 32 GB}{iOS}{6.1.3 (10B329)}{8,9 cm (3,5 Zoll)}{640 x 960}{Portrait}
	\mmet{Apple iPhone5s 16 GB}{iOS}{7.0.6 (11B651)}{10,2 cm (4,0 Zoll) }{640 x 1136}{Portrait}
	\mmet{Apple iPad mini Wi-Fi 32GB}{iOS}{7.0.4 (11B554a)}{20,1 cm (7,9 Zoll)}{1024 x 768}{Landschaft}
	\mmet{Microsoft Surfcae}{Windows}{8.1 Pro}{26,9 cm (10,6 Zoll)}{1920 x 1080}{Landschaft}
	
	%%
	%% ############# Komponenten
	%%
	\section{verwendete Software}
	
	\subsubsection{Virtuelle Maschine}
	\begin{table}[H]
	 \vspace{-20pt}
 		\centering
		\rowcolors{1}{white}{lightgray}
			\begin{tabular}{| p{4cm} | p{8cm}  |}
			\hline
				Hersteller			&	Oracle VM\\
				Product			&	VirtualBox\\
				Version			&	4.2.16 r86992\\
				Image			&	Windows Vista\\
				Virtueller Speicher	&	1024 MB\\
				\hline
				\end{tabular}
			\caption{verwendete virtuelle Maschine}
	\end{table}
	
	\subsubsection{verwendete Browser}	
	\begin{table}[H]
	 \vspace{-20pt}
 		\centering
		\rowcolors{1}{white}{lightgray}
			\begin{tabular}{| p{8cm} | p{4cm}  |}
			\hline
				Browser		 	&	Version	\\
			\hline

			\hline
				Google Chrome			&	34.0.1847.116\\
				Google Chrome (virtuell)		&	33.0.1750.149 m\\
				Mozilla Firefox				&	28.0\\
				Mozilla Firefox (virtuell)		&	25.0.1\\
				Opera					&	20.0\\
				Safari					&	6.1.3 (8537.75.14)\\
				Internet Explorer (virtuell)		&	9.0.8112.16421\\
				\hline
				\end{tabular}
			\caption{verwendete Browser}
	\end{table}
	
	
	
