\chapter{Grundlagen}
In diesem Abschnitt werden allgemeine technische Abläufe, die notwendig sind um diese Arbeit und die darin verwendetet Techniken zu verstehen, behandelt. Des Weiteren werden die verwendeten Hardwarekomponenten, sowie genutzte \Gls{Webbrowser} und Software aufgeführt.

	\section{Grundsätzlicher Aufbau}
	Alle in dieser Arbeit behandelten Softwareprodukte und \Gls{Framework}s benötigen entweder einen vorhandenen Webserver, wie z. B. \Gls{Apache}, um ihre Daten zu übermitteln, oder sie bringen einen softwareinternen Server mit sich. Als Vorraussetzung für eine funktionierende Kommunikation müssen die Testgeräte sich alle innerhalb eines gemeinsamen Netzwerkes befinden. Zum Arbeiten empfiehlt sich ausserdem eine durchgängige Stromzufuhr der verbundenen Geräte.
	
	
	\igp{../pictures/netzwerkaufbau}{Übersicht Netzarchitektur}{Übersicht Netzarchitektur}{400}{300}

	

	
	%%
	%% ############# verwendete Hardware
	%%
	\pagebreak
	\section{Verwendete Hardware}
	Alle in dieser Thesis aufgeführten Tests wurden mit den nachfolgenden Geräten und Umgebungen ausgeführt und validiert.
	
	\subsubsection{Apple iMac 27\texttt\dq}
	Zur Durchführung dieser Arbeit und der darin enthaltenen Evaluationsverfahren wurde ein Apple iMac mit folgenden 				Spezifikationen genutzt.
	
	\begin{table}[H]
	 \vspace{-20pt}
 		\centering
		\rowcolors{1}{white}{lightgray}
			\begin{tabular}{| p{4cm} | p{8cm}  |}
			\hline
				Prozessor			&	3,4GHz Intel Core i7 \\
				Speicher			&	8GB 1600Mhz DDR3\\
				Grafikkarte		&	NVIDIA GeForce GTX 675MX 1024 MB\\
				Betriebssystem		&	OS X 10.8.5 (12F45)\\

				\hline
				\end{tabular}
			\caption{verwendete Hardware}
	\end{table}

	\subsubsection{Mobile \Gls{moEn}e}
	Die in dieser Arbeit durchgeführten Tests nutzen folgende \Gls{moEn}e. 
	
	\mmet{Nokia Lumia 920}{Windows Phone}{8.0}{11,4 cm (4,5 Zoll)}{768x1280}{Portrait}
	\mmet{LG Nexus 4}{Android}{4.4.2 (KitKat)}{11,9 cm (4,7 Zoll)}{768x1280}{Portrait}
	\mmet{Apple iPhone4 32 GB}{iOS}{6.1.3 (10B329)}{8,9 cm (3,5 Zoll)}{640 x 960}{Portrait}
	\mmet{Apple iPhone5s 16 GB}{iOS}{7.0.6 (11B651)}{10,2 cm (4,0 Zoll) }{640 x 1136}{Portrait}
	\mmet{Apple iPad mini Wi-Fi 32GB}{iOS}{7.0.4 (11B554a)}{20,1 cm (7,9 Zoll)}{1024 x 768}{Landschaft}
	\mmet{Microsoft Surfcae}{Windows}{8.1 Pro}{26,9 cm (10,6 Zoll)}{1920 x 1080}{Landschaft}
	
	
	%%
	%% ############# Komponenten
	%%

	\section{Verwendete Software}
	
	\subsubsection{Virtuelle Maschine}
	\begin{table}[H]
	 \vspace{-20pt}
 		\centering
		\rowcolors{1}{white}{lightgray}
			\begin{tabular}{| p{4cm} | p{8cm}  |}
			\hline
				Hersteller			&	Oracle VM\\
				Product			&	VirtualBox\\
				Version			&	4.2.16 r86992\\
				Image			&	Windows Vista\\
				Virtueller Speicher	&	1024 MB\\
				\hline
				\end{tabular}
			\caption{verwendete virtuelle Maschine}
	\end{table}
	
	\subsubsection{Verwendete \Gls{Webbrowser}}	
	\begin{table}[H]
	 \vspace{-20pt}
 		\centering
		\rowcolors{1}{white}{lightgray}
			\begin{tabular}{| p{8cm} | p{4cm}  |}
			\hline
				Browser		 	&	Version	\\
			\hline

			\hline
				Google Chrome			&	34.0.1847.116\\
				Google Chrome (virtuell)		&	33.0.1750.149 m\\
				Mozilla Firefox				&	28.0\\
				Mozilla Firefox (virtuell)		&	25.0.1\\
				Opera					&	20.0\\
				Safari					&	6.1.3 (8537.75.14)\\
				Internet Explorer (virtuell)		&	9.0.8112.16421\\
				\hline
				\end{tabular}
			\caption{verwendete \Gls{Webbrowser}}
	\end{table}


	
	
	
