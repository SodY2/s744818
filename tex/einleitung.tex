\chapter{Einleitung}
\tr{Eine Einleitung bietet die Möglichkeiten den Sinn und Zweck der Diplomarbeit für einen Durchschnittsinformatiker (ohne die Spezialkenntnisse, die Sie jetzt haben) verständlich zu beschreiben. Hier können Sie Hintergründe darstellen, wie die Arbeit und das Thema entstanden und selbstverständlich für Ihre Arbeit werben. Interessierte Leser entscheiden hier, ob diese Arbeit für sie fachlich interessant ist.}

	\section{Einleitender Satz}
	\tr{	Eine kurze Beschreibung des allgemeinen Forschungsgebietes in ein bis zwei Absätzen. Die Einleitung sollte am Ende in ein bis zwei Sätzen die eigentlich untersuchte Fragestellung benennen.}

	\section{Hintergrund}
	\tr{Hier sollte der Hintergrund und die Motivation der Arbeit kurz angerissen werden. Nach Möglichkeit sollte man alle Aspekte, die im zweiten Kapitel ("Grundlagen") besprochen werden, hier schon einmal ansprechen, damit 		diese nicht später aus heiterem Himmel fallen. Insbesondere sollte der Hintergrund quasi "zwingend" den nächsten Teil der Einleitung motivieren:}
	
	\linebreak\linebreak
	Innerhalb des Entwicklungsprozesses einer mobilen Webanwendung durchläuft diese wiederholt die Qualitätssicherung und muss auf verschiedenen Geräten getestet werden. Diese werden in vielerlei Auflösungen mit 		unterschiedlichen Betriebssystemen (hier exemplarisch Android, iOS und Windows) in verschiedenen Versionen ausgeliefert. Während des Vorgangs der Qualitätssicherung wird die mobile Anwendung in Bezug auf ihre 		Seitennavigation, die erfolgreiche Umsetzung von HCI- Kriterien1 und ihre erfolgreiche Funktionsweise getestet. Erst nach erfolgreicher Freigabe durch die Qualitätssicherung darf der Entwicklungsprozess abgeschlossen 		werden.
