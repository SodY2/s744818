\chapter{Auswahl der \Gls{Framework}s}

\section{\Gls{Framework}s als Komplettlösung}
Unter diesem Punkt werden alle \Gls{Framework}s gelistet, welche damit werben das \Gls{parallel-synchron}e Testen von mobilen Seiten zu ermöglichen und somit den Entwicklungsprozess zu optimieren. Im Gegensatz zu den Einzelkomponenten, sollten diese im Idealfall keine weiteren Technologien benötigen um genutzt zu werden.
	
	\subsection{\mi{Ghostlab}}
	Ghostlab ist ein \Gls{Framework} des schweizer Unternehmens Vanamco. Es verspricht das synchrone Testen von Webseiten in Echtzeit. Weiterhin wirbt das Unternehmen mit einem umfangreichen Repertoire an nützlichen Fähigkeiten. Der Funktionsumfang umfasst das Scrollen innerhalb einer Seite, das Ausfüllen von  Formularen, das Wahrnehmen und Reproduzieren von Click-Events sowie dem Neuladen einer Seite. Ghostlab soll ebenso einen Inspektor besitzen, welcher die Analyse des \Gls{DOM}, der 'on the fly'\footnote{Anpassungen in Echtzeit, ohne Neuladen der Seite} Bearbeitung von CSS und der Analyse und Bearbeitung von \Gls{Javascript}dateien. Das \Gls{Framework} gibt an, für alle folgenden \Gls{Webbrowser} zu funktionieren, ohne diese konfigurieren zu müssen:

	\begin{table}[H]
 		\centering
		\rowcolors{1}{white}{lightgray}
			\begin{tabular}{| p{5cm} | p{5cm} |}
			
			\hline
				Browser 	& 	Version\\
			\hline
			\hline
				Firefox	&	latest\\
				Chrome	&	latest\\
				Safari	&	latest\\
				Internet Explorer	&	8/9/10\\
				Opera	&	11\\
				Opera Mobile	&	supportet\\
				Firefox Mobile	&	supportet\\
				Blackberry	&	supportet\\
				Windows Phone	&	supportet\\
				Safari mobile	&	supportet\\	
				Android	&	2.3 - 4.2\\
				\hline
				\end{tabular}
			\caption{von Ghostlab getestete \Gls{Webbrowser} (Stand 10.03.2014, Version 1.2.3)}
	\end{table}

	Der Kostenpunkt der Lizenz liegt bei Erstellung dieser Arbeit bei 49\$ (entspricht in etwa 35,30 € Stand: 20.3.2014). Zur Erstellung dieser Thesis wurde die 7-Tage-Testvollversion genutzt.
	
	\subsection{\mi{Adobe Edge Inspect}}
	Die Anwendung Edge Inspect stammt von Adobe und wird derzeit in der CC\footnote{Creative \Gls{Cloud}} Version vertrieben. Um Adobe Edge Inspect nutzen zu können, bedarf es drei separate Komponenten. Adobe wirbt mit synchronem Aufrufen und Auffrischen von Websites, sowie deren Inspizierung per weinre. Besonders angepriesen wird von Adobe die Nutzung und Verwendung der Adobe Edge Inspect API, welche auf GitHub zur Verfügung gestellt wird. Des Weiteren kann Adobe Edge Inspect in andere Edge Produkte\footnote{zum Beispiel Edge Reflow CC und Edge Code CC} integriert werden. 
	
	\\Adobe Edge Inspect CC steht 30 Tage kostenlos zum Testen bereit. Danach fallen ab 24,59 / Monat für die Nutzung des Einzelprodukt-Abos an.
	
	\\Die Anwendung läuft nur auf mobilen \Gls{moEn}en mit iOS oder Android Betriebssystem.
	
	\subsection{\mi{Remote Preview}}
	Remote Preview ist ein kleines \Gls{Javascript}framework von dem Web Designer Viljami Salminen aus Helsinki, Finnland. Es überprüft alle 1100ms per \Gls{Ajax}-Request ob sich die Quell-URL geändert hat und teilt dies dann den verbundenen Testgeräten mit. Er wirbt mit dem synchronen Aufruf von Webseiten auf einer Vielzahl von Plattformen: 
	
	\begin{table}[H]
 		\centering
		\rowcolors{1}{white}{lightgray}
			\begin{tabular}{| p{13cm} |}
			
			\hline
				Plattform\\
			\hline
			\hline
				Android OS 2.1 - 4.1.2 (Default browser + Chrome)\\
				Blackberry OS 7.0 (Default browser)\\
				iOS 4.2.1 - 6 (Default browser)\\
				Mac OS X (Safari, Chrome, Firefox, Opera)\\
				Maemo 5.0 (Default browser)\\
				Meego 1.2 (Default browser)\\
				Symbian 3 (Default browser)\\
				Symbian Belle (Default browser)\
				WebOS 3.0.5 (Default browser)\\
				Windows Phone 7.5 (Default browser)\\	
				Windows 7 (IE9)\\
				\hline
				\end{tabular}
			\caption{von Remote Preview unterstützte Plattformen (stand 19.03.2014, letzter \Gls{Commit} 7dc48caa84)}
	\end{table}
	Das \Gls{Framework} ist Kostenlos erhältlich und läuft unter der \Gls{MIT}. Zum Zeitpunkt dieser Arbeit scheint das Projekt nicht 		weiter entwickelt zu werden, da seit 5 Monaten auf der Projektseite keinerlei Aktualisierungen vorgenommen wurden.

		
	\subsection{\mi{Browser-Sync}}
	Browser-Sync wurde von Shane Osbourne entwickelt und soll im Zuge dieser Arbeit den Ansprüchen zur Verbesserung der Qualität gerecht werden. Es wirbt mit synchronisierter Steuerung, dem Entwickeln an CSS Styles und anderen Projektdateien in Echtzeit, der Installation unter Windows, MacOS und Linux und einer umfangreichen Palette an unterstützen Plattformen. Jedoch unterstützt das \Gls{Framework} im Gegensatz zu Ghostlab oder Adobe Edge Inspect keine Remoteinspection des \Gls{DOM} und den Netzwerkaktivitäten.
	
		\begin{table}[H]
 		\centering
		\rowcolors{1}{white}{lightgray}
			\begin{tabular}{| p{5cm} | p{5cm} |}
			
			\hline
				Browser 	& 	Version\\
			\hline
			\hline
				Firefox	&	latest\\
				Chrome	&	latest\\
				Safari	&	latest\\
				Internet Explorer	&	7/8/9/10\\
				Opera	&	latest\\
				Opera Mobile	&	supportet\\
				Firefox Mobile	&	supportet\\
				Blackberry	&	supportet\\
				Windows Phone	&	supportet\\
				Safari mobile	&	supportet\\	
				Android	&	supportet\\
				iOS		&	supportet\\
				\hline
				\end{tabular}
			\caption{von Browser-Sync getestete \Gls{Webbrowser} (Stand 21.03.2014, Version 0.7.2)}
	\end{table}
	
	Das \Gls{Framework} basiert auf dem \Gls{NodeJS} \Gls{Framework} und besitzt dadurch ein hohes Erweiterungspotential. Eine parallel zu Browser-Sync entwickelte Erweiterung kombiniert Browser-Sync mit \Gls{Grunt}, was automatisierte Abläufe ermöglicht. Diese fördert die Produktivität durch das Einbinden des \Gls{Framework}s in bestehende Arbeitsabläufe. Die Software ist kostenlos erhältlich und steht unter der \Gls{MIT}. Das Projekt befindet sich zum Zeitpunkt dieser Arbeit in der Version 0.7.2 und wird täglich weiterentwickelt.

\pagebreak
\section{\Gls{Framework}s als Teilkomponente}
	Als Teilkomponenten werden hier \Gls{Framework}s spezifiziert, welche dazu beitragen, das in der Thesis geforderte Werkzeug selbst zu entwickeln. Diese decken verschiedene spezielle Funktionen ab, wie zum Beispiel die Clientverwaltung, Steuerbefehle oder setzen die Voraussetzung für eigene \Gls{Test}szenarien.
	
	\subsection{\mi{NodeJS}}
	Node.JS Aufgabe besteht darin, anstelle von zum Beispiel \Gls{Apache}, einen Webserver zur Verfügung zu stellen, welcher nur auf \Gls{Javascript} basiert. Alle notwendigen serverseitigen Anfragen und Funktionen erfolgen in \Gls{Javascript}. Entwickelt wird Node.JS von der Kalifornischen Firma Joyent und befindet sich derzeit in Version 0.10.26. Geführt wird Node.JS unter der \Gls{MIT} und steht kostenlos auf nodejs.org oder unter GitHub zum Download bereit.
	
	\subsection{NPM \mi{socket.io}}
	socket.io ist ein \Gls{Framework} welches die WebSocket Technologie aktueller \Gls{Webbrowser} auf \Gls{Javascript} Ebene abbildet. Der Gedanke der Technologie dahinter verfolgt den Ansatz nicht in regelmäßigen Abständen Anfragen an den Server zu stellen und damit unnötig viel Datenvolumen zu generieren, sondern eine permanente Verbindung zum Server aufrecht zu halten um auf Statusänderungen am Server zu reagieren. socket.io wurde von Guillermo Rauch unter der \Gls{MIT} entwickelt und steht derzeit in der Version 0.9.16 auf GitHub oder per \Gls{NPM} zur Verfügung.
	
	\subsection{\mi{Zombie.js}}
	Das \Gls{Framework} Zombie.js ist ein Open-Source Projekt einer ganzen Gruppe von Entwicklern\footnote{https://github.com/assaf/	zombie/graphs/contributors}, welches von dem in Kalifornien sitzenden Assaf Arkin ins Leben gerufen wurde. Zombie.js wirbt mit 	seiner Einfachheit \Gls{Test}s zu erstellen und in \Gls{Test}suiten zu integrieren. Zombie.js emuliert einen sogenannten headless\footnote{Kopflos - ohne Gerüst das ihn umschließt, oder auch virtuell} \Gls{Webbrowser}. Dies hat zur Folge, dass natürlich nur nonvisuelle Aspekte in \Gls{Test}s integriert werden können, wie etwa das Ausfüllen von Formularen, das Navigieren durch den Navigationsbaum oder das Testen von Links.

	\subsection{\mi{Phantom Limb}}
	Phantom Limb ist ein von Brian Carstensen entwickeltes Werkzeug welches es ermöglichen soll, die \Gls{Computer}maus generierten Bewegungen in äquivalente \mbox{Touchevents} umwandelt. Das \Gls{Framework} läuft unter der \Gls{AL2} und kann kostenlos 	verwendet werden. Es kam in die Auswahl der \Gls{Framework}s, da seine Funktionalität zur Generierung von Steuerbefehlen geeignet ist.
	
	\subsection{\mi{jQuery UI Touch Punch}}
	jQuery UI Touch Punch ist eine Erweiterung zu der UI Bibliothek von jQuery die David Furfero entwickelt hat. Diese erlaubt von 	Hause aus nicht die Nutzung von Touchevents auf mobilen \Gls{moEn}en. Die Erweiterung hebt diese Restriktion auf, ohne weiter konfiguriert werden zu müssen. An Quellcode kommen lediglich weitere 584 Bytes hinzu. Die \Gls{Framework}erweiterung läuft unter der MIT und der \Gls{GPL}, wodurch es dem Endnutzer frei steht die Bibliothek unter den Lizenzen des eigenen Projektes zu verwenden.
	
	\subsection{\mi{jQuery Touchit}}
	Das von Daniel Glyde entwickelte \Gls{Framework} jQuery Touchit, wandelt Berührungen in äquivalente Mousevents um und ermittelt deren relative Position in Bezug auf den \Gls{Viewport}. Des Weiteren löst es das Problem bei bereits bestehenden jQuery Anwendungen und deren Darstellung auf mobilen \Gls{moEn}en, wo verschiedene Funktionalitäten, wie zum Beispiel die Verwendung von Slidern, nicht nutzbar sind.
	
	
	
	